\documentclass[x11names,reqno,14pt]{extarticle}
\input{preamble}
\DeclareMathOperator{\cl}{cl}
\usepackage[document]{ragged2e}
\usepackage{amsmath}
\pagestyle{fancy}{
	\fancyhead[L]{Winter 2023}
	\fancyhead[C]{221B - Cell Complexes}
	\fancyhead[R]{John White}
  
  \fancyfoot[R]{\footnotesize Page \thepage \ of \pageref{LastPage}}
	\fancyfoot[C]{}
	}
\fancypagestyle{firststyle}{
     \fancyhead[L]{}
     \fancyhead[R]{}
     \fancyhead[C]{}
     \renewcommand{\headrulewidth}{0pt}
	\fancyfoot[R]{\footnotesize Page \thepage \ of \pageref{LastPage}}
}

\title{221B - Cell Complexes}
\author{John White}
\date{Winter 2023}


\begin{document}

\section*{Lecture 1}

We will be using Hatcher's Algebraic Topology. The topology sequence is usually something like 
\begin{align*}
A & \text{Topological Spaces} \\
B & \text{Cell Complexes} \\
C & \text{Manifolds} \\
\end{align*}

\thm (BIG Theorem)

Given a ``reasonably nice" space, there is a bijection between connected covers of a space and subgroups of the fundamental group. 

\subsection*{\underline{Categories:}}

Algebraic structures that are much flabbier than a group. They consist of 
\begin{itemize}
\item A collection of arrows
\item A partial binary operation on these arrows
\item Objects, which arrows go between
\end{itemize}
We also want a composition law. That is, for objects and arrows
\[
\begin{tikzcd}
A \ar[r, "f"] & B \ar[r, "g"] & C 
\end{tikzcd}
\]
there is an arrow $\begin{tikzcd} A \ar[r, "g \circ f"]&  C \end{tikzcd}$. We want this composition to be associative, that is $(f \circ g) \circ h = f \circ (g \circ h)$, and we want objects to have identity arrows. 

Not all functions have inverses. Using sets and functions as an example, we have described the category Set. 

Here are some more examples of categories: 
\exm

\begin{itemize}
\item Groups and group homomorphisms (Grp)
\item Topological spaces and continuous functions (Top)
\item etc. 
\end{itemize}

We can make the following new category. 

\defn

We denote by $\Top^*$ the \underline{category of based topological spaces}, whose objects are pairs $(X, x_0)$, where $X$ is a topological space and $x_0 \in X$, and whose morphisms are continuous functions $f:(X,x_0)\to(Y, y_0)$ such that $f(x_0) = y_0$. 

\subsection*{Goal:}

Our goal is to get a functor from Top to Grp. The fundamental group functor $\pi_1$ will go from $\Top^*$ to Grp.

\section*{Lecture 2}

\subsection*{\underline{Topology review:}}

\defn 

A \underline{topological space} is a set $X$ along with a collection of subsets of $X$ called ``open sets," such that $X, \varnothing$ are open, and the arbitrary union and finite intersection of open sets are open. 

Notice the following diagram commutes using the product topology

\[
\begin{tikzcd}
&\ar[ld, "f"'] \ar[d, dotted, "\exists !"]Z \ar[rd, "g"] &  \\
X  &\ar[l, "P_X"'] X\times Y \ar[r, "P_Y"] & Y\\
\end{tikzcd}
\]

And in general 

\[
\begin{tikzcd}
Z \ar[d, dotted, "\exists !"']  \ar[rd, "f_\alpha"] & \\
\prod_{\alpha\in A}X_\alpha \ar[r, "P_\alpha"]& X_\alpha\\
\end{tikzcd}
\]
Maps are continuous; functions are not. 

\lem (Gluing lemma)

Suppose $f:A\to Y$, $g:B\to Y$ are continuous, and $f(x) = g(x)$ for all $x \in A \cap B$. Then $f\cup g: A \cup B \to Y$ is continuous. This only holds as long as $A, B \subseteq X$ are closed. 

\subsection*{\underline{Same Shape, Same Map}}
(maps up to wriggling things around a bit)

\defn Two maps are \underline{homotopic} if there exists a parametrized map $f_t:X\to Y$ such that $f_0 = f, f_1 = g$ for $f, g:X \to Y$. Equivalently, and more precisely, if there exists a map $F:X\times[0,1]\to Y$ such that $F(x, 0) = f(x), F(x, 1) = g(x)$ for all $x \in X$. 

$X, Y$ topological spaces are said to have the same shape if there exist maps $f:X\to Y, g:Y\to X$ such that $g\circ f\simeq \Id_X$ and $f\circ g \simeq \Id_Y$. We may say that $X, Y$ have the same \underline{homotopy type}

\defn

A \underline{deformation retraction} from $X \to A\subseteq X$ is a map from $X \times I \to X$ such that, for all $x \in A$, and $s, t \in I$,
\begin{align*}
f_0(x) &= x \\
f_1(x) &\in A \\
f_t(x) &= f_s(x) \\
\end{align*}







\end{document}