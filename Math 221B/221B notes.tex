\documentclass[x11names,reqno,14pt]{extarticle}
% Choomno Moos
% Portland State University
% Choom@pdx.edu


%% stupid experiment %%
%%%%%%%%%%%%% PACKAGES %%%%%%%%%%%%%

%%%% SYMBOLS AND MATH %%%%
\let\oldvec\vec
\usepackage{authblk}	% author block customization
\usepackage{microtype}	% makes stuff look real nice
\usepackage{amssymb} 	% math symbols
\usepackage{siunitx} 	% for SI units, and the degree symbol
\usepackage{mathrsfs}	% provides script fonts like mathscr
\usepackage{mathtools}	% extension to amsmath, also loads amsmath
\usepackage{esint}		% extended set of integrals
\mathtoolsset{showonlyrefs} % equation numbers only shown when referenced
\usepackage{amsthm}		% theorem environments
\usepackage{relsize}	%font size commands
\usepackage{bm}			% provides bold math
\usepackage{bbm}		% for blackboard bold 1

%%%% FIGURES %%%%
\usepackage{graphicx} % for including pictures
\usepackage{float} % allows [H] option on figures, so that they appear where they are typed in code
\usepackage{caption}
\usepackage{hyperref}
%\usepackage{titling}
\usepackage{tikz} % for drawing
\usetikzlibrary{shapes,arrows,chains,positioning,cd,decorations.pathreplacing,decorations.markings,hobby,knots,braids}
\usepackage{subcaption}	% subfigure environment in figures

%%%% MISC %%%%
\usepackage{enumitem} % for lists and itemizations
\setlist[enumerate]{leftmargin=*,label=\bf \arabic*.}

\usepackage{multicol}
\usepackage{multirow}
\usepackage{url}
\usepackage[symbol]{footmisc}
\renewcommand{\thefootnote}{\fnsymbol{footnote}}
\usepackage{lastpage} % provides the total number of pages for the "X of LastPage" page numbering
\usepackage{fancyhdr}
\usepackage{manfnt}
\usepackage{nicefrac}
%\usepackage{fontspec}
%\usepackage{polyglossia}
%\setmainlanguage{english}
%\setotherlanguages{khmer}
%\newfontfamily\khmerfont[Script=Khmer]{Khmer Busra}

%%% Khmer script commands for math %%%
%\newcommand{\ka}{\text{\textkhmer{ក}}}
%\newcommand{\ko}{\text{\textkhmer{ត}}}
%\newcommand{\kha}{\text{\textkhmer{ខ}}}

%\usepackage[
%backend=biber,
% numeric
%style=numeric,
% APA
%bibstyle=apa,
%citestyle=authoryear,
%]{biblatex}

\usepackage[explicit]{titlesec}
%%%%%%%% SOME CODE FOR REDECLARING %%%%%%%%%%

\makeatletter
\newcommand\RedeclareMathOperator{%
	\@ifstar{\def\rmo@s{m}\rmo@redeclare}{\def\rmo@s{o}\rmo@redeclare}%
}
% this is taken from \renew@command
\newcommand\rmo@redeclare[2]{%
	\begingroup \escapechar\m@ne\xdef\@gtempa{{\string#1}}\endgroup
	\expandafter\@ifundefined\@gtempa
	{\@latex@error{\noexpand#1undefined}\@ehc}%
	\relax
	\expandafter\rmo@declmathop\rmo@s{#1}{#2}}
% This is just \@declmathop without \@ifdefinable
\newcommand\rmo@declmathop[3]{%
	\DeclareRobustCommand{#2}{\qopname\newmcodes@#1{#3}}%
}
\@onlypreamble\RedeclareMathOperator
\makeatother

\makeatletter
\newcommand*{\relrelbarsep}{.386ex}
\newcommand*{\relrelbar}{%
	\mathrel{%
		\mathpalette\@relrelbar\relrelbarsep
	}%
}
\newcommand*{\@relrelbar}[2]{%
	\raise#2\hbox to 0pt{$\m@th#1\relbar$\hss}%
	\lower#2\hbox{$\m@th#1\relbar$}%
}
\providecommand*{\rightrightarrowsfill@}{%
	\arrowfill@\relrelbar\relrelbar\rightrightarrows
}
\providecommand*{\leftleftarrowsfill@}{%
	\arrowfill@\leftleftarrows\relrelbar\relrelbar
}
\providecommand*{\xrightrightarrows}[2][]{%
	\ext@arrow 0359\rightrightarrowsfill@{#1}{#2}%
}
\providecommand*{\xleftleftarrows}[2][]{%
	\ext@arrow 3095\leftleftarrowsfill@{#1}{#2}%
}
\makeatother

%%%%%%%% NEW COMMANDS %%%%%%%%%%

% settings
\newcommand{\N}{\mathbb{N}}                     	% Natural numbers
\newcommand{\Z}{\mathbb{Z}}                     	% Integers
\newcommand{\Q}{\mathbb{Q}}                     	% Rationals
\newcommand{\R}{\mathbb{R}}                     	% Reals
\newcommand{\C}{\mathbb{C}}                     	% Complex numbers
\newcommand{\K}{\mathbb{K}}							% Scalars
\newcommand{\F}{\mathbb{F}}                     	% Arbitrary Field
\newcommand{\E}{\mathbb{E}}                     	% Euclidean topological space
\renewcommand{\H}{{\mathbb{H}}}                   	% Quaternions / Half space
\newcommand{\RP}{{\mathbb{RP}}}                       % Real projective space
\newcommand{\CP}{{\mathbb{CP}}}                       % Complex projective space
\newcommand{\Mat}{{\mathrm{Mat}}}						% Matrix ring
\newcommand{\M}{\mathcal{M}}
\newcommand{\GL}{{\mathrm{GL}}}
\newcommand{\SL}{{\mathrm{SL}}}

\newcommand{\tgl}{\mathfrak{gl}}
\newcommand{\tsl}{\mathfrak{sl}}                  % Lie algebras; i.e., tangent space of SO/SL/SU
\newcommand{\tso}{\mathfrak{so}}
\newcommand{\tsu}{\mathfrak{sl}}


% typography
\newcommand{\noi}{\noindent}						% Removes indent
\newcommand{\tbf}[1]{\textbf{#1}}					% Boldface
\newcommand{\mc}[1]{\mathcal{#1}}               	% Calligraphic
\newcommand{\ms}[1]{\mathscr{#1}}               	% Script
\newcommand{\mbb}[1]{\mathbb{#1}}               	% Blackboard bold


% (in)equalities
\newcommand{\eqdef}{\overset{\mathrm{def}}{=}}		% Definition equals
\newcommand{\sub}{\subseteq}						% Changes default symbol from proper to improper
\newcommand{\psub}{\subset}						% Preferred proper subset symbol

% Categories
\newcommand{\catname}[1]{{\text{\sffamily {#1}}}}

\newcommand{\Cat}{{\catname{C}}}
\newcommand{\cat}[1]{{\catname{\ifblank{#1}{C}{#1}}}}
\newcommand{\CAT}{{\catname{Cat}}}
\newcommand{\Set}{{\catname{Set}}}

\newcommand{\Top}{{\catname{Top}}}
\newcommand{\Met}{{\catname{Met}}}
\newcommand{\PL}{{\catname{PL}}}
\newcommand{\Man}{{\catname{Man}}}
\newcommand{\Diff}{{\catname{Diff}}}

\newcommand{\Grp}{{\catname{Grp}}}
\newcommand{\Grpd}{{\catname{Grpd}}}
\newcommand{\Ab}{{\catname{Ab}}}
\newcommand{\Ring}{{\catname{Ring}}}
\newcommand{\CRing}{{\catname{CRing}}}
\newcommand{\Mod}{{\mhyphen\catname{Mod}}}
\newcommand{\Alg}{{\mhyphen\catname{Alg}}}
\newcommand{\Field}{{\catname{Field}}}
\newcommand{\Vect}{{\catname{Vect}}}
\newcommand{\Hilb}{{\catname{Hilb}}}
\newcommand{\Ch}{{\catname{Ch}}}

\newcommand{\Hom}{{\mathrm{Hom}}}
\newcommand{\End}{{\mathrm{End}}}
\newcommand{\Aut}{{\mathrm{Aut}}}
\newcommand{\Obj}{{\mathrm{Obj}}}
\newcommand{\op}{{\mathrm{op}}}

% Norms, inner products
\delimitershortfall=-1sp
\newcommand{\widecdot}{\, \cdot \,}
\newcommand\emptyarg{{}\cdot{}}
\DeclarePairedDelimiterX{\norm}[1]{\Vert}{\Vert}{\ifblank{#1}{\emptyarg}{#1}}
\DeclarePairedDelimiterX{\abs}[1]\vert\vert{\ifblank{#1}{\emptyarg}{#1}}
\DeclarePairedDelimiterX\inn[1]\langle\rangle{\ifblank{#1}{\emptyarg,\emptyarg}{#1}}
\DeclarePairedDelimiterX\cur[1]\{\}{\ifblank{#1}{\emptyarg,\emptyarg}{#1}}
\DeclarePairedDelimiterX\pa[1](){\ifblank{#1}{\emptyarg}{#1}}
\DeclarePairedDelimiterX\brak[1][]{\ifblank{#1}{\emptyarg}{#1}}
\DeclarePairedDelimiterX{\an}[1]\langle\rangle{\ifblank{#1}{\emptyarg}{#1}}
\DeclarePairedDelimiterX{\bra}[1]\langle\vert{\ifblank{#1}{\emptyarg}{#1}}
\DeclarePairedDelimiterX{\ket}[1]\vert\rangle{\ifblank{#1}{\emptyarg}{#1}}

% mathmode text operators
\RedeclareMathOperator{\Re}{\operatorname{Re}}		% Real part
\RedeclareMathOperator{\Im}{\operatorname{Im}}		% Imaginary part
\DeclareMathOperator{\Stab}{\mathrm{Stab}}
\DeclareMathOperator{\Orb}{\mathrm{Orb}}
\DeclareMathOperator{\Id}{\mathrm{Id}}
\DeclareMathOperator{\vspan}{\mathrm{span}}			% Vector span
\DeclareMathOperator{\tr}{\mathrm{tr}}
\DeclareMathOperator{\adj}{\mathrm{adj}}
\DeclareMathOperator{\diag}{\mathrm{diag}}
\DeclareMathOperator{\eq}{\mathrm{eq}}
\DeclareMathOperator{\coeq}{\mathrm{coeq}}
\DeclareMathOperator{\coker}{\mathrm{coker}}
\DeclareMathOperator{\dom}{\mathrm{dom}}
\DeclareMathOperator{\cod}{\mathrm{codom}}
\DeclareMathOperator{\im}{\mathrm{im}}
\DeclareMathOperator{\Dim}{\mathrm{dim}}
\DeclareMathOperator{\codim}{\mathrm{codim}}
\DeclareMathOperator{\Sym}{\mathrm{Sym}}
\DeclareMathOperator{\lcm}{\mathrm{lcm}}
\DeclareMathOperator{\Inn}{\mathrm{Inn}}
\DeclareMathOperator{\sgn}{sgn}						% sgn operator
\DeclareMathOperator{\intr}{\text{int}}             % Interior
\DeclareMathOperator{\co}{\mathrm{co}}				% dual/convex Hull
\DeclareMathOperator{\Ann}{\mathrm{Ann}}
\DeclareMathOperator{\Tor}{\mathrm{Tor}}


% misc symbols
\newcommand{\divides}{\big\lvert}
\newcommand{\grad}{\nabla}
\newcommand{\veps}{\varepsilon}						% Preferred epsilon
\newcommand{\vphi}{\varphi}
\newcommand{\del}{\partial}							% Differential/Boundary
\renewcommand{\emptyset}{\text{\O}}					% Traditional emptyset symbol
\newcommand{\tril}{\triangleleft}					% Quandle operation
\newcommand{\nabt}{\widetilde{\nabla}}				% Contravariant derivative
\newcommand{\later}{$\textcolor{red}{\blacksquare}$}% Laziness indicator

% misc
\mathchardef\mhyphen="2D							% mathomode hyphen
\renewcommand{\mod}[1]{\ (\mathrm{mod}\ #1)}
\renewcommand{\bar}[1]{\overline{#1}}				% Closure/conjugate
\renewcommand\qedsymbol{$\blacksquare$} 			% Changes default qed in proof environment
%%%%% raised chi
\DeclareRobustCommand{\rchi}{{\mathpalette\irchi\relax}}
\newcommand{\irchi}[2]{\raisebox{\depth}{$#1\chi$}}
\newcommand\concat{+\kern-1.3ex+\kern0.8ex}

% Arrows
\newcommand{\weak}{\rightharpoonup}					% Weak convergence
\newcommand{\weakstar}{\overset{*}{\rightharpoonup}}% Weak-star convergence
\newcommand{\inclusion}{\hookrightarrow}			% Inclusion/injective map
\renewcommand{\natural}{\twoheadrightarrow}				% Natural map
\newcommand{\oo}{\infty}

% Environments
\theoremstyle{plain}
\newtheorem{thm}{Theorem}[section]
%\newtheorem{lem}[thm]{Lemma}
\newtheorem{lem}{Lemma}
\newtheorem*{lems}{Lemma}
\newtheorem{cor}[thm]{Corollary}
\newtheorem{prop}{Proposition}
\newtheorem*{claim}{Claim}
\newtheorem*{cors}{Corollary}
\newtheorem*{props}{Proposition}
\newtheorem*{conj}{Conjecture}

\theoremstyle{definition}
\newtheorem{defn}{Definition}[section]
\newtheorem*{defns}{Definition}
\newtheorem{exm}{Example}[section]
\newtheorem{exer}{Exercise}[section]

\theoremstyle{remark}
\newtheorem*{rem}{Remark}

\newtheorem*{solnx}{Solution}
\newenvironment{soln}
    {\pushQED{\qed}\renewcommand{\qedsymbol}{$\Diamond$}\solnx}
    {\popQED\endsolnx}%

% Macros

\newcommand{\restr}[1]{_{\mkern 1mu \vrule height 2ex\mkern2mu #1}}
\newcommand{\Upushout}[5]{
    \begin{tikzcd}[ampersand replacement = \&]
    \&#2\ar[rd,"\iota_{#2}"]\ar[rrd,bend left,"f"]\&\&\\
    #1\ar[ur,"#4"]\ar[dr,"#5"]\&\&#2\oplus_{#1} #3\ar[r,dashed,"\vphi"]\&Z\\
    \&#3\ar[ur,"\iota_{#3}"']\ar[rru,bend right,"g"']\&\&
    \end{tikzcd}
}
\newcommand{\exactshort}[5]{
		\begin{tikzcd}[ampersand replacement = \&]
			0\ar[r]\&#1\ar[r,"#2"]\& #3 \ar[r,"#4"]\& #5 \ar[r]\&0
		\end{tikzcd}
}
\newcommand{\product}[6]{
		\begin{tikzcd}[ampersand replacement = \&]
			#1 \& #2 \ar[l,"#4"'] \\
			#3 \ar[u,"#5"] \ar[ur,"#6"']
		\end{tikzcd}
}
\newcommand{\coproduct}[6]{
		\begin{tikzcd}[ampersand replacement = \&]
			#1 \ar[r,"#4"] \ar[d,"#5"'] \& #2 \ar[dl,"#6"] \\
			#3
		\end{tikzcd}
}
%%%%%%%%%%%% PAGE FORMATTING %%%%%%%%%

\usepackage{geometry}
    \geometry{
		left=15mm,
		right=15mm,
		top=15mm,
		bottom=15mm	
		}

\usepackage{color} % to do: change to xcolor
\usepackage{listings}
\lstset{
    basicstyle=\ttfamily,columns=fullflexible,keepspaces=true
}
\usepackage{setspace}
\usepackage{setspace}
\usepackage{mdframed}
\usepackage{booktabs}
\DeclareMathOperator{\cl}{cl}
\usepackage[document]{ragged2e}
\usepackage{amsmath}
\pagestyle{fancy}{
	\fancyhead[L]{Winter 2023}
	\fancyhead[C]{221B - Cell Complexes}
	\fancyhead[R]{John White}
  
  \fancyfoot[R]{\footnotesize Page \thepage \ of \pageref{LastPage}}
	\fancyfoot[C]{}
	}
\fancypagestyle{firststyle}{
     \fancyhead[L]{}
     \fancyhead[R]{}
     \fancyhead[C]{}
     \renewcommand{\headrulewidth}{0pt}
	\fancyfoot[R]{\footnotesize Page \thepage \ of \pageref{LastPage}}
}
\DeclareMathOperator{\rel}{rel}

\title{221B - Cell Complexes}
\author{John White}
\date{Winter 2023}


\begin{document}

\section*{Lecture 1}

We will be using Hatcher's Algebraic Topology. The topology sequence is usually something like 
\begin{align*}
A & \text{Topological Spaces} \\
B & \text{Cell Complexes} \\
C & \text{Manifolds} \\
\end{align*}

\thm (BIG Theorem)

Given a ``reasonably nice" space, there is a bijection between connected covers of a space and subgroups of the fundamental group. 

\subsection*{\underline{Categories:}}

Algebraic structures that are much flabbier than a group. They consist of 
\begin{itemize}
\item A collection of arrows
\item A partial binary operation on these arrows
\item Objects, which arrows go between
\end{itemize}
We also want a composition law. That is, for objects and arrows
\[
\begin{tikzcd}
A \ar[r, "f"] & B \ar[r, "g"] & C 
\end{tikzcd}
\]
there is an arrow $\begin{tikzcd} A \ar[r, "g \circ f"]&  C \end{tikzcd}$. We want this composition to be associative, that is $(f \circ g) \circ h = f \circ (g \circ h)$, and we want objects to have identity arrows. 

Not all functions have inverses. Using sets and functions as an example, we have described the category Set. 

Here are some more examples of categories: 
\exm

\begin{itemize}
\item Groups and group homomorphisms (Grp)
\item Topological spaces and continuous functions (Top)
\item etc. 
\end{itemize}

We can make the following new category. 

\defn

We denote by $\Top^*$ the \underline{category of based topological spaces}, whose objects are pairs $(X, x_0)$, where $X$ is a topological space and $x_0 \in X$, and whose morphisms are continuous functions $f:(X,x_0)\to(Y, y_0)$ such that $f(x_0) = y_0$. 

\subsection*{Goal:}

Our goal is to get a functor from Top to Grp. The fundamental group functor $\pi_1$ will go from $\Top^*$ to Grp.

\section*{Lecture 2}

\subsection*{\underline{Topology review:}}

\defn 

A \underline{topological space} is a set $X$ along with a collection of subsets of $X$ called ``open sets," such that $X, \varnothing$ are open, and the arbitrary union and finite intersection of open sets are open. 

Notice the following diagram commutes using the product topology

\[
\begin{tikzcd}
&\ar[ld, "f"'] \ar[d, dotted, "\exists !"]Z \ar[rd, "g"] &  \\
X  &\ar[l, "P_X"'] X\times Y \ar[r, "P_Y"] & Y\\
\end{tikzcd}
\]

And in general 

\[
\begin{tikzcd}
Z \ar[d, dotted, "\exists !"']  \ar[rd, "f_\alpha"] & \\
\prod_{\alpha\in A}X_\alpha \ar[r, "P_\alpha"]& X_\alpha\\
\end{tikzcd}
\]
Maps are continuous; functions are not. 

\lem (Gluing lemma)

Suppose $f:A\to Y$, $g:B\to Y$ are continuous, and $f(x) = g(x)$ for all $x \in A \cap B$. Then $f\cup g: A \cup B \to Y$ is continuous. This only holds as long as $A, B \subseteq X$ are closed. 

\subsection*{\underline{Same Shape, Same Map}}
(maps up to wriggling things around a bit)

\defn Two maps are \underline{homotopic} if there exists a parametrized map $f_t:X\to Y$ such that $f_0 = f, f_1 = g$ for $f, g:X \to Y$. Equivalently, and more precisely, if there exists a map $F:X\times[0,1]\to Y$ such that $F(x, 0) = f(x), F(x, 1) = g(x)$ for all $x \in X$. 

$X, Y$ topological spaces are said to have the same shape if there exist maps $f:X\to Y, g:Y\to X$ such that $g\circ f\simeq \Id_X$ and $f\circ g \simeq \Id_Y$. We may say that $X, Y$ have the same \underline{homotopy type}

\defn

A \underline{deformation retraction} from $X \to A\subseteq X$ is a map from $X \times I \to X$ such that, for all $x \in A$, and $s, t \in I$,
\begin{align*}
f_0(x) &= x&\forall x \in X \\
f_1(x) &\in A&\forall x \in X \\
f_t(x) &= f_s(x)& \forall x \in A \\
\end{align*}

\section*{Lecture 3, 1/13/23}

\defn

Let $X$ be a topological space. A \underline{retraction} is a map $r:X\to X$ such that $r \circ r = r$. That is, $r(r(x)) = r(x)$ for any $x \in X$. 

Let $A = r(X)$. Then $r|_A = \Id_A$. 

\defn

Let $F:X\times I\to Y$. We say $f_0 \simeq f_1 \rel A\subseteq X$ are \underline{homotopic relative to $A$} if, for any $x \in A$, $f_t(x)$ is independent of $t$. That is, for any $s, t \in I$, $f_s(x) = f_t(x)$ for any $x \in A$. 


For any map $f:X\to Y$, there exists a space $Z \simeq Y$ via $g:Y \to Z$ such that $g \circ f:X\to Z$ is injective. That is, in the following diagram, we have a bijection between homotopy classes of maps $f$ and homotopy classes of maps $g\circ f$, and we can do this in a way that rigs $g \circ f$ to be injective. 
\[
\begin{tikzcd}
X \ar[r, "f"] \ar[rd, "g\circ f"'] & Y \ar[d, "g"] \\
& Z \\
\end{tikzcd}
\]

\defn

Given a map $f:X\to Y$ we can construct the \underline{Mapping Cylinder} $M_f$ by setting $M_f = X \times I \coprod Y/\sim$, where $(x, 0) \sim f(x)$. 

The visual intuition should be taking the disjoint union of $X$ and $Y$, and tieing a string between $x$ and $f(x)$ for each point. 

\claim 
$X \hookrightarrow M_f, Y \hookrightarrow M_f$, and the latter is in fact a homotopy equivalence. Further, the injection $X \hookrightarrow M_f$ is homotopic to $f(X) \hookrightarrow M_f$. 
\proof
You can construct a homotopy which ``squishes" the cylinder down to $f(X)$. 

\qed

\defn

A space $X$ is \underline{contractible} if it has the homotopy type of a point. A map is \underline{null-homotopic} if it is homotopic to a constant map. So $X$ is contractible if the identity is null-homotopic. 

Now he's drawing an example. The example is Bing's House with 2 rooms, which I will not reproduce here. But the point is that it's contractible, but not obviously so. 

\subsection*{\underline{Cell Complexes}}

Cell complexes are topological spaces which are built up inductively out of closed balls in Euclidean space. We write $\mathbb{D}^n := \{\vec{x} \in \R^n \mid \norm{\vec{x}} \leq 1 \}$, and $e^n := \{\vec{x}\in\R^n \mid \norm{\vec{x}} < 1\}$. We can see that $e^n = \operatorname{int}\mathbb{D}^n$, and $\del\mathbb{D}^n = \mathbb{S}^{n - 1}$. 

\subsubsection*{\underline{Base step}}

Start with some collection of points $X^0$, the $0$-skeleton, with the discrete topology.

\subsubsection*{\underline{Inductive step}}

Let $X^{n - 1}$ be the $n - 1$ skeleton, which has already been build and defined. Select some collection of $n$-dimensional balls $\{\mathbb{D}^n\}_{\alpha\in A}$, and some continuous ``attaching map" $\varphi_\alpha:\del\mathbb{D}^n_\alpha\to X^{n - 1}$. Then 
\[
(X^n = X^{n - 1}\coprod_{\alpha\in A}\mathbb{D}^n )/(x \sim\varphi_\alpha(x)\forall x \in \del\mathbb{D}^n)
\]

\section*{Lecture 4, 1/18/23}

A space $X$ is a \underline{cell complex} if it has been constructed using the above inductive procedure. If $n = \oo$, we use the weak topology, in which the open sets are the sets which are open when intersected with each $X^n$. 

For every $\mbb{D}^n_\alpha$ and corresponding ``attaching map $\varphi_\alpha:\del\mbb{D}^n_\alpha\to X^{n - 1}$, there is a subset of $X^n$ homeomorphic to $\operatorname{int}(\mbb{D}^n_\alpha)$, via the composition
\[
\operatorname{int}(\mbb{D}^n_\alpha) \hookrightarrow \mbb{D}^n_\alpha \hookrightarrow X^{n - 1}\coprod_{\alpha}\mbb{D}^n_\alpha \to X^n
\]
which we call $\Phi_\alpha:\mbb{D}^n_\alpha\to X^{n - 1}$. So the attaching map $\phi_\alpha:\del\mbb{D}^n_\alpha\to X^{n - 1}$ extends to a ``characteristic map" $\Phi_\alpha$.

We will now see many examples of things. 

\exm

If you stop after constructing $X^1$, it's a graph. 

\exm

$\mbb{S}^n$ has a cell structure with one $e_0$ and one $e_n$. 

\exm

Consider $\RP^2$. This can be expressed as $(\R^3\setminus\{0\})/(\vec{x}\sim\lambda\vec{x}, \lambda\neq0)$. We can replace $2$ with any $n$ and get $\RP^n$. Indeed, we can replace $\R$ with $\C$, $\mbb{H}$, or indeed any field. 

\underline{Homogenous coordinates}

For $(x, y, z)\neq(0,0,0)$, we have $[x, y, z]\eqdef\{(\lambda x, \lambda y, \lambda z) \mid \lambda \neq0\}$. For example, $[1,2,3]=[2,4,6]$. 

\section*{Lecture 5, 1/20/23}

\defn

A \underline{subcomplex} of a complex $X$ is a closed disjoint union of open cells $e^{n_i}_{\alpha_i}$ in $X$ such that they form a cell complex on their own. 

\rem We keep talking about ``CW Complexes." The C is for ``closure finite," and the W is for "weak topology." 

Recall: $\RP^n \eqdef \R^n/(x\sim\lambda x, \lambda\neq0)$. $\CP^n$ can be defined similarly. 

We can write $\RP^n = e^0 \cup e^1 \cup e^2 \cup \cdots \cup e^n$, and $\CP^n = e^0 \cup e^2 \cup e^3 \cup \cdots \cup e^{2n}$. We can do the same thing with the quaternions. 

Next time, we will cover operations on complexes. 

\section*{Lecture 6, 1/23/23}

This lecture, we will cover operations on cell complexes, and two big theorems.

\subsection*{\underline{Operations on Cell Complexes}}

\begin{enumerate}

\item If $X, Y$ have cell structures, then $X \times Y$ has a natural cell structure.

\item If $(X, A)$ is a CW-pair, then $X/A$ has a natural cell structure ($X/A$ denotes identifying all points in $A$ together). 
\[
\begin{tikzcd}
\mbb{D}'_\alpha \supseteq \del\mbb{D}'_\alpha \ar[d, "\phi_\alpha"] \ar[dr] & \\
X^0 \ar[r, "q"'] & (X/A)^0 \\
\end{tikzcd}
\] 

\item Cones and Suspensions. The cone on $X$, $CX$, is defined as
\[
CX = (X\times I)/(X\times\{0\})
\]
Note that $CX$ is contractible for any $X$. The suspension on $X$, $SX$, is defined as 
\[
SX = CX/(X\times\{1\})
\]

If $f:X\to Y$ is a map, there exists a natural map $Sf:SX\to SY$. Indeed, if $f:X\to Y$, then $f\times\Id:X\times I\to Y\times I$, and so we can factor $f \times \Id$ through the quotient map:
\[
\begin{tikzcd}
X\ar[r, "f"]\ar[d, hook]\ar[dr, dotted] & Y\ar[d, hook] \\
X\times I\ar[d,"q"]\ar[r, "f\times\Id"]\ar[dr, dotted] & Y\times I \ar[d] \\
SX\ar[r, "\exists!Sf"'] & SY \\
\end{tikzcd}
\]
Note $S(\mbb{S}^n) = \mbb{S}^{n + 1}$.

\item Joins. If $X, Y$ are cell structures, then we define their join $X\star Y$ as
\[
X\star Y = \frac{X\times Y \times I}{(x, y_1, 0) \sim (x, y_2, 0), (x_1, y, 1) \sim (x_2, y, 1)}
\]
This is a useful construction for simplices. 

\item Wedge product. If $X, Y$ are cell structures, with distinguished points $x_0 \in X, y_0 \in Y$, then we define their wedge product $X \wedge Y$ as
\[
X\wedge Y = \frac{X\coprod Y}{x_0 \sim y_0}
\]
This is just gluing $X$ and $Y$ together at a distinguished point. This raises an obvious question: does the wedge product depend on the points $x_0, y_0$? Yes, but not if they are (connected) cell complexes!

If $x_0$ is a 0-cell of $X$, and $y_0$ a 0-cell of $Y$, then $X \vee Y$ has a natural cell structure AND $X\vee Y$ is a subcomplex of $X\times Y$.

\item Smash product. If $X, Y$ are spaces with distinguished points $x_0, y_0$, then the smash product is defined as 
\[
X\wedge Y = \frac{X\times Y}{X\vee Y}
\]
For example, the smash product $S^1\wedge S^1$ is a Torus quotiented out by the longitudinal and meridian circles. By arguing from some cell nonsense, we can say this is $S^2$. 

\end{enumerate}

Here are two big theorems. 

\thm If $(X, A)$ is a CW-pair, and $A$ is contractible, then $X/A$ is homotopy equivalent to $X$, with the quotient mapping itself providing a homotopy equivalence. 

\thm Suppose $(X_1, A)$ is a CW-pair, and $f, g:A\to Y$ are maps. If $f \simeq g$, and everything in sight is a cell complex, then
\[
X_1\coprod_f Y \simeq X_1 \coprod_g Y
\]
That is, if $f, g$ are used as attaching maps, then the resulting spaces will be homotopy equivalent. 

\section*{Lecture 7, 1/25/23}

\defn

Let $X$ be a cell complex. If we let $f_i$ be the number of $i$-dimensional cells in the cell structure, then we define
\[
\chi(X) = f_0 - f_1 + f_2 - f_3 + \cdots
\]
The more general definition is the alternating sum of the Betti numbers of $X$, where the $i$th Betti number is $\dim H^i(X)$. 

\defn

Let $X$ be a topological space, and let $A\subseteq X$ be a subspace. We say that $(X, A)$ has the \underline{homotopy extension property} (HEP) if for all topological spaces $Y$ and for all maps $f:X\times\{0\} \cup A \times I \to Y$, there exists an extension of $f$, $\bar{f}:X\times I \to Y$, such that $\bar{f}|_{X\times\{0\}\cup A\times I} = f$. 

Slogan: ``A homotopy on the subspace can be extended to a homotopy on the entire space." 

\section*{Lecture 8, 1/27/23}

\prop

$(X, A)$ has the homotopy extension property if and only if $X \times I$ retracts to $X\times\{0\} \cup A \times I$.

\proof

\qed

\exm 

Does $(\mbb{D}^2, \del\mbb{D}^2)$ have the property? Does $\mbb{D}^2\times I$ retract onto $\mbb{D}^2\times\{0\}\cup \del\mbb{D}^2 \times I$? Yes. This is easy to see by drawing a picture. 

Here is a non-example. Let $X = I$, and let $A = \{\frac{1}{n}\}_{n\in\N}\cup\{0\}$. $X \times I$ is the square, and $X \times\{0\}$ is the bottom of the square, so $X\times \{0\} \cup A\times I$ is the comb space. The square doesn't retract to this. 

\prop 

If $(X, A)$ is a CW pair, then $(X, A)$ has the homotopy extension property. 

\proof

Later

\qed

\thm 

If $(X, A)$ has the homotopy extension property, and $A$ is contractible, then the quotient map $X \to X/A$ is a homotopy equivalence. 

\proof

Consider identity map $\Id:A\to A$. We have a homotopy $F:A\times I \to A$ which is a witness to $A$ being contractible. That is, $f_0 = \Id_A$, $f_1 \equiv \{p\}$ for some point $p \in A$. 

Then there is an extension to a homotopy $H:X\times I \to X$. We have the following commutative diagram
\[
\begin{tikzcd}
X\ar[r, "f_t"] \ar[d, "q"] & X \ar[d, "q"] \\
X/A \ar[r, "\bar{f_t}"] & X/A \\
\end{tikzcd}
\]
Because all of $A$ goes to a point for $t = 1$, then by the universal property of quotients, there is a map $g$ making the diagram commute:

\[
\begin{tikzcd}
X\ar[r, "f_1"] \ar[d, "q"] & X \ar[d, "q"] \\
X/A \ar[ur, "g"]\ar[r, "\bar{f_1}"'] & X/A \\
\end{tikzcd}
\]
So $qg$ is homotopic to the identity map, and $gq$ is homotopic to the identity map. This completes the proof? 

\qed

\section*{Lecture 9, 1/30/23}

\defn

We say that $(X, A)$ and $(Y, A)$ are homotopy equivalent relative to $A$ if there exist maps $f:X\to Y, g:Y\to X$ such that $f|_A = \Id_A, g|_A = \Id_A$ and $g \circ f \simeq \Id_X$ relative to $A$, and $f \circ g \simeq \Id_Y$ relative to $A$. 

\thm

If $(X, A)$ is a CW Pair, and $f, g:A\to X_0$ are homotopic maps, then $X_0 \coprod_fX_1 \simeq X_0\coprod_gX_1$ relative to $X_0$. 

\proof

Bunch of pictures I can't write down. 

\qed

\prop

If $(X, A), (Y, A)$ both have the homotopy extension property, and $f:X\to Y$ is a homotopy equivalence such that $f|_A = \Id_A$, then $f$ is a homotopy equivalence relative to $A$. 

\proof

\qed

\cor

If $(X, A)$ has the homotopy extension property and $A \hookrightarrow X$ is a homotopy equivalence, then $X$ deformation retracts to $A$. 

\proof

\qed

\cor

A map $f:X\to Y$ is a homotopy equivalence if and only if $X$ is a deformation retraction of $M_f$. 

\proof

\qed

\section*{Lecture 10, 2/3/23}

\defn 

Given any path $f:I\to X$, we write $[f]$ for the set of paths $g:I\to X$ such that $g \simeq f$ relative to $\del I$. If we don't fix endpoints, each $[f]$ would be a path-component. Sometimes we use $\pi^0$ to denote the set of path components. 

Let $f, g:I\to\R^n, f\simeq g$ by $h_t(u) = tg(u) + (1 - t)f(u)$. Then $h_0 = f, h)1 = g$. $h$ is called the ``straight line homotopy."

In fact, we could change $I$ to \underline{any} topological space, and $f \simeq g$ \underline{still} 

We could also change $\R^n$ to any $U\subseteq\R^n, U$ convex, and $f\simeq g$ \underline{still}. 

We could change $X$ to $U$, any metric space iwth unique shortest path which vary continuosly as the endpoints vary. 

Concatenation: If $f(1) = g(0)$, then define $f\star g:I\to X$ by 
\[
(f\star g)(t) = \begin{cases} f(2t) & t\in[0,\frac{1}{2}] \\ g(2t-1) & t \in [\frac{1}{2}, 1] \end{cases} 
\]

\defn

Assume $f(1) = g(0)$. Then $[f]\star[g] = [f\star g]$ is well defined by handwaving. 

Constants: The constant path $c_x$ is the path which is constantly $x$. Note $[c_{f(0)}]\star[f] = [f], [f]\star[c_{f(1)}] = [f]$. 

Inverses: Define $\bar{f}(u) = f(1 - u)$. Note $[f][\bar{f}] = [c_x]$. 

Associativity: $(f\star g)\star h \simeq f\star(g\star h)$. 

\defn A category where every $f:A\to B$ has an inverse (i.e. an arrow $f^{-1}:B\to A$ such that $ff^{-1} = \Id_B = f^{-1}f)$ is called a \underline{groupoid}. 

$\pi_0$ is a functor, (objects$\to$objects, arrow$\to$arrows, compositions$\to$compositions, identities$\to$identities)

$(X, x_0) \mapsto \pi_1(X, x_0) = \{[f]\mid f:I\to X, f(0) = f(1) = x_0\}$. 

$\begin{tikzcd} (X, x_0)\ar[r, "\pi_1"]&\pi_1(X, x_0) \end{tikzcd}$. 
\[
\begin{tikzcd}
(X, x_0)\ar[d, "h"]\ar[r, "\pi_1"]&\pi_1(X, x_0) & f \ar[d] \\
(Y, y_0) \ar[r, "\pi_1"]&\pi_1(Y, y_0) & h\circ f \\
\end{tikzcd}
\]

\section*{Lecture 11, 2/6/23}

just homework review

\section*{Lecture 12, 2/8/23}

Review for the midterm

Types of questions:

\begin{enumerate}

\item State Definitions (particularly important and complicated definitions)

\item Carefully state important theorems

\item Give key counterexamples

\item Prove easy propositions

\item Do simple constructions/modifications (applying our theorems)

\end{enumerate}

HEP: For any $f:X\times\{0\}\cup A\times I$, there exists an extesnsion $\bar{f}:X\times I \to Y$. This is equivalent to $X\times I$ def retracting to $X\times\{0\}\cup A\times I$
\[
\begin{tikzcd}
X\times\{0\}\cup A\times I \ar[d, "\iota"] \ar[r, "f"] & Y \\
X\times I \ar[u, "r"] \\ & 
\end{tikzcd}
\]

Questions from previous exams;

1. Carefully define the notion of a cell complex. 

2. Let $f:S^1\to S^1$ be continuous. Define the mapping cylinder $M_f$ and describe an explicit cell structure on it. 

3. What does it mean to say that computing the fundamental gorup is a functor from $\Top*$ to $\Grp$

4. In each case, find a simpler space with the same homotopy type, briefly explain reasons and use pictures. 
	a. suspension of disjoint union of 3 circles as a wedge product. 

	b. View $S^2$ as a subspace of $S^3$ and describe the quotient $S^3/S^2$. 

	c. Remove both the Z axis and the unit circle in the xy plane and describe a 2-dimensional object with the same homotopy type

5. Let $X$ be a topological space, prove that $f:S^1\to X$ extends to a map $F:D^2\to X$ if and only if $f$ is nullhomotopic. 

6. Define the homotopy extension property and then prove that a pair $(X, A)$ has the property if and only if there is a retraction $X\times\{0\}\cup A\times I$. 

7. Give examples of each of the following:
		retract of a cell complex onto a cell complex which does not extend to a def retraction

		Give an example of a contractible space that does not deformation retract to a point. 




















\end{document}