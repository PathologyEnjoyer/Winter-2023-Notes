\documentclass[x11names,reqno,14pt]{extarticle}
% Choomno Moos
% Portland State University
% Choom@pdx.edu


%% stupid experiment %%
%%%%%%%%%%%%% PACKAGES %%%%%%%%%%%%%

%%%% SYMBOLS AND MATH %%%%
\let\oldvec\vec
\usepackage{authblk}	% author block customization
\usepackage{microtype}	% makes stuff look real nice
\usepackage{amssymb} 	% math symbols
\usepackage{siunitx} 	% for SI units, and the degree symbol
\usepackage{mathrsfs}	% provides script fonts like mathscr
\usepackage{mathtools}	% extension to amsmath, also loads amsmath
\usepackage{esint}		% extended set of integrals
\mathtoolsset{showonlyrefs} % equation numbers only shown when referenced
\usepackage{amsthm}		% theorem environments
\usepackage{relsize}	%font size commands
\usepackage{bm}			% provides bold math
\usepackage{bbm}		% for blackboard bold 1

%%%% FIGURES %%%%
\usepackage{graphicx} % for including pictures
\usepackage{float} % allows [H] option on figures, so that they appear where they are typed in code
\usepackage{caption}
\usepackage{hyperref}
%\usepackage{titling}
\usepackage{tikz} % for drawing
\usetikzlibrary{shapes,arrows,chains,positioning,cd,decorations.pathreplacing,decorations.markings,hobby,knots,braids}
\usepackage{subcaption}	% subfigure environment in figures

%%%% MISC %%%%
\usepackage{enumitem} % for lists and itemizations
\setlist[enumerate]{leftmargin=*,label=\bf \arabic*.}

\usepackage{multicol}
\usepackage{multirow}
\usepackage{url}
\usepackage[symbol]{footmisc}
\renewcommand{\thefootnote}{\fnsymbol{footnote}}
\usepackage{lastpage} % provides the total number of pages for the "X of LastPage" page numbering
\usepackage{fancyhdr}
\usepackage{manfnt}
\usepackage{nicefrac}
%\usepackage{fontspec}
%\usepackage{polyglossia}
%\setmainlanguage{english}
%\setotherlanguages{khmer}
%\newfontfamily\khmerfont[Script=Khmer]{Khmer Busra}

%%% Khmer script commands for math %%%
%\newcommand{\ka}{\text{\textkhmer{ក}}}
%\newcommand{\ko}{\text{\textkhmer{ត}}}
%\newcommand{\kha}{\text{\textkhmer{ខ}}}

%\usepackage[
%backend=biber,
% numeric
%style=numeric,
% APA
%bibstyle=apa,
%citestyle=authoryear,
%]{biblatex}

\usepackage[explicit]{titlesec}
%%%%%%%% SOME CODE FOR REDECLARING %%%%%%%%%%

\makeatletter
\newcommand\RedeclareMathOperator{%
	\@ifstar{\def\rmo@s{m}\rmo@redeclare}{\def\rmo@s{o}\rmo@redeclare}%
}
% this is taken from \renew@command
\newcommand\rmo@redeclare[2]{%
	\begingroup \escapechar\m@ne\xdef\@gtempa{{\string#1}}\endgroup
	\expandafter\@ifundefined\@gtempa
	{\@latex@error{\noexpand#1undefined}\@ehc}%
	\relax
	\expandafter\rmo@declmathop\rmo@s{#1}{#2}}
% This is just \@declmathop without \@ifdefinable
\newcommand\rmo@declmathop[3]{%
	\DeclareRobustCommand{#2}{\qopname\newmcodes@#1{#3}}%
}
\@onlypreamble\RedeclareMathOperator
\makeatother

\makeatletter
\newcommand*{\relrelbarsep}{.386ex}
\newcommand*{\relrelbar}{%
	\mathrel{%
		\mathpalette\@relrelbar\relrelbarsep
	}%
}
\newcommand*{\@relrelbar}[2]{%
	\raise#2\hbox to 0pt{$\m@th#1\relbar$\hss}%
	\lower#2\hbox{$\m@th#1\relbar$}%
}
\providecommand*{\rightrightarrowsfill@}{%
	\arrowfill@\relrelbar\relrelbar\rightrightarrows
}
\providecommand*{\leftleftarrowsfill@}{%
	\arrowfill@\leftleftarrows\relrelbar\relrelbar
}
\providecommand*{\xrightrightarrows}[2][]{%
	\ext@arrow 0359\rightrightarrowsfill@{#1}{#2}%
}
\providecommand*{\xleftleftarrows}[2][]{%
	\ext@arrow 3095\leftleftarrowsfill@{#1}{#2}%
}
\makeatother

%%%%%%%% NEW COMMANDS %%%%%%%%%%

% settings
\newcommand{\N}{\mathbb{N}}                     	% Natural numbers
\newcommand{\Z}{\mathbb{Z}}                     	% Integers
\newcommand{\Q}{\mathbb{Q}}                     	% Rationals
\newcommand{\R}{\mathbb{R}}                     	% Reals
\newcommand{\C}{\mathbb{C}}                     	% Complex numbers
\newcommand{\K}{\mathbb{K}}							% Scalars
\newcommand{\F}{\mathbb{F}}                     	% Arbitrary Field
\newcommand{\E}{\mathbb{E}}                     	% Euclidean topological space
\renewcommand{\H}{{\mathbb{H}}}                   	% Quaternions / Half space
\newcommand{\RP}{{\mathbb{RP}}}                       % Real projective space
\newcommand{\CP}{{\mathbb{CP}}}                       % Complex projective space
\newcommand{\Mat}{{\mathrm{Mat}}}						% Matrix ring
\newcommand{\M}{\mathcal{M}}
\newcommand{\GL}{{\mathrm{GL}}}
\newcommand{\SL}{{\mathrm{SL}}}

\newcommand{\tgl}{\mathfrak{gl}}
\newcommand{\tsl}{\mathfrak{sl}}                  % Lie algebras; i.e., tangent space of SO/SL/SU
\newcommand{\tso}{\mathfrak{so}}
\newcommand{\tsu}{\mathfrak{sl}}


% typography
\newcommand{\noi}{\noindent}						% Removes indent
\newcommand{\tbf}[1]{\textbf{#1}}					% Boldface
\newcommand{\mc}[1]{\mathcal{#1}}               	% Calligraphic
\newcommand{\ms}[1]{\mathscr{#1}}               	% Script
\newcommand{\mbb}[1]{\mathbb{#1}}               	% Blackboard bold


% (in)equalities
\newcommand{\eqdef}{\overset{\mathrm{def}}{=}}		% Definition equals
\newcommand{\sub}{\subseteq}						% Changes default symbol from proper to improper
\newcommand{\psub}{\subset}						% Preferred proper subset symbol

% Categories
\newcommand{\catname}[1]{{\text{\sffamily {#1}}}}

\newcommand{\Cat}{{\catname{C}}}
\newcommand{\cat}[1]{{\catname{\ifblank{#1}{C}{#1}}}}
\newcommand{\CAT}{{\catname{Cat}}}
\newcommand{\Set}{{\catname{Set}}}

\newcommand{\Top}{{\catname{Top}}}
\newcommand{\Met}{{\catname{Met}}}
\newcommand{\PL}{{\catname{PL}}}
\newcommand{\Man}{{\catname{Man}}}
\newcommand{\Diff}{{\catname{Diff}}}

\newcommand{\Grp}{{\catname{Grp}}}
\newcommand{\Grpd}{{\catname{Grpd}}}
\newcommand{\Ab}{{\catname{Ab}}}
\newcommand{\Ring}{{\catname{Ring}}}
\newcommand{\CRing}{{\catname{CRing}}}
\newcommand{\Mod}{{\mhyphen\catname{Mod}}}
\newcommand{\Alg}{{\mhyphen\catname{Alg}}}
\newcommand{\Field}{{\catname{Field}}}
\newcommand{\Vect}{{\catname{Vect}}}
\newcommand{\Hilb}{{\catname{Hilb}}}
\newcommand{\Ch}{{\catname{Ch}}}

\newcommand{\Hom}{{\mathrm{Hom}}}
\newcommand{\End}{{\mathrm{End}}}
\newcommand{\Aut}{{\mathrm{Aut}}}
\newcommand{\Obj}{{\mathrm{Obj}}}
\newcommand{\op}{{\mathrm{op}}}

% Norms, inner products
\delimitershortfall=-1sp
\newcommand{\widecdot}{\, \cdot \,}
\newcommand\emptyarg{{}\cdot{}}
\DeclarePairedDelimiterX{\norm}[1]{\Vert}{\Vert}{\ifblank{#1}{\emptyarg}{#1}}
\DeclarePairedDelimiterX{\abs}[1]\vert\vert{\ifblank{#1}{\emptyarg}{#1}}
\DeclarePairedDelimiterX\inn[1]\langle\rangle{\ifblank{#1}{\emptyarg,\emptyarg}{#1}}
\DeclarePairedDelimiterX\cur[1]\{\}{\ifblank{#1}{\emptyarg,\emptyarg}{#1}}
\DeclarePairedDelimiterX\pa[1](){\ifblank{#1}{\emptyarg}{#1}}
\DeclarePairedDelimiterX\brak[1][]{\ifblank{#1}{\emptyarg}{#1}}
\DeclarePairedDelimiterX{\an}[1]\langle\rangle{\ifblank{#1}{\emptyarg}{#1}}
\DeclarePairedDelimiterX{\bra}[1]\langle\vert{\ifblank{#1}{\emptyarg}{#1}}
\DeclarePairedDelimiterX{\ket}[1]\vert\rangle{\ifblank{#1}{\emptyarg}{#1}}

% mathmode text operators
\RedeclareMathOperator{\Re}{\operatorname{Re}}		% Real part
\RedeclareMathOperator{\Im}{\operatorname{Im}}		% Imaginary part
\DeclareMathOperator{\Stab}{\mathrm{Stab}}
\DeclareMathOperator{\Orb}{\mathrm{Orb}}
\DeclareMathOperator{\Id}{\mathrm{Id}}
\DeclareMathOperator{\vspan}{\mathrm{span}}			% Vector span
\DeclareMathOperator{\tr}{\mathrm{tr}}
\DeclareMathOperator{\adj}{\mathrm{adj}}
\DeclareMathOperator{\diag}{\mathrm{diag}}
\DeclareMathOperator{\eq}{\mathrm{eq}}
\DeclareMathOperator{\coeq}{\mathrm{coeq}}
\DeclareMathOperator{\coker}{\mathrm{coker}}
\DeclareMathOperator{\dom}{\mathrm{dom}}
\DeclareMathOperator{\cod}{\mathrm{codom}}
\DeclareMathOperator{\im}{\mathrm{im}}
\DeclareMathOperator{\Dim}{\mathrm{dim}}
\DeclareMathOperator{\codim}{\mathrm{codim}}
\DeclareMathOperator{\Sym}{\mathrm{Sym}}
\DeclareMathOperator{\lcm}{\mathrm{lcm}}
\DeclareMathOperator{\Inn}{\mathrm{Inn}}
\DeclareMathOperator{\sgn}{sgn}						% sgn operator
\DeclareMathOperator{\intr}{\text{int}}             % Interior
\DeclareMathOperator{\co}{\mathrm{co}}				% dual/convex Hull
\DeclareMathOperator{\Ann}{\mathrm{Ann}}
\DeclareMathOperator{\Tor}{\mathrm{Tor}}


% misc symbols
\newcommand{\divides}{\big\lvert}
\newcommand{\grad}{\nabla}
\newcommand{\veps}{\varepsilon}						% Preferred epsilon
\newcommand{\vphi}{\varphi}
\newcommand{\del}{\partial}							% Differential/Boundary
\renewcommand{\emptyset}{\text{\O}}					% Traditional emptyset symbol
\newcommand{\tril}{\triangleleft}					% Quandle operation
\newcommand{\nabt}{\widetilde{\nabla}}				% Contravariant derivative
\newcommand{\later}{$\textcolor{red}{\blacksquare}$}% Laziness indicator

% misc
\mathchardef\mhyphen="2D							% mathomode hyphen
\renewcommand{\mod}[1]{\ (\mathrm{mod}\ #1)}
\renewcommand{\bar}[1]{\overline{#1}}				% Closure/conjugate
\renewcommand\qedsymbol{$\blacksquare$} 			% Changes default qed in proof environment
%%%%% raised chi
\DeclareRobustCommand{\rchi}{{\mathpalette\irchi\relax}}
\newcommand{\irchi}[2]{\raisebox{\depth}{$#1\chi$}}
\newcommand\concat{+\kern-1.3ex+\kern0.8ex}

% Arrows
\newcommand{\weak}{\rightharpoonup}					% Weak convergence
\newcommand{\weakstar}{\overset{*}{\rightharpoonup}}% Weak-star convergence
\newcommand{\inclusion}{\hookrightarrow}			% Inclusion/injective map
\renewcommand{\natural}{\twoheadrightarrow}				% Natural map
\newcommand{\oo}{\infty}

% Environments
\theoremstyle{plain}
\newtheorem{thm}{Theorem}[section]
%\newtheorem{lem}[thm]{Lemma}
\newtheorem{lem}{Lemma}
\newtheorem*{lems}{Lemma}
\newtheorem{cor}[thm]{Corollary}
\newtheorem{prop}{Proposition}
\newtheorem*{claim}{Claim}
\newtheorem*{cors}{Corollary}
\newtheorem*{props}{Proposition}
\newtheorem*{conj}{Conjecture}

\theoremstyle{definition}
\newtheorem{defn}{Definition}[section]
\newtheorem*{defns}{Definition}
\newtheorem{exm}{Example}[section]
\newtheorem{exer}{Exercise}[section]

\theoremstyle{remark}
\newtheorem*{rem}{Remark}

\newtheorem*{solnx}{Solution}
\newenvironment{soln}
    {\pushQED{\qed}\renewcommand{\qedsymbol}{$\Diamond$}\solnx}
    {\popQED\endsolnx}%

% Macros

\newcommand{\restr}[1]{_{\mkern 1mu \vrule height 2ex\mkern2mu #1}}
\newcommand{\Upushout}[5]{
    \begin{tikzcd}[ampersand replacement = \&]
    \&#2\ar[rd,"\iota_{#2}"]\ar[rrd,bend left,"f"]\&\&\\
    #1\ar[ur,"#4"]\ar[dr,"#5"]\&\&#2\oplus_{#1} #3\ar[r,dashed,"\vphi"]\&Z\\
    \&#3\ar[ur,"\iota_{#3}"']\ar[rru,bend right,"g"']\&\&
    \end{tikzcd}
}
\newcommand{\exactshort}[5]{
		\begin{tikzcd}[ampersand replacement = \&]
			0\ar[r]\&#1\ar[r,"#2"]\& #3 \ar[r,"#4"]\& #5 \ar[r]\&0
		\end{tikzcd}
}
\newcommand{\product}[6]{
		\begin{tikzcd}[ampersand replacement = \&]
			#1 \& #2 \ar[l,"#4"'] \\
			#3 \ar[u,"#5"] \ar[ur,"#6"']
		\end{tikzcd}
}
\newcommand{\coproduct}[6]{
		\begin{tikzcd}[ampersand replacement = \&]
			#1 \ar[r,"#4"] \ar[d,"#5"'] \& #2 \ar[dl,"#6"] \\
			#3
		\end{tikzcd}
}
%%%%%%%%%%%% PAGE FORMATTING %%%%%%%%%

\usepackage{geometry}
    \geometry{
		left=15mm,
		right=15mm,
		top=15mm,
		bottom=15mm	
		}

\usepackage{color} % to do: change to xcolor
\usepackage{listings}
\lstset{
    basicstyle=\ttfamily,columns=fullflexible,keepspaces=true
}
\usepackage{setspace}
\usepackage{setspace}
\usepackage{mdframed}
\usepackage{booktabs}
\DeclareMathOperator{\cl}{cl}
\usepackage[document]{ragged2e}
\usepackage{amsmath}
\pagestyle{fancy}{
	\fancyhead[L]{Winter 2023}
	\fancyhead[C]{221B - Cell Complexes}
	\fancyhead[R]{John White}
  
  \fancyfoot[R]{\footnotesize Page \thepage \ of \pageref{LastPage}}
	\fancyfoot[C]{}
	}
\fancypagestyle{firststyle}{
     \fancyhead[L]{}
     \fancyhead[R]{}
     \fancyhead[C]{}
     \renewcommand{\headrulewidth}{0pt}
	\fancyfoot[R]{\footnotesize Page \thepage \ of \pageref{LastPage}}
}
\DeclareMathOperator{\rel}{rel}

\title{221B - Cell Complexes}
\author{John White}
\date{Winter 2023}


\begin{document}

\section*{Lecture 1}

We will be using Hatcher's Algebraic Topology. The topology sequence is usually something like 
\begin{align*}
A & \text{Topological Spaces} \\
B & \text{Cell Complexes} \\
C & \text{Manifolds} \\
\end{align*}

\thm (BIG Theorem)

Given a ``reasonably nice" space, there is a bijection between connected covers of a space and subgroups of the fundamental group. 

\subsection*{\underline{Categories:}}

Algebraic structures that are much flabbier than a group. They consist of 
\begin{itemize}
\item A collection of arrows
\item A partial binary operation on these arrows
\item Objects, which arrows go between
\end{itemize}
We also want a composition law. That is, for objects and arrows
\[
\begin{tikzcd}
A \ar[r, "f"] & B \ar[r, "g"] & C 
\end{tikzcd}
\]
there is an arrow $\begin{tikzcd} A \ar[r, "g \circ f"]&  C \end{tikzcd}$. We want this composition to be associative, that is $(f \circ g) \circ h = f \circ (g \circ h)$, and we want objects to have identity arrows. 

Not all functions have inverses. Using sets and functions as an example, we have described the category Set. 

Here are some more examples of categories: 
\exm

\begin{itemize}
\item Groups and group homomorphisms (Grp)
\item Topological spaces and continuous functions (Top)
\item etc. 
\end{itemize}

We can make the following new category. 

\defn

We denote by $\Top^*$ the \underline{category of based topological spaces}, whose objects are pairs $(X, x_0)$, where $X$ is a topological space and $x_0 \in X$, and whose morphisms are continuous functions $f:(X,x_0)\to(Y, y_0)$ such that $f(x_0) = y_0$. 

\subsection*{Goal:}

Our goal is to get a functor from Top to Grp. The fundamental group functor $\pi_1$ will go from $\Top^*$ to Grp.

\section*{Lecture 2}

\subsection*{\underline{Topology review:}}

\defn 

A \underline{topological space} is a set $X$ along with a collection of subsets of $X$ called ``open sets," such that $X, \varnothing$ are open, and the arbitrary union and finite intersection of open sets are open. 

Notice the following diagram commutes using the product topology

\[
\begin{tikzcd}
&\ar[ld, "f"'] \ar[d, dotted, "\exists !"]Z \ar[rd, "g"] &  \\
X  &\ar[l, "P_X"'] X\times Y \ar[r, "P_Y"] & Y\\
\end{tikzcd}
\]

And in general 

\[
\begin{tikzcd}
Z \ar[d, dotted, "\exists !"']  \ar[rd, "f_\alpha"] & \\
\prod_{\alpha\in A}X_\alpha \ar[r, "P_\alpha"]& X_\alpha\\
\end{tikzcd}
\]
Maps are continuous; functions are not. 

\lem (Gluing lemma)

Suppose $f:A\to Y$, $g:B\to Y$ are continuous, and $f(x) = g(x)$ for all $x \in A \cap B$. Then $f\cup g: A \cup B \to Y$ is continuous. This only holds as long as $A, B \subseteq X$ are closed. 

\subsection*{\underline{Same Shape, Same Map}}
(maps up to wriggling things around a bit)

\defn Two maps are \underline{homotopic} if there exists a parametrized map $f_t:X\to Y$ such that $f_0 = f, f_1 = g$ for $f, g:X \to Y$. Equivalently, and more precisely, if there exists a map $F:X\times[0,1]\to Y$ such that $F(x, 0) = f(x), F(x, 1) = g(x)$ for all $x \in X$. 

$X, Y$ topological spaces are said to have the same shape if there exist maps $f:X\to Y, g:Y\to X$ such that $g\circ f\simeq \Id_X$ and $f\circ g \simeq \Id_Y$. We may say that $X, Y$ have the same \underline{homotopy type}

\defn

A \underline{deformation retraction} from $X \to A\subseteq X$ is a map from $X \times I \to X$ such that, for all $x \in A$, and $s, t \in I$,
\begin{align*}
f_0(x) &= x&\forall x \in X \\
f_1(x) &\in A&\forall x \in X \\
f_t(x) &= f_s(x)& \forall x \in A \\
\end{align*}

\section*{Lecture 3, 1/13/23}

\defn

Let $X$ be a topological space. A \underline{retraction} is a map $r:X\to X$ such that $r \circ r = r$. That is, $r(r(x)) = r(x)$ for any $x \in X$. 

Let $A = r(X)$. Then $r|_A = \Id_A$. 

\defn

Let $F:X\times I\to Y$. We say $f_0 \simeq f_1 \rel A\subseteq X$ are \underline{homotopic relative to $A$} if, for any $x \in A$, $f_t(x)$ is independent of $t$. That is, for any $s, t \in I$, $f_s(x) = f_t(x)$ for any $x \in A$. 


For any map $f:X\to Y$, there exists a space $Z \simeq Y$ via $g:Y \to Z$ such that $g \circ f:X\to Z$ is injective. That is, in the following diagram, we have a bijection between homotopy classes of maps $f$ and homotopy classes of maps $g\circ f$, and we can do this in a way that rigs $g \circ f$ to be injective. 
\[
\begin{tikzcd}
X \ar[r, "f"] \ar[rd, "g\circ f"'] & Y \ar[d, "g"] \\
& Z \\
\end{tikzcd}
\]

\defn

Given a map $f:X\to Y$ we can construct the \underline{Mapping Cylinder} $M_f$ by setting $M_f = X \times I \coprod Y/\sim$, where $(x, 0) \sim f(x)$. 

The visual intuition should be taking the disjoint union of $X$ and $Y$, and tieing a string between $x$ and $f(x)$ for each point. 

\claim 
$X \hookrightarrow M_f, Y \hookrightarrow M_f$, and the latter is in fact a homotopy equivalence. Further, the injection $X \hookrightarrow M_f$ is homotopic to $f(X) \hookrightarrow M_f$. 
\proof
You can construct a homotopy which ``squishes" the cylinder down to $f(X)$. 

\qed

\defn

A space $X$ is \underline{contractible} if it has the homotopy type of a point. A map is \underline{null-homotopic} if it is homotopic to a constant map. So $X$ is contractible if the identity is null-homotopic. 

Now he's drawing an example. The example is Bing's House with 2 rooms, which I will not reproduce here. But the point is that it's contractible, but not obviously so. 

\subsection*{\underline{Cell Complexes}}

Cell complexes are topological spaces which are built up inductively out of closed balls in Euclidean space. We write $\mathbb{D}^n := \{\vec{x} \in \R^n \mid \norm{\vec{x}} \leq 1 \}$, and $e^n := \{\vec{x}\in\R^n \mid \norm{\vec{x}} < 1\}$. We can see that $e^n = \operatorname{int}\mathbb{D}^n$, and $\del\mathbb{D}^n = \mathbb{S}^{n - 1}$. 

\subsubsection*{\underline{Base step}}

Start with some collection of points $X^0$, the $0$-skeleton, with the discrete topology.

\subsubsection*{\underline{Inductive step}}

Let $X^{n - 1}$ be the $n - 1$ skeleton, which has already been build and defined. Select some collection of $n$-dimensional balls $\{\mathbb{D}^n\}_{\alpha\in A}$, and some continuous ``attaching map" $\varphi_\alpha:\del\mathbb{D}^n_\alpha\to X^{n - 1}$. Then 
\[
(X^n = X^{n - 1}\coprod_{\alpha\in A}\mathbb{D}^n )/(x \sim\varphi_\alpha(x)\forall x \in \del\mathbb{D}^n)
\]

\section*{Lecture 4, 1/18/23}

A space $X$ is a \underline{cell complex} if it has been constructed using the above inductive procedure. If $n = \oo$, we use the weak topology, in which the open sets are the sets which are open when intersected with each $X^n$. 

For every $\mbb{D}^n_\alpha$ and corresponding ``attaching map $\varphi_\alpha:\del\mbb{D}^n_\alpha\to X^{n - 1}$, there is a subset of $X^n$ homeomorphic to $\operatorname{int}(\mbb{D}^n_\alpha)$, via the composition
\[
\operatorname{int}(\mbb{D}^n_\alpha) \hookrightarrow \mbb{D}^n_\alpha \hookrightarrow X^{n - 1}\coprod_{\alpha}\mbb{D}^n_\alpha \to X^n
\]
which we call $\Phi_\alpha:\mbb{D}^n_\alpha\to X^{n - 1}$. So the attaching map $\phi_\alpha:\del\mbb{D}^n_\alpha\to X^{n - 1}$ extends to a ``characteristic map" $\Phi_\alpha$.

We will now see many examples of things. 

\exm

If you stop after constructing $X^1$, it's a graph. 

\exm

$\mbb{S}^n$ has a cell structure with one $e_0$ and one $e_n$. 

\exm

Consider $\RP^2$. This can be expressed as $(\R^3\setminus\{0\})/(\vec{x}\sim\lambda\vec{x}, \lambda\neq0)$. We can replace $2$ with any $n$ and get $\RP^n$. Indeed, we can replace $\R$ with $\C$, $\mbb{H}$, or indeed any field. 

\underline{Homogenous coordinates}

For $(x, y, z)\neq(0,0,0)$, we have $[x, y, z]\eqdef\{(\lambda x, \lambda y, \lambda z) \mid \lambda \neq0\}$. For example, $[1,2,3]=[2,4,6]$. 


























\end{document}