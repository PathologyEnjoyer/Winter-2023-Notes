\documentclass[x11names,reqno,14pt]{extarticle}
\input{preamble}
\usepackage[document]{ragged2e}
\usepackage{epsfig}

\pagestyle{fancy}{
	\fancyhead[L]{Winter 2023}
	\fancyhead[C]{236A}
	\fancyhead[R]{John White}
  
  \fancyfoot[R]{\footnotesize Page \thepage \ of \pageref{LastPage}}
	\fancyfoot[C]{}
	}
\fancypagestyle{firststyle}{
     \fancyhead[L]{}
     \fancyhead[R]{}
     \fancyhead[C]{}
     \renewcommand{\headrulewidth}{0pt}
	\fancyfoot[R]{\footnotesize Page \thepage \ of \pageref{LastPage}}
}
\newcommand{\pmat}[4]{\begin{pmatrix} #1 & #2 \\ #3 & #4 \end{pmatrix}}
\newcommand{\A}{\mathbb{A}}
\newcommand{\fin}{``\in"}
\DeclareMathOperator{\Perm}{Perm}
\newcommand{\Rmod}{R-\text{mod}}

\newcommand{\exactlon}[5]{
		\begin{tikzcd}
			0\ar[r]&#1\ar[r,"#2"]& #3 \ar[r,"#4"]& #5 \ar[r]&0
		\end{tikzcd}
}

\title{236A - Homological Algebra}
\author{John White}
\date{Winter 2023}


\begin{document}

\section*{Lecture 1, 1/11/13}

Homological algebra is the study of complexes of $R$-modules, where $R$ is a ring with identity $1\neq0$. Notationally, $R$-mod is the category of all left $R$-modules, and $R$-mod is the category of all finitely generated $R$-modules. 

\defn

Let $A_n\fin R$-mod for $n\in\Z$ and $d_n\in\Hom_R(A_n, A_{n - 1})$ such that $d_{n - 1}\circ d_n = 0$ for all $n \in\Z$. Then the sequence
\[
\begin{tikzcd}
\cdots\ar[r, "d_{n + 2}"]& A_{n + 1}\ar[r, "d_{n + 1}"] & A_n \ar[r, "d_n"] & A_{n - 1} \ar[r, "d_{n - 1}"] & \cdots\\
\end{tikzcd}
\]
is called a \underline{complex of $R$-modules}, assuming $\im(d_n) \subseteq \ker(d_{n - 1})$. The sequence
\[
\begin{tikzcd}
0\ar[r]&A_m\ar[r]&A_{m - 1}\ar[r]&\cdots\ar[r]&A_{n + 1}\ar[r]&A_n\ar[r]&0 \\
\end{tikzcd}
\]
will occur more frequently. A complex $\A$ is an \underline{exact sequence} if $\im(d_n) = \ker(d_{n - 1})$ for all $n \in \Z$. This is called a \underline{short exact sequence} if there are no more than 3 non-zero terms. Given a complex $\A$, the \underline{nth homology modules} (or groups, in some cases) of $\A$ is
\[
H_n(\A) = \frac{\ker(d_{n -1})}{\im(d_n)}
\]

\rem

Given a short exact sequence (hereby abbv. as SES) 
\[
\begin{tikzcd}
0\ar[r]&A\ar[r,"f"]&B\ar[r,"g"]&C\ar[r]&0
\end{tikzcd}
\]
$f$ is a mono and $g$ is an epi, so $C\simeq B/\im(f)$. If $A, B$ are known, but not $f$, then infinitely many $C$ are available to complete the short exact sequence.

\exm 

Let $R = k$, a field, and take $A = B = k^{(\N)} = \oplus_{i\in\N}k$.

\begin{itemize}
\item[(i)] $\begin{tikzcd} 0\ar[r]&A\ar[r, "\Id"]&B\ar[r]&0 \end{tikzcd}$ is a SES.
\item[(ii)] Define $f:A\to B$ by 

\begin{align*}
f(b_i) & = b_{2i} \text{ for }i\in \N \\
g(b_0) & = \begin{cases} 0 & i\text{ even } \\ b_{\tau(i)} & i\text{ odd } \end{cases}
\end{align*}

Where $\tau:(2\N-1)\to\N$ is a bijection. If $A = B = C = \kappa^{(\N)}$, then 
\[
\exactshort{A}{f}{B}{g}{C}
\]
is a SES. 
\item[(iii)] Let $R = \Z$. Then 
\[
\exactshort{3\Z}{\iota}{\Z}{\bar{-}}{\Z/3\Z}
\]
is a SES. 
\item[(iv)] Let $R = \Z$. The sequence 
\[
\exactshort{\overbrace{6\Z}^{A_1}}{\iota}{\overbrace{\Z}^{A_0}}{\bar{-}}{\overbrace{\Z/3\Z}^{A_{-1}}}
\]
is a complex which is not exact. In fact, $H_0(\A) = \overbrace{3\Z}^{\ker(g)}/\underbrace{6\Z}_{\im(f)} \cong \Z/2\Z$. 

\item[(v)] Let $R = \kappa[x, y]$, $\kappa$ a field. Let $f$ be the inclusion $(x)\hookrightarrow R[x, y]$. The sequence
\[
\exactshort{(x)}{f}{R}{g}{\kappa[y]}
\]
where
\[
g\left(\sum_{i, j = 0}^{\text{finite}}a_{ij}x^iy^j\right) = \sum_{j > 0}^{\text{finite}}a_{\sigma_j}y^j
\]
is exact. 

\item[(vi)] Let $R = \kappa[x, y]$. Define $\A$ as
\[
\exactshort{\overbrace{(x)}^{A_1}}{f}{\overbrace{R}^{A_0}}{g}{\overbrace{\underbrace{\kappa}_{=R/(x, y)}}^{A_{-1}}}
\]
where 
\[
g\left(\sum_{i, j = 0}^{\text{finite}}a_{ij}x^iy^j\right) = a_{\oo}
\]
then $\ker(g) = (x, y)$ and $\im(f) = (x)$, so $\A$ is not exact. In fact, 
\begin{align*}
H_0(\A) & = (x, y)/(x) \\
		  & \simeq (y) \\
		  & \simeq R \\
\end{align*}


\end{itemize}

Note: If $R$ is an integral domain and $x\in R\setminus\{0\}$, then $(x)\simeq R$ (as $R$-modules, \underline{not} as rings!), with isomorphism $r\mapsto rx$. 

Typical questions addressed by homological algebra: 

\begin{itemize}
\item[(i)] Suppose
\[\A:
\begin{tikzcd}
\cdots\ar[r]&A_{n + 1}\ar[r, "d_{n + 1}"]&A_n\ar[r,"d_n"]&A_{n - 1}\ar[r]&\cdots \\
\end{tikzcd}
\]
is an exact sequence in $R$-mod and $F:\Rmod\to S-\text{mod}$ is a functor. Is the sequence
\[
\begin{tikzcd}
\cdots\ar[r]&F(A_{n + 1})\ar[r, "F(d_{n + 1})"] &F(A_n)\ar[r,"F(d_n)"] &F(A_{n - 1})\ar[r] &\cdots
\end{tikzcd}
\]
exact? $F(\A)$ is a complex when $F$ is additive, but it may or may not be exact. 

\item[(ii)] Given $A, C\fin \Rmod$, characterize all modules $B$ such that there exists an exact sequence
\[
\exactshort{A}{}{B}{}{C}
\]
As an example, $R =\Z, A = C = \A/p\Z$, $p$ prime, then
\[
\exactshort{\Z/p\Z}{f}{\Z/p\Z\oplus\Z/p\Z}{g}{\Z/p\Z}
\]
with $f:x\mapsto(x,0)$ and $g:(x,y)\mapsto y$ is a SES. Alternatively, we could take $f:x + p\Z\mapsto px + p^2\Z$ and $g:y + p^2\Z\mapsto y + p\Z$ to make 
\[
\exactshort{\Z/p\Z}{f}{\Z/p^2\Z}{g}{\Z/p\Z}
\]
a SES. These are the only possibilities in this case! In general, though, there are infinitely many possibilities for $B$. Why is this interesting? If $R$ is an artinian ring and $M\fin\Rmod$, then there are only finitely many simple $s_1,\dots,s_n\fin\Rmod$ up to isomorphism. Moreover, for $M\fin\Rmod$, there is a chain
\[
M = M_0\supset M_1 \supset \cdots \supset M_{\ell} = 0
\]
such that $M_i/M_{i + 1}$ is simple for all $i<\ell$. If the answer to question $(ii)$ is known, then all objects in $\Rmod$ of fixed length $\ell$ are known up to isomorphism! Simply proved by induction.
\end{itemize}

\subsection*{\underline{Algebraic Topology}}

\defn
The \underline{standard $n$-simplex} $\bigtriangleup_n$ in $\R^n$ is the convex hull of $v_0, v_1, \dots, v_n$, where $v_0 = 0$ and $v_i = (0, \dots, 0, \overbrace{1}^{i}, 0, \dots, 0)$ (so the standard basis). 

An \underline{oriented simplex} is $(\bigtriangleup_x, [\pi])$, where $[\pi]$ is an equivalence class of permutations of $\{0, \dots, n\}$, where $\pi\sim\pi'\iff \sgn(\pi)=\sgn(\pi')$. We write
\[
(\bigtriangleup_x, \pi) = [v_{\pi(0)}, v_{\pi(1)},\dots,v_{\pi(n)}
\]
and identify $\bigtriangleup_n$ with $[0, 1, \dots, n]$. The ngative is $-[w_0, \dots, w_n]$. 

\defn

Let $X$ be a topological space. An \underline{$n$-simplex in $X$} is a continuous map
\[
\sigma:\bigtriangleup_n\to X
\]
The \underline{group of $n$-chains of $X$}, $S_n(x)$, is the free abelian group having as basis the $n$-simplices in $X$. The \underline{singular chain complex of $X$} is
\[
\begin{tikzcd} 
\cdots\ar[r]&S_n(X)\ar[r,"\del_n"] &S_{n -1}(X)\ar[r, "\del_{n - 1}"] &\cdots\ar[r] &S_0(X)\ar[r] &0
\end{tikzcd}
\]
denoted $\mbb{S}$, where $\del_n:S_n(X)\to S_{n - 1}(X)$ is the \underline{$n$th boundary map}, which can be defined if we define $\del_n(\sigma)$ for all $n$-simplices $\sigma$ in $X$ (i.e. in the basis of $S_n(X)$). Consider the map
\begin{align*}
\tau_i:\R^{n - 1} & \to \R^n \\
(a_1, \dots, a_{n - 1})&\mapsto(a_1, \dots, a_{i - 1}, 0, a_{i + 1}, \dots, a_n) \\
\end{align*}
For $i \in \{0, \dots, n\}$. Then $\tau_i$ is continuous and $\tau_i(\bigtriangleup_{n - 1}) = \bigtriangleup_n$. Define
\[
\del_n(\sigma) = \sum_{i = 0}^n(-1)^i\sigma(\tau_i)
\]

\thm $\del_{n - 1}\circ\del_n = 0$ for all $n \in \N$, i.e. $\mbb{S}$ is a complex in $\Z$-mod. 

\defn

The \underline{group of $n$-cycles} is $Z_n(X) = \ker(\del_{n - 1})$, and the \underline{group of $n$-boundaries} is $B_n = \im(\del_n)$. 

The \underline{$n$th homology group} is $H_n(X) = Z_n(X)/B_n(X)$.

\section*{Lecture 2, 1/13/23}

\subsection*{\underline{Chapter I: Categories and functors}}

There is a definition page on the Gaucho that has all the most basic definitions - objects, morphisms, compositions, etc. 

If $f \in \Hom_C(A, B)$, we often write $\begin{tikzcd} A\ar[r, "f"] & B \end{tikzcd}$ even if $f$ is not literally a map. 

\exm

\begin{enumerate}

\item The category of all sets, $\Set$. The object class consists of all sets, and the morphisms are just set maps. 

\item The category of all topological spaces, $\Top$. The object class consists of all topological spaces, and the morphisms are continuous functions. 

\item The category of all groups, $\Grp$. The object class consists of all groups, and the morphisms are group homomorphisms. 

\item Let $(P, \leq)$ be a partially ordered set with a relation $\leq$ which is reflexive, antisymmetric,  and transitive. Then we can make $P$ into a category, whose objects are the elements of $p$, and for $u, s \in P$, $\Hom_P(u, s) = \begin{cases} (u, s) & u \leq s \\ \varnothing & u\not\leq s \\ \end{cases}$. We define the composition $(s, t)(u, s) \eqdef (u, t)$. 

\item The opposite category of a category $C$, $C^\op$. 

\item Let $R$ be a ring. $R$-Mod is the category of left $R$ modules. $R$-mod is the finitely generated $R$-modules, and similarly for Mod-$R$ and mod-$R$, which are the right $R$-modules.

\item $R$-comp. The object class consists of complexes of left $R$-modules.

Let $\mathbb{A}, \mathbb{A}'$ be objects of $R$-comp. Note: it is problematic to say ``$\mathbb{A}, \mathbb{A}' \in R$-comp, as $R$-comp is not a set!

Say $\mathbb{A} = \begin{tikzcd} \cdots\ar[r] &A_{n + 1} \ar[r, "d_{n + 1}"] & A_n \ar[r, "d_n"] & A_{n - 1} \ar[r]& \cdots \end{tikzcd}$, and similarly for $\mathbb{A}'$. An element of $\Hom_{R-comp}(\mathbb{A},\mathbb{A}')$ will be a sequence of $R$-module homomorphisms $f_n:A_n\to A'_n$ which make the following diagram commute:

\[
\begin{tikzcd}
\cdots \ar[r] \ar[d] & A_{n} \ar[d, "f_n"'] \ar[r, "d_{n}"] & A_{n - 1} \ar[r] \ar[d, "f_{n - 1}"] & \ar[d]\cdots  \\
\cdots \ar[r] & A'_n \ar[r, "d'_n"'] & A'_{n - 1} \ar[r] & \cdots \\
\end{tikzcd}
\]

\item The category of rings Ring, whose obejcts are rings and whose morphisms are ring homomorphisms. 

\item The category of $\Z$-modules is usually denoted $\Ab$. This is also the category of Abelian groups, and is the prototypical example of an Abelian category.

\end{enumerate}

\defn A category $\mc{C}$ is called \underline{pre-additive} if for all $A, B$ objects of $\mc{C}$, the set $\Hom_{\mc{C}}(A, B)$ is an additive Abelian group (additive means we use the symbol ``$+$") such that for all eligible morphisms $f, g, h, k$, 
\begin{align*}
h(f + g) & = hf + hg \\
(f + g)k & = fk + gk \\
\end{align*}
where ``elibigle" means that these expressions make sense and are well-defined. 

\exm

\begin{enumerate}

\item $R$-mod (in particular $\Ab$)

\item $R$-comp

\item Ring fails to be pre-additive, because the identity morphisms add to be something which is not the identity morphism. 

\end{enumerate}

\defn

Let $\mc{C}, \mc{D}$ be categories. A \underline{functor} $F:\mc{C}\to\mc{D}$ consists of an assignment $F_0:\Obj(\mc{C})\to\Obj(\mc{D})$, and for each pair of objects $A, B ``\in" \Obj(\mc{C})$, a map (this actually is a map because we assume hom-sets are in fact sets). $F_{A, B}:\Hom_{\mc{C}}(A, B) \to \Hom_{\mc{D}}(F(A), F(B))$ such that, for all eligible morphisms $f, g$, and all $A \fin \mc{C}$
\begin{enumerate}[label=(\alph*)]

\item $F(\Id_A) = \Id_{F(A)}$
\item $F(f\circ g) = F(f)\circ F(g)$

\end{enumerate}

\exm

\begin{enumerate}

\item Let $\mc{C}$ be a category. Then we have the identity functor $\Id_{\mc{C}}$, which assigns $\Id_{\mc{C}}(A) = A$, and $\Id_{\mc{A}}(f) = f$ for any eligible $A \fin\Obj(\mc{D})$ and morphisms $f$. 

\item Functors $\pi_n:\Top\to\Grp$ which sends $X\mapsto\pi_n(X)$

\item $\mathbb{S}:\Top\to\Z$-comp, which sends $X\mapsto \mathbb{S}(X)$, which is a complex
\[
\begin{tikzcd}
\cdots \ar[r]&   S_{n + 1}(X)  \ar[r, "\del_{n + 1}"] &  S_n(X) \ar[r, "\del_n"]& S_{n - 1}(X) \ar[r, "\del_{n - 1}"] &\cdots \ar[r] &\ar[r,"\del_0"] S_0(x) & 0\\ 
\end{tikzcd}
\]
Let $\phi:X\to Y$ be continuous for $X, Y \fin \Top$. Then $\mathbb{S}(\phi)_n:S_n(X)\to S_n(Y)$ is given by $\sigma\mapsto\phi\circ\sigma$, and we can extend this for $\sigma$ an $n$-simplex of $X$. 

\end{enumerate}




\end{document}