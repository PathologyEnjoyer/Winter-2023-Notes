\documentclass[x11names,reqno,14pt]{extarticle}
\input{preamble}
\usepackage[document]{ragged2e}
\usepackage{amsmath}
\pagestyle{fancy}{
	\fancyhead[L]{Winter 2023}
	\fancyhead[C]{202B - Complex Analysis}
	\fancyhead[R]{John White}
  
  \fancyfoot[R]{\footnotesize Page \thepage \ of \pageref{LastPage}}
	\fancyfoot[C]{}
	}
\fancypagestyle{firststyle}{
     \fancyhead[L]{}
     \fancyhead[R]{}
     \fancyhead[C]{}
     \renewcommand{\headrulewidth}{0pt}
	\fancyfoot[R]{\footnotesize Page \thepage \ of \pageref{LastPage}}
}
\newcommand{\bigt}{\bigtriangleup}
\DeclareMathOperator{\ind}{ind}
\DeclareMathOperator{\res}{res}
\DeclareMathOperator{\supp}{supp}



\title{220B}
\author{John White}
\date{Winter 2022}

\begin{document}

\section*{Lecture 1}

\subsection*{\underline{Foundational problems}}

\begin{enumerate}

\item \underline{Interpolation:} given a set of points $a_j$, can we get an analytic $f$ such that $f(a_j) = c_j$ for some fixed $c_j$?

\item \underline{Approximation:} Given an analytif $f$, can we describe $f = \lim_mh_m$ as the limit of a sequence of functions $h_m$?

\item \underline{Value distribution:} studying sets of the form $\{\lambda \mid f(\lambda) = c\}$. This is discrete in general. 

\end{enumerate}

Let $\Omega\subseteq\C$ be open. We denote by $\mc{O}(\Omega)$ the set of analytic functions $f:\Omega\to\C$. 

How can we topologize this? 

Let $\Omega$ be the unit disk, and let $f \in \mc{O}(\Omega)$. Consider $f = \frac{1}{z - 1}$. The sup of the modulus of this blows up on $\Omega$. So, the sup norm of the modulus will not work. 

We will do it as follows. 

\defn

Let $K \subseteq \Omega$ be compact. We define
\[
\norm{f}_{\oo, K} = \sup_{z\in K}|f(z)| < \oo
\]

\defn We say that a sequence $f_n \in \ms{O}(\Omega), f_n \to f$ (uniformly on compact sets) if for every compact $K \subseteq \Omega$, $\norm{f - f_n}_{\oo, K} \to 0$. 

\thm (Weirstrass)

This notion of convergence induces a topology which is metrizable, and this metric is complete. 
\proof
\qed

Suppose $K_j \subseteq K_{j + 1} \subseteq K_{j + 1} \cdots\subset\subset \Omega$, and the union of the $K_j$ is $\Omega$.

\lem Let $f_n \in \mc{O}(\Omega)$. Then $f_n \to f$ uniformly on compact sets, i.e. $\norm{f - f_n}_{\Omega, K_j}\to0$, if and only if it converges with respect to the following metric. 

\defn For $f, g \in \mc{O}(\Omega)$, define
\[
\sigma(f, g) = \sum_{j=1}^{\oo} \frac{\norm{f - g}_{\oo, K_j}}{1 + \norm{f - g}_{\oo, K_j}}
\]

\proof

We will not prove that this is a metric. Let $f_n \in \mc{O}(\Omega)$. Let $h:\Omega\to \C$, with $d(f_n, h) \to 0$. Then $h$ is analytic by the Cauchy formula. To see this, let $a\in\Omega$, and consider $\bar{B_\delta(a)}\subseteq\Omega$ for some $\delta>0$. Then 
\[
f_n(z) = \frac{1}{2\pi i}\int_{|\zeta - a| = \delta}\frac{f_n(\zeta)}{\zeta - z}\,d\zeta
\]
We may pass the limit inside and get 
\[
h(z) = \frac{1}{2\pi i}\int_{|\zeta - a| = \delta}\frac{h(\zeta)}{\zeta - z}\,d\zeta
\]
meaning that $h$ is analytic. 

Now, suppose that $f_n$ is a Cauchy sequence with respect to $\sigma$. So for all $j$, $f_n$ is Cauchy on $(C(K_j), \norm{\cdot}_{\oo})$. 

We now review the Cauchy-Riemann equations. We write $z = x + iy\in\C$. $\bar{\del} = \frac{\del}{\del\bar{z}} = \frac{1}{2}(\frac{\del}{\del x} + i\frac{\del}{\del y})$

Recall for a $1$-form $\phi\,dz$, we define
\[
d(\phi dz) = d\phi\wedge dz = \del\phi dz \wedge dz + \bar{\del}\phi d\bar{z}\wedge dz
\]
Also, we say $dA = dx \wedge dz = \frac{1}{2i}d\bar{z}\wedge dz$ (area measure)

\thm Let $\phi \in C^{(\ell)}(\Omega)$. Then $\bar{\del}(\phi) = 0 \iff \phi \in \mc{O}(\Omega) \iff d(\phi dz) = 0$ (that is it is a closed form).  

\cor $\ker(\bar{\del}:C^{(1)}(\Omega)\to C(\Omega)) = \mc{O}(\Omega)$

\thm (Cauchy, Cauchy-Paupeir (sp?))

Suppose $\Omega\subset\subset \C$ be precompact (meaning the closure is compact), with $\del\Omega$ piecewise smooth, and let $\phi\in C^{(1)}(\bar{\Omega})$. Then for any $z \in \Omega$, we have

\[
\phi(z) = \frac{1}{2\pi i}\int_{\del\Omega}\frac{\phi(\zeta)}{\zeta - z}\,d\zeta - \frac{1}{\pi}\int\frac{(\bar{\del}\phi)(\zeta)}{\zeta - z}\,dA(\zeta)
\]

\proof

First, $\frac{\bar{\del}\phi(\zeta)}{\zeta - z}$ is $dA$-integrable because, if we write $\zeta - z = re^{i\theta}$, then $dA(\zeta) = rdr\wedge d\theta$. So we have

\[
\frac{\bar{\del}\phi(\zeta)}{\zeta - z}dA(\zeta) = \frac{\bar{\del}\phi(z + re^{i\theta})}{re^{i\theta}}rdr\wedge d\theta
\]

Let $\varepsilon>0$. Denote by $\Omega_\varepsilon = \Omega\setminus B_{\varepsilon}(z)$. If we take the differential 1-form $\omega = \frac{\phi(\zeta)}{\zeta - z}d\zeta$. We have 

\[
d\omega = \frac{\bar{\del}\phi(\zeta)}{\zeta - z}d\bar{\zeta}\wedge d\zeta
\]
Now, we use Stoke's theorem, which tells us $\int_{\del\Omega_\varepsilon}\omega = \int_{\Omega_\varepsilon}d\omega$. So we get
\[
\int_{\del\Omega}\frac{\phi(\zeta)}{\zeta - z}\,d\zeta - \int_{\del B_{\varepsilon}(z)}\frac{\phi(\zeta)}{\zeta - z}\,d\zeta = \int_{\Omega_\varepsilon}\frac{(\bar{\del}\phi)(\zeta)}{\zeta - z}\,d\bar{\zeta}\wedge d\zeta
\]
As we let $\varepsilon\to 0$, 
\[
\int_{\del\Omega}\frac{\phi(\zeta)}{\zeta - z}\,d\zeta - \underbrace{\int_{-\pi}^{\pi}\frac{\phi(z + \varepsilon e^{i\theta})}{\varepsilon e^{i\theta}}\varepsilon ie^{i\theta}\,d\theta}_{2\pi i \phi(z)} = \int_{\Omega}\frac{(\bar{\del}\phi)(\zeta)}{\zeta - z}\,d\bar{\zeta}\wedge d\zeta
\]

\defn Let $\mu$ be a measure on $\C$. We define the \underline{Cauchy transform $C\mu$} by
\[
u(\zeta) = (C\mu)(\zeta) \eqdef \int\frac{d\mu(z)}{z - \zeta}
\]
The \underline{support of $\mu$} is the smallest closed set $L_1$ with the property that $\mu(\psi) = 0$ if $\supp(\psi) \cap L_1 = \varnothing$


\thm (H\"ormander)

Let $\mu$ be a Radon measure, with compact support. Then $u(\zeta)$ is analytic on the complement of $\supp(\mu)$. Further, if $\mu = \frac{1}{2\pi i }\phi(z)dz\wedge d\bar{z}$, and $\phi\in C^{(k)}$ on an open set $\omega$, then $u \in C^{(k)}$ on $\omega$ and $\bar{\del}u = \phi$ on $\omega$. 

\proof

\qed

Let $L$ be compact. Then $C(L)^* = \mc{M}(L)$ is a collection of Radon measures $\mu$, such that $\phi\mapsto\mu(\phi)$ is linear and continuous with respect to the sup norm.

\exm Let $L = [0, 1]$, $\mu(\phi) = \sum_{n=1}^{\oo}\phi(\frac{1}{n})3^{-n}$. Then $\supp(\mu) = \{\frac{1}{n} \mid n \in \N\} \cup\{0\}$. 

\end{document}