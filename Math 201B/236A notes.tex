\documentclass[x11names,reqno,14pt]{extarticle}
% Choomno Moos
% Portland State University
% Choom@pdx.edu


%% stupid experiment %%
%%%%%%%%%%%%% PACKAGES %%%%%%%%%%%%%

%%%% SYMBOLS AND MATH %%%%
\let\oldvec\vec
\usepackage{authblk}	% author block customization
\usepackage{microtype}	% makes stuff look real nice
\usepackage{amssymb} 	% math symbols
\usepackage{siunitx} 	% for SI units, and the degree symbol
\usepackage{mathrsfs}	% provides script fonts like mathscr
\usepackage{mathtools}	% extension to amsmath, also loads amsmath
\usepackage{esint}		% extended set of integrals
\mathtoolsset{showonlyrefs} % equation numbers only shown when referenced
\usepackage{amsthm}		% theorem environments
\usepackage{relsize}	%font size commands
\usepackage{bm}			% provides bold math
\usepackage{bbm}		% for blackboard bold 1

%%%% FIGURES %%%%
\usepackage{graphicx} % for including pictures
\usepackage{float} % allows [H] option on figures, so that they appear where they are typed in code
\usepackage{caption}
\usepackage{hyperref}
%\usepackage{titling}
\usepackage{tikz} % for drawing
\usetikzlibrary{shapes,arrows,chains,positioning,cd,decorations.pathreplacing,decorations.markings,hobby,knots,braids}
\usepackage{subcaption}	% subfigure environment in figures

%%%% MISC %%%%
\usepackage{enumitem} % for lists and itemizations
\setlist[enumerate]{leftmargin=*,label=\bf \arabic*.}

\usepackage{multicol}
\usepackage{multirow}
\usepackage{url}
\usepackage[symbol]{footmisc}
\renewcommand{\thefootnote}{\fnsymbol{footnote}}
\usepackage{lastpage} % provides the total number of pages for the "X of LastPage" page numbering
\usepackage{fancyhdr}
\usepackage{manfnt}
\usepackage{nicefrac}
%\usepackage{fontspec}
%\usepackage{polyglossia}
%\setmainlanguage{english}
%\setotherlanguages{khmer}
%\newfontfamily\khmerfont[Script=Khmer]{Khmer Busra}

%%% Khmer script commands for math %%%
%\newcommand{\ka}{\text{\textkhmer{ក}}}
%\newcommand{\ko}{\text{\textkhmer{ត}}}
%\newcommand{\kha}{\text{\textkhmer{ខ}}}

%\usepackage[
%backend=biber,
% numeric
%style=numeric,
% APA
%bibstyle=apa,
%citestyle=authoryear,
%]{biblatex}

\usepackage[explicit]{titlesec}
%%%%%%%% SOME CODE FOR REDECLARING %%%%%%%%%%

\makeatletter
\newcommand\RedeclareMathOperator{%
	\@ifstar{\def\rmo@s{m}\rmo@redeclare}{\def\rmo@s{o}\rmo@redeclare}%
}
% this is taken from \renew@command
\newcommand\rmo@redeclare[2]{%
	\begingroup \escapechar\m@ne\xdef\@gtempa{{\string#1}}\endgroup
	\expandafter\@ifundefined\@gtempa
	{\@latex@error{\noexpand#1undefined}\@ehc}%
	\relax
	\expandafter\rmo@declmathop\rmo@s{#1}{#2}}
% This is just \@declmathop without \@ifdefinable
\newcommand\rmo@declmathop[3]{%
	\DeclareRobustCommand{#2}{\qopname\newmcodes@#1{#3}}%
}
\@onlypreamble\RedeclareMathOperator
\makeatother

\makeatletter
\newcommand*{\relrelbarsep}{.386ex}
\newcommand*{\relrelbar}{%
	\mathrel{%
		\mathpalette\@relrelbar\relrelbarsep
	}%
}
\newcommand*{\@relrelbar}[2]{%
	\raise#2\hbox to 0pt{$\m@th#1\relbar$\hss}%
	\lower#2\hbox{$\m@th#1\relbar$}%
}
\providecommand*{\rightrightarrowsfill@}{%
	\arrowfill@\relrelbar\relrelbar\rightrightarrows
}
\providecommand*{\leftleftarrowsfill@}{%
	\arrowfill@\leftleftarrows\relrelbar\relrelbar
}
\providecommand*{\xrightrightarrows}[2][]{%
	\ext@arrow 0359\rightrightarrowsfill@{#1}{#2}%
}
\providecommand*{\xleftleftarrows}[2][]{%
	\ext@arrow 3095\leftleftarrowsfill@{#1}{#2}%
}
\makeatother

%%%%%%%% NEW COMMANDS %%%%%%%%%%

% settings
\newcommand{\N}{\mathbb{N}}                     	% Natural numbers
\newcommand{\Z}{\mathbb{Z}}                     	% Integers
\newcommand{\Q}{\mathbb{Q}}                     	% Rationals
\newcommand{\R}{\mathbb{R}}                     	% Reals
\newcommand{\C}{\mathbb{C}}                     	% Complex numbers
\newcommand{\K}{\mathbb{K}}							% Scalars
\newcommand{\F}{\mathbb{F}}                     	% Arbitrary Field
\newcommand{\E}{\mathbb{E}}                     	% Euclidean topological space
\renewcommand{\H}{{\mathbb{H}}}                   	% Quaternions / Half space
\newcommand{\RP}{{\mathbb{RP}}}                       % Real projective space
\newcommand{\CP}{{\mathbb{CP}}}                       % Complex projective space
\newcommand{\Mat}{{\mathrm{Mat}}}						% Matrix ring
\newcommand{\M}{\mathcal{M}}
\newcommand{\GL}{{\mathrm{GL}}}
\newcommand{\SL}{{\mathrm{SL}}}

\newcommand{\tgl}{\mathfrak{gl}}
\newcommand{\tsl}{\mathfrak{sl}}                  % Lie algebras; i.e., tangent space of SO/SL/SU
\newcommand{\tso}{\mathfrak{so}}
\newcommand{\tsu}{\mathfrak{sl}}


% typography
\newcommand{\noi}{\noindent}						% Removes indent
\newcommand{\tbf}[1]{\textbf{#1}}					% Boldface
\newcommand{\mc}[1]{\mathcal{#1}}               	% Calligraphic
\newcommand{\ms}[1]{\mathscr{#1}}               	% Script
\newcommand{\mbb}[1]{\mathbb{#1}}               	% Blackboard bold


% (in)equalities
\newcommand{\eqdef}{\overset{\mathrm{def}}{=}}		% Definition equals
\newcommand{\sub}{\subseteq}						% Changes default symbol from proper to improper
\newcommand{\psub}{\subset}						% Preferred proper subset symbol

% Categories
\newcommand{\catname}[1]{{\text{\sffamily {#1}}}}

\newcommand{\Cat}{{\catname{C}}}
\newcommand{\cat}[1]{{\catname{\ifblank{#1}{C}{#1}}}}
\newcommand{\CAT}{{\catname{Cat}}}
\newcommand{\Set}{{\catname{Set}}}

\newcommand{\Top}{{\catname{Top}}}
\newcommand{\Met}{{\catname{Met}}}
\newcommand{\PL}{{\catname{PL}}}
\newcommand{\Man}{{\catname{Man}}}
\newcommand{\Diff}{{\catname{Diff}}}

\newcommand{\Grp}{{\catname{Grp}}}
\newcommand{\Grpd}{{\catname{Grpd}}}
\newcommand{\Ab}{{\catname{Ab}}}
\newcommand{\Ring}{{\catname{Ring}}}
\newcommand{\CRing}{{\catname{CRing}}}
\newcommand{\Mod}{{\mhyphen\catname{Mod}}}
\newcommand{\Alg}{{\mhyphen\catname{Alg}}}
\newcommand{\Field}{{\catname{Field}}}
\newcommand{\Vect}{{\catname{Vect}}}
\newcommand{\Hilb}{{\catname{Hilb}}}
\newcommand{\Ch}{{\catname{Ch}}}

\newcommand{\Hom}{{\mathrm{Hom}}}
\newcommand{\End}{{\mathrm{End}}}
\newcommand{\Aut}{{\mathrm{Aut}}}
\newcommand{\Obj}{{\mathrm{Obj}}}
\newcommand{\op}{{\mathrm{op}}}

% Norms, inner products
\delimitershortfall=-1sp
\newcommand{\widecdot}{\, \cdot \,}
\newcommand\emptyarg{{}\cdot{}}
\DeclarePairedDelimiterX{\norm}[1]{\Vert}{\Vert}{\ifblank{#1}{\emptyarg}{#1}}
\DeclarePairedDelimiterX{\abs}[1]\vert\vert{\ifblank{#1}{\emptyarg}{#1}}
\DeclarePairedDelimiterX\inn[1]\langle\rangle{\ifblank{#1}{\emptyarg,\emptyarg}{#1}}
\DeclarePairedDelimiterX\cur[1]\{\}{\ifblank{#1}{\emptyarg,\emptyarg}{#1}}
\DeclarePairedDelimiterX\pa[1](){\ifblank{#1}{\emptyarg}{#1}}
\DeclarePairedDelimiterX\brak[1][]{\ifblank{#1}{\emptyarg}{#1}}
\DeclarePairedDelimiterX{\an}[1]\langle\rangle{\ifblank{#1}{\emptyarg}{#1}}
\DeclarePairedDelimiterX{\bra}[1]\langle\vert{\ifblank{#1}{\emptyarg}{#1}}
\DeclarePairedDelimiterX{\ket}[1]\vert\rangle{\ifblank{#1}{\emptyarg}{#1}}

% mathmode text operators
\RedeclareMathOperator{\Re}{\operatorname{Re}}		% Real part
\RedeclareMathOperator{\Im}{\operatorname{Im}}		% Imaginary part
\DeclareMathOperator{\Stab}{\mathrm{Stab}}
\DeclareMathOperator{\Orb}{\mathrm{Orb}}
\DeclareMathOperator{\Id}{\mathrm{Id}}
\DeclareMathOperator{\vspan}{\mathrm{span}}			% Vector span
\DeclareMathOperator{\tr}{\mathrm{tr}}
\DeclareMathOperator{\adj}{\mathrm{adj}}
\DeclareMathOperator{\diag}{\mathrm{diag}}
\DeclareMathOperator{\eq}{\mathrm{eq}}
\DeclareMathOperator{\coeq}{\mathrm{coeq}}
\DeclareMathOperator{\coker}{\mathrm{coker}}
\DeclareMathOperator{\dom}{\mathrm{dom}}
\DeclareMathOperator{\cod}{\mathrm{codom}}
\DeclareMathOperator{\im}{\mathrm{im}}
\DeclareMathOperator{\Dim}{\mathrm{dim}}
\DeclareMathOperator{\codim}{\mathrm{codim}}
\DeclareMathOperator{\Sym}{\mathrm{Sym}}
\DeclareMathOperator{\lcm}{\mathrm{lcm}}
\DeclareMathOperator{\Inn}{\mathrm{Inn}}
\DeclareMathOperator{\sgn}{sgn}						% sgn operator
\DeclareMathOperator{\intr}{\text{int}}             % Interior
\DeclareMathOperator{\co}{\mathrm{co}}				% dual/convex Hull
\DeclareMathOperator{\Ann}{\mathrm{Ann}}
\DeclareMathOperator{\Tor}{\mathrm{Tor}}


% misc symbols
\newcommand{\divides}{\big\lvert}
\newcommand{\grad}{\nabla}
\newcommand{\veps}{\varepsilon}						% Preferred epsilon
\newcommand{\vphi}{\varphi}
\newcommand{\del}{\partial}							% Differential/Boundary
\renewcommand{\emptyset}{\text{\O}}					% Traditional emptyset symbol
\newcommand{\tril}{\triangleleft}					% Quandle operation
\newcommand{\nabt}{\widetilde{\nabla}}				% Contravariant derivative
\newcommand{\later}{$\textcolor{red}{\blacksquare}$}% Laziness indicator

% misc
\mathchardef\mhyphen="2D							% mathomode hyphen
\renewcommand{\mod}[1]{\ (\mathrm{mod}\ #1)}
\renewcommand{\bar}[1]{\overline{#1}}				% Closure/conjugate
\renewcommand\qedsymbol{$\blacksquare$} 			% Changes default qed in proof environment
%%%%% raised chi
\DeclareRobustCommand{\rchi}{{\mathpalette\irchi\relax}}
\newcommand{\irchi}[2]{\raisebox{\depth}{$#1\chi$}}
\newcommand\concat{+\kern-1.3ex+\kern0.8ex}

% Arrows
\newcommand{\weak}{\rightharpoonup}					% Weak convergence
\newcommand{\weakstar}{\overset{*}{\rightharpoonup}}% Weak-star convergence
\newcommand{\inclusion}{\hookrightarrow}			% Inclusion/injective map
\renewcommand{\natural}{\twoheadrightarrow}				% Natural map
\newcommand{\oo}{\infty}

% Environments
\theoremstyle{plain}
\newtheorem{thm}{Theorem}[section]
%\newtheorem{lem}[thm]{Lemma}
\newtheorem{lem}{Lemma}
\newtheorem*{lems}{Lemma}
\newtheorem{cor}[thm]{Corollary}
\newtheorem{prop}{Proposition}
\newtheorem*{claim}{Claim}
\newtheorem*{cors}{Corollary}
\newtheorem*{props}{Proposition}
\newtheorem*{conj}{Conjecture}

\theoremstyle{definition}
\newtheorem{defn}{Definition}[section]
\newtheorem*{defns}{Definition}
\newtheorem{exm}{Example}[section]
\newtheorem{exer}{Exercise}[section]

\theoremstyle{remark}
\newtheorem*{rem}{Remark}

\newtheorem*{solnx}{Solution}
\newenvironment{soln}
    {\pushQED{\qed}\renewcommand{\qedsymbol}{$\Diamond$}\solnx}
    {\popQED\endsolnx}%

% Macros

\newcommand{\restr}[1]{_{\mkern 1mu \vrule height 2ex\mkern2mu #1}}
\newcommand{\Upushout}[5]{
    \begin{tikzcd}[ampersand replacement = \&]
    \&#2\ar[rd,"\iota_{#2}"]\ar[rrd,bend left,"f"]\&\&\\
    #1\ar[ur,"#4"]\ar[dr,"#5"]\&\&#2\oplus_{#1} #3\ar[r,dashed,"\vphi"]\&Z\\
    \&#3\ar[ur,"\iota_{#3}"']\ar[rru,bend right,"g"']\&\&
    \end{tikzcd}
}
\newcommand{\exactshort}[5]{
		\begin{tikzcd}[ampersand replacement = \&]
			0\ar[r]\&#1\ar[r,"#2"]\& #3 \ar[r,"#4"]\& #5 \ar[r]\&0
		\end{tikzcd}
}
\newcommand{\product}[6]{
		\begin{tikzcd}[ampersand replacement = \&]
			#1 \& #2 \ar[l,"#4"'] \\
			#3 \ar[u,"#5"] \ar[ur,"#6"']
		\end{tikzcd}
}
\newcommand{\coproduct}[6]{
		\begin{tikzcd}[ampersand replacement = \&]
			#1 \ar[r,"#4"] \ar[d,"#5"'] \& #2 \ar[dl,"#6"] \\
			#3
		\end{tikzcd}
}
%%%%%%%%%%%% PAGE FORMATTING %%%%%%%%%

\usepackage{geometry}
    \geometry{
		left=15mm,
		right=15mm,
		top=15mm,
		bottom=15mm	
		}

\usepackage{color} % to do: change to xcolor
\usepackage{listings}
\lstset{
    basicstyle=\ttfamily,columns=fullflexible,keepspaces=true
}
\usepackage{setspace}
\usepackage{setspace}
\usepackage{mdframed}
\usepackage{booktabs}
\usepackage[document]{ragged2e}
\usepackage{epsfig}

\pagestyle{fancy}{
	\fancyhead[L]{Winter 2023}
	\fancyhead[C]{236A}
	\fancyhead[R]{John White}
  
  \fancyfoot[R]{\footnotesize Page \thepage \ of \pageref{LastPage}}
	\fancyfoot[C]{}
	}
\fancypagestyle{firststyle}{
     \fancyhead[L]{}
     \fancyhead[R]{}
     \fancyhead[C]{}
     \renewcommand{\headrulewidth}{0pt}
	\fancyfoot[R]{\footnotesize Page \thepage \ of \pageref{LastPage}}
}
\newcommand{\pmat}[4]{\begin{pmatrix} #1 & #2 \\ #3 & #4 \end{pmatrix}}
\newcommand{\A}{\mathbb{A}}
\newcommand{\fin}{``\in"}
\DeclareMathOperator{\Perm}{Perm}
\DeclareMathOperator{\pdim}{pdim}
\DeclareMathOperator{\gldim}{gldim}
\DeclareMathOperator{\lgldim}{lgldim}
\DeclareMathOperator{\rgldim}{rgldim}
\newcommand{\Rmod}{R-\text{mod}}
\newcommand{\onto}{\twoheadrightarrow}
\newcommand{\barf}{\bar{f}}

\newcommand{\exactlon}[5]{
		\begin{tikzcd}
			0\ar[r]&#1\ar[r,"#2"]& #3 \ar[r,"#4"]& #5 \ar[r]&0
		\end{tikzcd}
}

\title{236A - Homological Algebra}
\author{John White}
\date{Winter 2023}


\begin{document}

\section*{Lecture 1, 1/11/13}

\subsection*{\underline{Section 1: Vocabulary and easy definitions}}

Homological algebra is the study of complexes of $R$-modules, where $R$ is a ring with identity $1\neq0$. Notationally, $R$-Mod is the category of all left $R$-modules, and $R$-mod is the category of all finitely generated $R$-modules. 

\defn

Let $A_n\fin R$-mod for $n\in\Z$ and $d_n\in\Hom_R(A_n, A_{n - 1})$ such that $d_{n - 1}\circ d_n = 0$ for all $n \in\Z$. Then the sequence
\[
\begin{tikzcd}
\cdots\ar[r, "d_{n + 2}"]& A_{n + 1}\ar[r, "d_{n + 1}"] & A_n \ar[r, "d_n"] & A_{n - 1} \ar[r, "d_{n - 1}"] & \cdots\\
\end{tikzcd}
\]
is called a \underline{complex of $R$-modules}, assuming $\im(d_n) \subseteq \ker(d_{n - 1})$. The sequence
\[
\begin{tikzcd}
0\ar[r]&A_m\ar[r]&A_{m - 1}\ar[r]&\cdots\ar[r]&A_{n + 1}\ar[r]&A_n\ar[r]&0 \\
\end{tikzcd}
\]
will occur more frequently. A complex $\A$ is an \underline{exact sequence} if $\im(d_n) = \ker(d_{n - 1})$ for all $n \in \Z$. This is called a \underline{short exact sequence} if there are no more than 3 non-zero terms. Given a complex $\A$, the \underline{nth homology modules} (or groups, in some cases) of $\A$ is
\[
H_n(\A) = \frac{\ker(d_{n -1})}{\im(d_n)}
\]

\rem

Given a short exact sequence (hereby abbv. as SES) 
\[
\begin{tikzcd}
0\ar[r]&A\ar[r,"f"]&B\ar[r,"g"]&C\ar[r]&0
\end{tikzcd}
\]
$f$ is a mono and $g$ is an epi, so $C\simeq B/\im(f)$. If $A, B$ are known, but not $f$, then infinitely many $C$ are available to complete the short exact sequence.

\exm 

Let $R = k$, a field, and take $A = B = k^{(\N)} = \oplus_{i\in\N}k$.

\begin{itemize}
\item[(i)] $\begin{tikzcd} 0\ar[r]&A\ar[r, "\Id"]&B\ar[r]&0 \end{tikzcd}$ is a SES.
\item[(ii)] Define $f:A\to B$ by 

\begin{align*}
f(b_i) & = b_{2i} \text{ for }i\in \N \\
g(b_0) & = \begin{cases} 0 & i\text{ even } \\ b_{\tau(i)} & i\text{ odd } \end{cases}
\end{align*}

Where $\tau:(2\N-1)\to\N$ is a bijection. If $A = B = C = \kappa^{(\N)}$, then 
\[
\exactshort{A}{f}{B}{g}{C}
\]
is a SES. 
\item[(iii)] Let $R = \Z$. Then 
\[
\exactshort{3\Z}{\iota}{\Z}{\bar{-}}{\Z/3\Z}
\]
is a SES. 
\item[(iv)] Let $R = \Z$. The sequence 
\[
\exactshort{\overbrace{6\Z}^{A_1}}{\iota}{\overbrace{\Z}^{A_0}}{\bar{-}}{\overbrace{\Z/3\Z}^{A_{-1}}}
\]
is a complex which is not exact. In fact, $H_0(\A) = \overbrace{3\Z}^{\ker(g)}/\underbrace{6\Z}_{\im(f)} \cong \Z/2\Z$. 

\item[(v)] Let $R = \kappa[x, y]$, $\kappa$ a field. Let $f$ be the inclusion $(x)\hookrightarrow R[x, y]$. The sequence
\[
\exactshort{(x)}{f}{R}{g}{\kappa[y]}
\]
where
\[
g\left(\sum_{i, j = 0}^{\text{finite}}a_{ij}x^iy^j\right) = \sum_{j > 0}^{\text{finite}}a_{\sigma_j}y^j
\]
is exact. 

\item[(vi)] Let $R = \kappa[x, y]$. Define $\A$ as
\[
\exactshort{\overbrace{(x)}^{A_1}}{f}{\overbrace{R}^{A_0}}{g}{\overbrace{\underbrace{\kappa}_{=R/(x, y)}}^{A_{-1}}}
\]
where 
\[
g\left(\sum_{i, j = 0}^{\text{finite}}a_{ij}x^iy^j\right) = a_{\oo}
\]
then $\ker(g) = (x, y)$ and $\im(f) = (x)$, so $\A$ is not exact. In fact, 
\begin{align*}
H_0(\A) & = (x, y)/(x) \\
		  & \simeq (y) \\
		  & \simeq R \\
\end{align*}


\end{itemize}

Note: If $R$ is an integral domain and $x\in R\setminus\{0\}$, then $(x)\simeq R$ (as $R$-modules, \underline{not} as rings!), with isomorphism $r\mapsto rx$. 

Typical questions addressed by homological algebra: 

\begin{itemize}
\item[(i)] Suppose
\[\A:
\begin{tikzcd}
\cdots\ar[r]&A_{n + 1}\ar[r, "d_{n + 1}"]&A_n\ar[r,"d_n"]&A_{n - 1}\ar[r]&\cdots \\
\end{tikzcd}
\]
is an exact sequence in $R$-mod and $F:\Rmod\to S-\text{mod}$ is a functor. Is the sequence
\[
\begin{tikzcd}
\cdots\ar[r]&F(A_{n + 1})\ar[r, "F(d_{n + 1})"] &F(A_n)\ar[r,"F(d_n)"] &F(A_{n - 1})\ar[r] &\cdots
\end{tikzcd}
\]
exact? $F(\A)$ is a complex when $F$ is additive, but it may or may not be exact. 

\item[(ii)] Given $A, C\fin \Rmod$, characterize all modules $B$ such that there exists an exact sequence
\[
\exactshort{A}{}{B}{}{C}
\]
As an example, $R =\Z, A = C = \A/p\Z$, $p$ prime, then
\[
\exactshort{\Z/p\Z}{f}{\Z/p\Z\oplus\Z/p\Z}{g}{\Z/p\Z}
\]
with $f:x\mapsto(x,0)$ and $g:(x,y)\mapsto y$ is a SES. Alternatively, we could take $f:x + p\Z\mapsto px + p^2\Z$ and $g:y + p^2\Z\mapsto y + p\Z$ to make 
\[
\exactshort{\Z/p\Z}{f}{\Z/p^2\Z}{g}{\Z/p\Z}
\]
a SES. These are the only possibilities in this case! In general, though, there are infinitely many possibilities for $B$. Why is this interesting? If $R$ is an artinian ring and $M\fin\Rmod$, then there are only finitely many simple $s_1,\dots,s_n\fin\Rmod$ up to isomorphism. Moreover, for $M\fin\Rmod$, there is a chain
\[
M = M_0\supset M_1 \supset \cdots \supset M_{\ell} = 0
\]
such that $M_i/M_{i + 1}$ is simple for all $i<\ell$. If the answer to question $(ii)$ is known, then all objects in $\Rmod$ of fixed length $\ell$ are known up to isomorphism! Simply proved by induction.
\end{itemize}

\subsection*{\underline{Algebraic Topology}}

\defn
The \underline{standard $n$-simplex} $\bigtriangleup_n$ in $\R^n$ is the convex hull of $v_0, v_1, \dots, v_n$, where $v_0 = 0$ and $v_i = (0, \dots, 0, \overbrace{1}^{i}, 0, \dots, 0)$ (so the standard basis). 

An \underline{oriented simplex} is $(\bigtriangleup_x, [\pi])$, where $[\pi]$ is an equivalence class of permutations of $\{0, \dots, n\}$, where $\pi\sim\pi'\iff \sgn(\pi)=\sgn(\pi')$. We write
\[
(\bigtriangleup_x, \pi) = [v_{\pi(0)}, v_{\pi(1)},\dots,v_{\pi(n)}
\]
and identify $\bigtriangleup_n$ with $[0, 1, \dots, n]$. The ngative is $-[w_0, \dots, w_n]$. 

\defn

Let $X$ be a topological space. An \underline{$n$-simplex in $X$} is a continuous map
\[
\sigma:\bigtriangleup_n\to X
\]
The \underline{group of $n$-chains of $X$}, $S_n(x)$, is the free abelian group having as basis the $n$-simplices in $X$. The \underline{singular chain complex of $X$} is
\[
\begin{tikzcd} 
\cdots\ar[r]&S_n(X)\ar[r,"\del_n"] &S_{n -1}(X)\ar[r, "\del_{n - 1}"] &\cdots\ar[r] &S_0(X)\ar[r] &0
\end{tikzcd}
\]
denoted $\mbb{S}$, where $\del_n:S_n(X)\to S_{n - 1}(X)$ is the \underline{$n$th boundary map}, which can be defined if we define $\del_n(\sigma)$ for all $n$-simplices $\sigma$ in $X$ (i.e. in the basis of $S_n(X)$). Consider the map
\begin{align*}
\tau_i:\R^{n - 1} & \to \R^n \\
(a_1, \dots, a_{n - 1})&\mapsto(a_1, \dots, a_{i - 1}, 0, a_{i + 1}, \dots, a_n) \\
\end{align*}
For $i \in \{0, \dots, n\}$. Then $\tau_i$ is continuous and $\tau_i(\bigtriangleup_{n - 1}) = \bigtriangleup_n$. Define
\[
\del_n(\sigma) = \sum_{i = 0}^n(-1)^i\sigma(\tau_i)
\]

\thm $\del_{n - 1}\circ\del_n = 0$ for all $n \in \N$, i.e. $\mbb{S}$ is a complex in $\Z$-mod. 

\defn

The \underline{group of $n$-cycles} is $Z_n(X) = \ker(\del_{n - 1})$, and the \underline{group of $n$-boundaries} is $B_n = \im(\del_n)$. 

The \underline{$n$th homology group} is $H_n(X) = Z_n(X)/B_n(X)$.

\section*{Lecture 2, 1/13/23}

\subsection*{\underline{Chapter I: Categories and functors}}

There is a definition page on the Gaucho that has all the most basic definitions - objects, morphisms, compositions, etc. 

If $f \in \Hom_C(A, B)$, we often write $\begin{tikzcd} A\ar[r, "f"] & B \end{tikzcd}$ even if $f$ is not literally a map. 

\exm

\begin{enumerate}

\item The category of all sets, $\Set$. The object class consists of all sets, and the morphisms are just set maps. 

\item The category of all topological spaces, $\Top$. The object class consists of all topological spaces, and the morphisms are continuous functions. 

\item The category of all groups, $\Grp$. The object class consists of all groups, and the morphisms are group homomorphisms. 

\item Let $(P, \leq)$ be a partially ordered set with a relation $\leq$ which is reflexive, antisymmetric,  and transitive. Then we can make $P$ into a category, whose objects are the elements of $p$, and for $u, s \in P$, $\Hom_P(u, s) = \begin{cases} (u, s) & u \leq s \\ \varnothing & u\not\leq s \\ \end{cases}$. We define the composition $(s, t)(u, s) \eqdef (u, t)$. 

\item The opposite category of a category $C$, $C^\op$. 

\item Let $R$ be a ring. $R$-Mod is the category of left $R$ modules. $R$-mod is the finitely generated $R$-modules, and similarly for Mod-$R$ and mod-$R$, which are the right $R$-modules.

\item $R$-comp. The object class consists of complexes of left $R$-modules.

Let $\mathbb{A}, \mathbb{A}'$ be objects of $R$-comp. Note: it is problematic to say ``$\mathbb{A}, \mathbb{A}' \in R$-comp, as $R$-comp is not a set!

Say $\mathbb{A} = \begin{tikzcd} \cdots\ar[r] &A_{n + 1} \ar[r, "d_{n + 1}"] & A_n \ar[r, "d_n"] & A_{n - 1} \ar[r]& \cdots \end{tikzcd}$, and similarly for $\mathbb{A}'$. An element of $\Hom_{R-comp}(\mathbb{A},\mathbb{A}')$ will be a sequence of $R$-module homomorphisms $f_n:A_n\to A'_n$ which make the following diagram commute:

\[
\begin{tikzcd}
\cdots \ar[r] \ar[d] & A_{n} \ar[d, "f_n"'] \ar[r, "d_{n}"] & A_{n - 1} \ar[r] \ar[d, "f_{n - 1}"] & \ar[d]\cdots  \\
\cdots \ar[r] & A'_n \ar[r, "d'_n"'] & A'_{n - 1} \ar[r] & \cdots \\
\end{tikzcd}
\]

\item The category of rings Ring, whose obejcts are rings and whose morphisms are ring homomorphisms. 

\item The category of $\Z$-modules is usually denoted $\Ab$. This is also the category of Abelian groups, and is the prototypical example of an Abelian category.

\end{enumerate}

\defn A category $\mc{C}$ is called \underline{pre-additive} if for all $A, B$ objects of $\mc{C}$, the set $\Hom_{\mc{C}}(A, B)$ is an additive Abelian group (additive means we use the symbol ``$+$") such that for all eligible morphisms $f, g, h, k$, 
\begin{align*}
h(f + g) & = hf + hg \\
(f + g)k & = fk + gk \\
\end{align*}
where ``elibigle" means that these expressions make sense and are well-defined. 

\exm

\begin{enumerate}

\item $R$-mod (in particular $\Ab$)

\item $R$-comp

\item Ring fails to be pre-additive, because the identity morphisms add to be something which is not the identity morphism. 

\end{enumerate}

\defn

Let $\mc{C}, \mc{D}$ be categories. A \underline{functor} $F:\mc{C}\to\mc{D}$ consists of an assignment $F_0:\Obj(\mc{C})\to\Obj(\mc{D})$, and for each pair of objects $A, B ``\in" \Obj(\mc{C})$, a map (this actually is a map because we assume hom-sets are in fact sets). $F_{A, B}:\Hom_{\mc{C}}(A, B) \to \Hom_{\mc{D}}(F(A), F(B))$ such that, for all eligible morphisms $f, g$, and all $A \fin \mc{C}$
\begin{enumerate}[label=(\alph*)]

\item $F(\Id_A) = \Id_{F(A)}$
\item $F(f\circ g) = F(f)\circ F(g)$

\end{enumerate}

\exm

\begin{enumerate}

\item Let $\mc{C}$ be a category. Then we have the identity functor $\Id_{\mc{C}}$, which assigns $\Id_{\mc{C}}(A) = A$, and $\Id_{\mc{A}}(f) = f$ for any eligible $A \fin\Obj(\mc{D})$ and morphisms $f$. 

\item Functors $\pi_n:\Top\to\Grp$ which sends $X\mapsto\pi_n(X)$

\item $\mathbb{S}:\Top\to\Z$-comp, which sends $X\mapsto \mathbb{S}(X)$, which is a complex
\[
\begin{tikzcd}
\cdots \ar[r]&   S_{n + 1}(X)  \ar[r, "\del_{n + 1}"] &  S_n(X) \ar[r, "\del_n"]& S_{n - 1}(X) \ar[r, "\del_{n - 1}"] &\cdots \ar[r] &\ar[r,"\del_0"] S_0(x) & 0\\ 
\end{tikzcd}
\]
Let $\phi:X\to Y$ be continuous for $X, Y \fin \Top$. Then $\mathbb{S}(\phi)_n:S_n(X)\to S_n(Y)$ is given by $\sigma\mapsto\phi\circ\sigma$, and we can extend this for $\sigma$ an $n$-simplex of $X$. 

\end{enumerate}

\section*{Lecture 4, 1/18/23}

\subsection*{\underline{Functors:}}
\defn
Let $\mc{C}, \mc{D}$ be categories. A \underline{covariant functor} from $\mc{C}$ to $\mc{D}$ consists of ``maps" $F_0$ and $F|_{A, B}$ for any $A, B \in \Obj(\mc{C})$ such that
\begin{itemize}
\item $F_0:\Obj(\mc{C})\to\Obj(\mc{D})$
\item $F_{A, B}:\Hom_{\mc{C}}(A, B)\to \Hom_{\mc{D}}(F_0A,F_0B)$ for any $A, B \fin \Obj(\mc{C})$
\end{itemize}
such that 
\begin{enumerate}[label=(\alph*)]
\item $F_{A, C}(fg) = F_{B, C}(f)F_{A, B}(g)$ for all eligible $f, g$ 
\item $F_{A, A}(\Id_A) = \Id_{F(A)}$
\end{enumerate}
from here on we don't care at all about indices. For simplicity, we will denote the action of a functor $F$ as simply $FA$ or $Ff$. 

\defn

A \underline{contravariant functor} from $\mc{C}$ to $\mc{D}$ amounts to a covariant functor from $\mc{C}$ to $\mc{D}^\op$.

More examples of functors

\exm

\underline{Homology functors} $H_n:R-comp\to \Z-mod$ which sends $\A$ to $H_n(A)$. That is, $\A\to F\A = \frac{\ker(d_n)}{\Im(d_{n + 1})}$

Let $f \in \Hom_{R-comp}(\A,\A')$. That is, the following diagram commutes

\[
\begin{tikzcd}
\cdots \ar[r] \ar[d] & A_{n} \ar[d, "f_n"'] \ar[r, "d_{n}"] & A_{n - 1} \ar[r] \ar[d, "f_{n - 1}"] & \ar[d]\cdots  \\
\cdots \ar[r] & A'_n \ar[r, "d'_n"'] & A'_{n - 1} \ar[r] & \cdots \\
\end{tikzcd}
\]
$Ff$ acts by $a_n + \Im(d_{n + 1})\to f_n(a_n) + \Im(d_{n + 1}')$. Let's prove that this is actually well-defined. 

\underline{Check}

First, $a_n \in \ker(d_n)$ implies $f_n(a_n) \in \ker(d_n')$. This can be seen by doing a diagram chase on the above diagram. Since $d_n(a_n) = 0$, we have $0 = f_{n - 1}d_n(a_n) = d_n'f_n(a_n)$, i.e. $f_n(a_n)\in\ker(d_n')$. 


``Don't do much thinking. It's almost harmful" - Birge on doing diagram chasing. Also ``follow your nose."

Now, $a_n\in\Im(d_{n + 1})$ implies $f_n(a_n)\in\Im(d_{n + 1}')$. So $a_n = d_{n + 1}(x)$ with $x \in A_{n + 1}$. hence $f_n(a_n) = f_nd_{n + 1}(x) = a_{n + 1}'f_{n + 1}(x) \in \Im(d_{n + 1}')$. 

\exm

Let $\mc{C}, \mc{D}$ be pre-additive categories (definition on the top of page 6). A functor $F$ ``from" $\mc{C}$ to $\mc{D}$ is called \underline{additive} if, for all $A, B \fin \Obj(\mc{C})$, the map $F:\Hom_{\mc{C}}(A, B) \to \Hom_{\mc{D}}(F(A), F(B))$ is a homomorphism of abelian groups. 

\rem Note that $H_n:R-comp\to\Z-mod$ is an additive functor. The $\pi_n$ functor is \underline{not} additive, as $\Top$ is not preadditive. 

\exm

\underline{Forgetful functors} e.g. $F:R-mod\to\Z-mod$ which sends $M\mapsto M$, where the $M$ on the left hand side is an $R$-module, and $M$ on the right is just an abelian group, which is a $\Z$-module. Or $F:R-mod\to\Set$ which sends an $R$-module $M$ to the set of its elements, ``forgetting" the module structure. 

Moreover, if $\mc{C},\mc{D}$ are pre-additive, and $F:\mc{C}\to\mc{D}$ is a forgetful functor of some sort, then $F$ is additive. 

\exm

Let $F:R-mod\to S-mod$ be an additive functor. Then $F$ induces an additive functor $\tilde{F}:R-comp\to S-comp$, sending $\A$ to $F(\A)$. 

If $\A$ is a complex 
\[
\begin{tikzcd}
\cdots \ar[r]& A_{n + 1}\ar[r,"d_{n + 1}"] & A_n\ar[r, "d_n"] & A_{n - 1} \ar[r] & \cdots 
\end{tikzcd}
\]
then $F(\A)$ is 
\[
\begin{tikzcd}
\cdots\ar[r]&F(A_n)\ar[r,"F(d_{n + 1})"]&F(A_n)\ar[r, "F(d_n)"]&F(A_{n - 1})\ar[r]&\cdots 
\end{tikzcd}
\]
An extremely important question: if $\A$ is exact, is $F(\A)$ exact? If not, how far does it deviate from being an exact sequence? 

\exm

Let $F:\mc{C}\to\mc{D}$, $G:\mc{D}\to\mc{E}$ be functors. Then $G \circ F:\mc{C}\to\mc{E}$ is a functor. \underline{WARNING:} we use $\circ$ but this isn't actually a function composition. This is just notation!!!

$G \circ F$ acts how one might think: for $A\fin\Obj(\mc{C})$, $G\circ F(A) = G(F(A))$, and for $f\in\Hom_{\mc{C}}(A, B)$, $G \circ F(f) = G(F(f))\in\Hom_{\mc{E}}(G(F(A)), G(F(B)))$.

Of interest to us: $H_n \circ \tilde{F}$, where $F:R-mod\to S-mod$ is additive. This functor sends a complex $\A$ to $H_n(F(\A))$. This is especially of interest if $\A$ is exact, but $F(\A)$ is not. 

\rem

Let $F:\mc{C}\to\mc{D}$ be a functor. Then $F$ sends isomorphisms in $\mc{C}$ to isomorphisms in $\mc{D}$. This is immediate from the definition of a functor. 

\subsection*{\underline{Section 2: two types of functors that will follow us}}

\begin{enumerate}[label=(\roman*)]
\item Hom-functors: Whenever $\mc{C}$ is a category, there is a bifunctor 
\[
\Hom_{\mc{C}}(-,-):\mc{C}\times\mc{C}\to\Set
\]
which sends a pair $(A, B)$ to $\Hom_{\mc{C}}(A, B)$, and on maps (note that this is covariant in the first factor and contravariant inh the second), they act as follows. Let $f:A\to A', g:B\to B'$ be morphisms in $\mc{C}$. Then 
\[
\Hom(f, g):\Hom_{\mc{C}}(A', B)\to\Hom_{\mc{C}}(A, B')
\]
acts by $\phi\mapsto g \circ \phi \circ f$
\end{enumerate}

\section*{Lecture 5, 1/20/23}

Whenever $\mc{C}$ is a category, there is a bifunctor $\Hom_{\mc{C}}(-,-):\mc{C}\times\mc{C}\to\Set$, which sends $(A, B)$ to $\Hom_{\mc{C}}(A, B)$. On maps, when $f:A\to A'$ and $g:B\to B'$ are morphisms, then
\begin{align*}
\Hom_{\mc{C}}(f, g):\Hom_{\mc{C}}(A', B)& \to \Hom_{\mc{C}}(A, B') \\
\varphi&\mapsto g \circ \varphi \circ f \\
\end{align*}
We will split this into two parts. Let $C\fin\mc{C}$. Then we have a covariant functor 
\begin{align*}
\Hom_{\mc{C}}(C,-):\mc{C}&\to\mc{C}\\
C'&\mapsto\Hom_{\mc{C}}(C, C') \\
g&\mapsto\Hom_{\mc{C}}(C, A) \to\Hom_{\mc{C}}(C, B) \\
& \varphi  \mapsto g \circ \varphi \\
\end{align*}
We also have the contravariant functor $\Hom_{\mc{C}}(-, D)$, which acts similarly. 

As a special case, consider $\mc{C} = R-mod$. Then
\begin{align*}
\Hom_R(M, -):R-mod & \to \Z-mod \\
\Hom_R(-, N):R-mod & \to \Z-mod \\
\end{align*}
but we can have additional structure on $\Hom_R(M, N)$. Suppose $_RM_S$ is a bimodule ($S$ is a ring and $(rm)s = r(ms)$) and let $_RN_T$ be an $R-T$ module. Then $\Hom_R(M, N)$ is a left $S$, right $T$ bimodule. For $f \in \Hom_R(M, n)$, $s \in S, t \in T$, define 
\begin{align*}
(sf)(m) & = f(ms) \\
(ft)(m) & = f(m)t \\ 
\end{align*}
If $R$ is commutative, then 
\begin{align*}
\Hom_R(M, -):R-mod & \to R-mod = Mod-R \\
\Hom_R(-, N):R-mod & \to R-mod = Mod-R \\
\end{align*}

If $_RM_S$ is a bimodule, then 
\[
\Hom_R(M, -):R-mod \to S-mod
\]
If $_RN_T$ is a bimodule, then
\[
\Hom(-, N):R-mod\to Mod-T
\]
Basic properties: 
\begin{enumerate}[label=(\roman*)]
\item $M\fin R-mod \implies \underbrace{\Hom_R(R, M) \cong M}_{f\mapsto f(1)}$ in $\Z-mod$

\item $\Hom_R(\otimes_{i\in I}M_i, N) \cong \prod_{i\in I}\Hom_R(M_i, N)$. Prove this! 
\item $\Hom_R(M, \prod_{i\in I}N_i)\cong\prod_{i\in I}\Hom_R(M, N_i)$. Prove this!
\end{enumerate}

\defn

Let $M \fin Mod-R$, $N\fin R-mod$. Then an abelian group $T$ is called a \underline{tensor product} of $M$ and $N$ if there exists a map 
\[
\tau:M\times N \to T
\]
Which is $\Z$-bilinear and \underline{$R$-balanced}, i.e. 
\[
\tau(mr, n) = \tau(m, rn)
\]
with the following universal property. 

Whenever $A$ is an abelian group and $\sigma:M\times N \to A$ is $\Z$-bilinear and $R$-balanced, there exists a unique $\Z$-linear map $\sigma':T\to A$ such that this diagram commutes:
\[
\begin{tikzcd}
M\times N \ar[rd, "\sigma"'] \ar[r, "\tau"] & T \ar[d, "\sigma'"] \\
& A \\ 
\end{tikzcd}
\]
We denote $T = M\otimes_RN$. 

\thm If $M \fin Mod-R$ and $N\fin R-mod$, then a tensor product $M\otimes_RN$ exists and is unique up to isomorphism. 

\proof

Let $F$ be the free abelian group with basis $M\times N$, i.e.
\[
F = \otimes_{m\in M, n\in N}\Z(m, n)
\]
Define
\[
M\otimes_RN = F/U
\]
where $U$ is the submodule generated by all elements of the form 
\begin{align*}
(m_1 + m_2, n) - (m, n) - (m_2, n) \\
(m, n_1 + n_2) - (m, n_1) - (m, n_2) \\
(mr, n) - (m, rn) \\
\end{align*}
for all eligible $m_i, m \in M, n_i, n \in N, r \in R$. 

Define
\begin{align*}
\tau:&M\times N\to M\otimes_RN \\
&(m, n)\mapsto m\otimes n \\
\end{align*}
Then $\tau$ is $\Z$-bilinear and $R$-balanced (check!). Moreover, $M\otimes_RN$ with $\tau$ satisfies the universal proeprty: let $A$ be an abelian group and $\sigma:M\times N\to A$ be $\Z$-bilinear and $R$-balanced. Define
\begin{align*}
\tilde{\sigma}:&F\to A \\
& (m, n) \mapsto \sigma(m, n) \\
\end{align*}
and extend linearly. By construction, $\tilde{\sigma}(U) = 0$, i.e. $U \subseteq \ker(\tilde{\sigma})$. Hence there exists $\sigma':F/U\to A$ with the property that 
\[
\sigma'(m, n) = \tilde{\sigma}((m, n) + n) = \tilde{\sigma}(m\otimes n)
\]
Now show $\sigma'$ is unique, and the proof is complete. 

\qed


\section*{Lecture 6, 1/23/23}

Our two mainstay types of functors: 

\begin{enumerate}[label=(\roman*)]

\item Hom functors. 

\item Tensor functors. For $(M, N) \fin Mod-R\times R-mod$, we constructed an abelian group $M\otimes_R N = R^{(M\times N)}/u$, together with $\tau:M\times N \to M\otimes_RN$ given by $\tau(m, n) = m\otimes n = (m, n) + u$ such that $(M\otimes_RN,\tau)$ has the key universal property. 
\end{enumerate}

\underline{Note:} The elements $m\otimes n\in M\times N$ form a generating set of $M\otimes_RN$, but not a basis. 

\subsection*{\underline{The tensor functor}}

We have a bifunctor $-\otimes-:Mod-R\times R-mod \to \Z-mod$, $(M, N)\mapsto M\otimes_RN$. Let $(f, g), f \in \Hom_R(M,M'), g \in \Hom(N, N')$. Then 
\begin{align*}
f\otimes g: & M\otimes_RN \to M'\otimes_RN' \\
			   & m\otimes n \mapsto f(m)\otimes g(n) \\
\end{align*}
To show this is well-defined, check that $\phi:M\times N \to M'\otimes N'$, $(m, n)\mapsto f(m)\otimes g(n)$ is $\Z$-bilinear and $R$-balanced. 

Split $-\otimes_R-$ into two functors. So, we have a functor $M\otimes_R-:R-mod\to\Z-mod$ and a functor $-\otimes_RN:Mod-R\to\Z-mod$. The action on objects and morphisms is clear from the discussion up to now. 

\subsection*{\underline{Additional structure on $M\otimes_RN$}} 

Suppose $_SM_R$ and $_RN_T$ are bimodules. Then $M\otimes_RN$ is a $S-T$ bimodule, with
\[
s(m\otimes n)t = (sm)\otimes(nt)
\]
It is an exercise to check well-definedness. 

\underline{Uses}

Suppose $_\R V$ is a real vector space. We want to ``complexify" $V$, making it a complex vector space. We could consider $\C\times V$, and define $c(d, v) = (cd, v)$. But this does not define a $\C$-vector space, because multiplication must be multilinear. But $\C\otimes_\R V$ will do it. 

\underline{Basic properties}

Consider $R\otimes_RM$. This is in fact isomorphic to $M$. Not just as Abelian groups, but as left $R$-modules. This is because $R$ satisfies the associative law relative to multiplication. One isomorphism between them is $m\mapsto 1\otimes m$.

In general, unlike the hom-functor, the tensor functor will \underline{not} commute with direct products/coproducts, unless ``the sky is very benevolent."



\dbend\dbend\dbend


The meaning of $m\otimes n$ depends on the meaning of $M, N$!

\exm 

Consider $2\otimes\bar{1}\in\Z\otimes_\Z(\Z/2\Z)$. This is the same as $1\otimes\bar{2} = 1\otimes0 = 0$.

By contrast, look at $2\otimes\bar{1} \in 2\Z\otimes(\Z/2\Z)$. This is nonzero! Let's show that. We know $2\Z\cong\Z$, with an isomorphism given by $x\mapsto\frac{x}{2}$. So 
\begin{align*}
f\otimes\Id_{\Z/2\Z}:& 2\Z\otimes(\Z/2\Z)\to\Z\otimes(\Z/2\Z) \\
							& x\otimes y \mapsto f(x)\otimes y \\
\end{align*}
But functors take isomorphisms to isomorphisms, so $\underbrace{2\otimes\bar{1}}_{\in2\Z\otimes\Z/2\Z}\mapsto\overbrace{1\otimes\bar{1}}^{\in\Z/2\Z}\neq0$. Why is this last term nonzero? Because $\Z\otimes(\Z/2\Z) \cong \Z$, with the isomorphism sending $1\otimes\bar{1}$ to $\bar{1}$, which is not zero. 

\subsection*{\underline{Natural Transformations, Equivalences, and Dualities}}

\defn

\begin{enumerate}
\item Let $F, G:\mc{C}\to\mc{D}$ be functors. A \underline{morphism of functors}, or \underline{a natural transformation} from $F$ to $G$, is a family $(\phi(C))_{C\fin\Obj(\mc{C})}$ of morphisms, $\phi(C):F(C)\to G(C)$ such that for any $f \in \Hom_{\mc{C}}(C, C')$, the square

\[
\begin{tikzcd}
\ar[d, "\phi(C)"']F(C)\ar[r,"F(f)"] & F(C')\ar[d, "\phi(C)"] \\
G(C)\ar[r, "G(f)"']& G(C') \\
\end{tikzcd}
\]
commutes for all eligible morphisms $f$ in the category $\mc{C}$. This is a covariant equivalence. A contravariant equivalence is an equivalence between contravariant functors, i.e. it makes the following square commute. 

\[
\begin{tikzcd}
\ar[d, "\phi(C)"']F(C) & \ar[l, "F(f)"']F(C')\ar[d, "\phi(C)"] \\
G(C)&\ar[l, "G(f)"] G(C') \\
\end{tikzcd}
\]

\item Call $(\phi(C))_{C\fin\Obj(\mc{C})}$ an \underline{isomorphism of functors}, or \underline{a natural equivalence}, if $\phi(C)$ is an isomorphism for each $C\fin\Obj(\mc{C})$. 

\item Two categories $\mc{C}, \mc{D}$ are \underline{equivalent categories} if there are functors $F:\mc{C}\to\mc{D}, G:\mc{D}\to\mc{C}$ such that $G \circ F \simeq \Id_{\mc{C}}$ and $F \circ G \simeq \Id_{\mc{D}}$, with $``\simeq"$ meaning ``is naturally equivalent to." The $F, G$ are called ``mutually inverse equivalences."

\item A contravariant equivalence is called a \underline{duality}.

\item Let $R, S$ be rings. Call $R, S$ \underline{Morita equivalent}, denoted $R \sim S$, if $R-mod, S-mod$ are naturally equivalent. This is equivalent to saying $mod-R, mod-S$ are equivalent. 

\end{enumerate}

\section*{Lecture 7, 1/25/23}

\defn

Let $F:\mc{C}\to\mc{D}, G:\mc{D}\to\mc{C}$ be functors. We say that $(F, G)$ form an \underline{adjoint pair} if the following two bifunctors $\mc{C}\times\mc{D}\to\Set$ are naturally isomorphic:
\[
\Hom_{\mc{D}}(F(-),-) \cong \Hom_{\mc{C}}(-,G(-))
\]
That is, for every $(C, D) \fin \mc{C}\times\mc{D}$, we have an isomorphism
\[
\phi(C, D):\Hom_{\mc{D}}(F(C), D) \to \Hom_{\mc{C}}(C, G(D))
\]
and the collection of all $\phi(C, D)$ form a natural isomorphism.

\[
\begin{tikzcd}
\Hom_{\mc{D}}(F(C), D)\ar[r]\ar[d,"\phi_{C, D}"]& \Hom_{\mc{D}}(F(C'), D')\ar[d,"\phi_{C', D'}"] \\
\Hom_{\mc{C}}(C, G(D))\ar[r] &\Hom_{\mc{C}}(C', G(D')) \\
\end{tikzcd}
\]

\exm
\,

\begin{enumerate}

\item 
\begin{enumerate}[label=(\alph*)]
\item $R\otimes_R - \cong \Id_{R-mod}$. $R\otimes_R-:R-mod\to R-mod$ is well-defined since $_RR_R$ is a bimodule. 
\item $\Hom_R(_RR_R, -) \cong \Id_{mod-R}$
\end{enumerate}

\item For any ring $R$, $R\sim M_n(R)$, where $\sim$ indicates Morita equivalence, defined above. Why? Let $_RF = R^n$, $S = \End_R(F) \cong M_n(R)$. Let $F^* = _R(\Hom(_SF, R))_S$

\claim 

The functors $\Hom(F, -):mod-R\to mod-S$, $\Hom(F^*, -):mod-S\to mod-R$ are mutually inverse functors. 

\proof

We want to show that, for $M \fin Mod-R$, 
\[
\Id_{Mod-R}\cong \Hom_S(F^*, \Hom_R(F, -))
\]
Consider
\[
\Phi(M):m\mapsto (F^* = \Hom(F, R) \ni f \mapsto (x \mapsto mf(x)))
\]
Check that this is a $R$-module hiomomorphisms, and in fact an isomorphism of $R$-modules. 

\item Let $R = k$ be a field. Then we have a duality 
\[
k-mod\to k-mod
\]
Let $v \fin k-mod$, and consider 
\[
\Phi(V):V\to V^{**}= \Hom_k(\Hom_k(V, k), k)
\]
, and
\[
x\mapsto(\Hom_k(V, K) \ni f \mapsto f(x) \in k)
\]
A duality from $k-mod$ to $k-mod$. 

We may extend $\Phi$ to a functor $k-Mod\to k-Mod$, but this is not surjective if $\dim V = \oo$ (homework problem). So we have a natural equivalence $\Id_{k-mod} \cong (-)^{**}$

\item Here is an examples of an adjoint pair. Let $_SB_R$ be an $S-R$ bimodule. Then the functor
\[
B\otimes_R-:R\to R-mod
\]
is a left adjoint to 
\[
\Hom_S(R, -):S-mod \to R-mod
\]
\end{enumerate}

\section*{Lecture 8, 1/27/23}

\subsection*{\underline{Section 4: Additive and Abelian categories}}

\defn

A pre-additive category $\mc{C}$ is called \underline{an additive category} if it has a zero object, and finite direct sums/products. 

\defn 

An additive category $\mc{C}$ is called \underline{an Abelian category} if every map $f$ has a kernel and cokernal, and every mono is a kernel, and every epi is a cokernel. 

\exm

In the category of rings (we assume these are unital rings, so this category is \underline{not} preadditive, recall) there are categorical epis that fail to be surjective. For example, $f:\Z\hookrightarrow\Q$ is a categorical epimorphism. 

Let $g, h \in \Hom_{\Ring}(\Q, R)$ be such that $gf = hf$. Then $g|_{\Z} = h|_{\Z}$, so it follows that $g = h$.

In $R-mod, R-comp$, categorical monos coincide with injective homomorphisms, and similarly, categorical epimorphisms coincide with surjections. 

\exm of Abelian categories: 

$R-Mod$, in particular $\Z-Mod = \Ab$, $R-comp$. Let $\ms{T}-\Ab$ be the full subcategory of $\Ab$ consisting of the torsion Abelian groups. 

Is the full subcategory of $\Ab$ consisting of torsion-free groups Abelian? No! The map $f:\Z\to\Z$ given by multiplication by 2 doesn't have a cokernel. 

$R-mod$ is not Abelian if $R$ is not left Noetherian!!!! (A ring is left Noetherian if every left ideal is finitely generated). But $R-Mod$

For example, let $k$ be a field, and consider $R = k^\N$. Let $I = k^{(\N)}$ (which means the direct sum, as opposed to the direct product). This is not a finitely generated left ideal, and $I \hookrightarrow R$. But $\pi:R\to R/I$ does \underline{not} have a kernel in $R$-mod even though $R, R/I$ are in $R-mod$, because we will get something not in $R-mod$, but in $R-Mod$. 

She started talking about some stuff we won't see until later, and said it was ``music of the future." 

\subsection*{\underline{Chapter 2: On the road to derived functors}}

\subsubsection*{\underline{Section 1: Exactness properties of functors}}

\underline{Note:} We'll develop the theory for the Abelian category $R-mod$, but it easily adapts to arbitrary categories. 

\defn

Let $R, S$ be rings, $F$ an additive functor from $R-mod$ to $S-mod$. 

\begin{enumerate}

\item $F$ is called \underline{exact} if for all exact sequences
\[
\exactshort{A}{f}{B}{g}{C}
\]
the sequence
\[
\exactshort{FA}{Ff}{FB}{Fg}{FC}
\]
is also exact. If $F$ is contravariant, then instead we want the sequence
\[
\exactshort{FC}{Fg}{FB}{Ff}{FA}
\]
to be exact. 

\item $F$ is called \underline{left-exact} or \underline{right-exact} if it sends left (or right) exact sequences to left (or right) exact sequences


\end{enumerate}

\section*{Lecture 9, 2/1/23}

Recall: 

A functor $F:R-mod\to S-mod$ is left-exact if for all exact sequences
\[
\begin{tikzcd}
0 \ar[r] & A\ar[r, "f"] & B \ar[r, "g"] & C
\end{tikzcd}
\]
the sequence
\[
\begin{tikzcd}
0 \ar[r] & FA\ar[r, "Ff"] & FB \ar[r, "Fg"] & FC
\end{tikzcd}
\]
is exact. Similarly, it is right exact if the image of

\[
\begin{tikzcd}
A\ar[r, "f"] & B \ar[r, "g"] & C\ar[r]&0
\end{tikzcd}
\]
is exact. 

\rem 

\begin{enumerate}

\item A functor $F:R-Mod\to S-Mod$ induces a functor $F:R-comp\to S-comp$. 

\item If $F \cong G:R-mod\to S-mod$ (meaning $F, G$ are naturally isomorphic), then $F, G$ have the same exactness properties. 

\item If $F:R-mod\to S-mod$ is an equivalence (or a duality) then $F$ is exact. 

\end{enumerate}

\thm (our favorite functors)

\begin{enumerate}

\item Let $M\fin R-Mod$. Then $\Hom_R(M, -)$ and $\Hom_R(-, M)$ are left-exact functors from $R-Mod\to\Z-Mod$. 

\item Let $M\fin Mod-R$. Then $M\otimes_R-:R-mod\to\Z-mod$ is right exact.

\end{enumerate}

\proof

Note: Starting from now, I will denote the image of $f$ under the functor $\Hom_R(M, -)$ by $f*$. The reason is that for some reason tikzcd won't let me but $\Hom_R$ inside an arrow's name. 

We'll just prove part 1 for the covariant $\Hom$. The contravariant case is homework. 

So, we want to show that $\Hom_R(M, -)$ is left-exact. Let
\[
\begin{tikzcd}
0 \ar[r] & A\ar[r, "f"] & B \ar[r, "g"] & C
\end{tikzcd}
\]
be exact in $R-Mod$. Then its image is a complex
\[
\begin{tikzcd}
0 \ar[r] & \Hom_R(M,A)\ar[r, "f*"] & \Hom_R(M,B) \ar[r, "g*"] & \Hom_R(M, C)
\end{tikzcd}
\]
For $\phi:M\to A$, $f*(\phi) = f\circ \phi$, and similarly for $g*$. 

We first show that $f*$ is a mono: Indeed, from $f \circ \phi = 0$, we obtain $\phi = 0$, since $f$ is a mono. So the sequence is exact at $\Hom_R(M, A)$. 

We know $\Im(\Hom_R(M, f)) \subseteq \ker(\Hom_R(M, g))$. To show the reverse direction, let $\psi\in\ker(\Hom_R(M, g))$, i.e. $g \circ \psi = 0$, i.e. $\Im(\psi)\subseteq\ker(\psi)$. 

Consider 
\[
\begin{tikzcd}
0\ar[r]&A\ar[r, "\tilde{f}"] & \Im(f) \ar[r, hook] & B \ar[r] & C \ar[r] & 0 
\end{tikzcd} 
\]
Set $\phi = \tilde{f}^{-1}\circ\psi$ and check that $\Hom_R(M, f)(\phi) = \underbrace{f \circ \tilde{f}^{-1}}_{=\Id_A}\circ\psi = \psi$. So $\psi\in\Im(\Hom(M, f))$. This gives exactness at $\Hom_R(M, B)$. So, we have shown that the covariant $\Hom$ functor is left-exact. 

Now for part 2. 

\dbend\dbend\dbend

Let 
\[
\begin{tikzcd}
A\ar[r, "f"] & B \ar[r, "g"] & C\ar[r] & 0 
\end{tikzcd}
\]
be exact in $R-Mod$. The sequence
\[
\begin{tikzcd}
M\otimes_RA\ar[r, "f*"] & M\otimes_RB \ar[r, "g*"] & M\otimes_RC\ar[r] & 0 
\end{tikzcd}
\]
is a complex. Clearly, $M\otimes_Rg, m\otimes b\mapsto m\otimes_R(f(g))$ is an epi because $g$ is an epi. 

We have $\underbrace{\Im(M\otimes_Rf)}_{E} \subseteq \ker(M\otimes_Rg)$ (Birge - ``The image of $M\otimes_Rf$, which I baptize $E$, ... "), and we want the reverse inclusion. Factor $M\otimes_R g$ in the form $M\otimes B \to M\otimes_R B/E \to M\otimes_R C$. (I missed a bit here because TeXwriter was being wonky, so this proof is nonsense I think. It's a standard proof that can be googled tho). Then $M \otimes_Rg = G \circ ?$, so $\ker(G) = \ker(M\otimes g)/E$. Thus suffices to show that $G$ is a mono. 

Plan: Construct $H\in\Hom_{\Z}(M\otimes_RC, M\otimes_RB/E)$ such that $HG = \Id_{M\otimes_RB/E}$

Define $H':M\times C\to M\otimes_RB/E$ by $(m, c)\mapsto(m\otimes b + E)$, where $b \in B$ is such that $g(b) = c$. We check well-definedness. 

Suppose $g(b) = g(b'), b, b' \in B$. Then $b - b' \in \ker(g) = \Im(f)$ by hypothesis, hence $m\otimes_Rb - m\otimes_Rb' = m\otimes(b - b') \in E$, thus $m\otimes_Rb + E = m\otimes_Rb' + E$. Check $H'$ is $\Z$-bilinear and $R$-balanced. 

Hence the universal property of the tensor product yields $H \in \Hom_{\Z}(M\otimes C', M\otimes B/E)$ with $H(m\otimes c) = m\otimes b$, where $g(b) = c$. 

\qed

\exm

Here are some examples showing that in general, we shouldn't expect better than this previous theorem. That is, some witnesses to the non-left(right) exactness of $\Hom_R(M, -)$(resp. $M\otimes_R-$)

\defn

\begin{enumerate}[label=(\roman*)]

\item Let $A\fin\Z-Mod$. Then $a \in A$ is a \underline{torsion element} iff there exists $n\in\Z\setminus\{0\}$ such that $n\cdot a = 0$. We use $T(A)$ to denote the torsion elements of $A$, which will always be a subgroup. 

We say that $A$ is torsion iff $A = T(A)$, and we say $A$ is \underline{torsion-free} iff $T(A) = 0$. 

\item $a \in A$ is \underline{divisible} iff $a \in nA$ for all $n \in \Z\setminus\{0\}$ (i.e. there exists $b \in A$ such that $a = nb$). An abelian group is called a \underline{divisible group} if every element is divisible, i.e. if $nA = A$ for every $n\in\Z\setminus\{0\}$. 

\end{enumerate}

\rem 

If $f\in\Hom_\Z(A, B)$ and $A$ is divisible, then $f(A)$ is a divisible subbgroup of $B$. 

\section*{Lecture 10, 2/3/23}

Consider the sequence
\[
\exactshort{\Z}{\cdot 2}{\Z}{\pi}{\Z/2\Z}
\]
This sequence is exact. However, $\Hom(\Z/2\Z, \Z) = 0$, and $\Hom(\Z/2\Z,\Z/2\Z)\neq 0$. So the image of this sequence under the functor $\Hom(\Z/2\Z, -)$ is not exact (in particular, it is not exact on the right). 

The above argument will work for any integer instead of $2$, so this sequence is a witness to the inexactness of the functor $\Hom_{\Z}(\Z/n\Z, -)$ for $n \geq 2$. 

Consider an epi $g:\Z^{(\N)}\to\Q$, and apply $F$. The map
\[
\begin{tikzcd}
\Hom_{\Z}(\Q,\Z^{(\N)}) \ar[r, "F(g)"]&\Hom_{\Z}(\Q,\Q)
\end{tikzcd}
\]
cannot be an epi, as $\Hom_{\Z}(\Q,\Z^{(\N)}) = 0$ (as $\Q$ is divisible while $\Z^{(\N)}$ is reduced), but $\Hom_{\Z}(\Q,\Q)\neq0$. 

\exm

Here is an example to show that the tensor functor is not left-exact. Let $R = \Z$. We consider the functor $F(-) = \Z/n\Z\otimes_{\Z}-$, $n \geq 2$. Consider the inclusion $\begin{tikzcd} \Z\ar[r, hook, "\iota"]&\Q \end{tikzcd}$. This is a mono. However, $\Z/n\Z\otimes_{\Z}\Q=0$, so $F(\iota):\Z/n\Z\otimes_{\Z}\Z\to0$. Hoever, $\Z/n\Z\otimes_{\Z}\Z = \Z/n\Z \neq 0$, so $F(\iota)$ is not a mono. 

\subsection*{\underline{Short-term program}}

\begin{enumerate}

\item Find exact functors 

\item Characterize the exact sequences $\A$ in $R-comp$ such that $F(\A)$ is exact in $S-comp$ for any additive functor $F:R-mod\to S-mod$. 

\end{enumerate}

\thm

If $F:R-mod\to S-mod$ is an exact functor (i.e. $F$ takes short exact sequences to short exact sequences) then $F(\A)$ is exact in $S-mod$ whenever
\[\A:
\begin{tikzcd}
\cdot\ar[r]&A_{n + 1}\ar[r, "f_{n + 1}"]& A_n\ar[r, "f"]& A_{n - 1} \ar[r, "f_{n - 1}"]&\cdots \\ 
\end{tikzcd}
\]
is exact in $R-mod$. 

\proof

Suppose $F:R-mod\to S-mod$ is exact, and $\A$ as in the claim is an exact sequence in $R-mod$. We can factor each $f_n$ 
\[
\begin{tikzcd}
A_n\ar[r, "\tilde{f_n}"]&\Im(A)\ar[r, hook, "\iota_n"]& A_{n - 1} \\
x\ar[r]&f_n(x_n)\ar[r]&f_n(x_n) \\
\end{tikzcd}
\]
Consider the short exact sequence 
\[
\exactshort{\Im(f_{n + 1})}{\iota_{n + 1}}{A_n}{\tilde{f_n}}{\Im(f_n)}
\]
Since $F$ is exact, we obtain an exact sequence
\[
\exactshort{F(\Im(f_{n + 1}))}{F(\iota_{n + 1})}{F(A_n)}{F(\tilde{f_n})}{F(\Im(\tilde{f_n}))}
\]
In particular $\Im(F(\iota_{n + 1})) = \ker(F(\tilde{f_n}))$ for $n \in \N$. 

\claim
\,

\begin{enumerate}[label=(\alph*)]

\item $\ker(F(f_n)) = \ker(F(\tilde{f_n}))$
\item $\ker(F(\tilde{f_n})) = \Im(F(\iota_{n + 1})) = \Im(F(f_{n + 1}))$

\end{enumerate}

\proof

\begin{enumerate}[label=(\alph*)]

\item $f_n = \iota_n\circ\tilde{f_n}$, so $F(f_n) = F(\iota_n) \circ F(\tilde{f_n})$ by functoriality. Because $F$ is exact, $F(\iota_n)$ is a mono, so $\ker(F(f_n)) = \ker(F(\tilde{f_n}))$. 

\item $F(f_{n + 1}) = F(\iota_{n + 1}) \circ F(\tilde{f_{n + 1}})$. Because $F$ is exact, $F(\tilde{f_{n + 1}})$ is an epi.

\end{enumerate}

So $\im(F(f_{n + 1})) = \Im(F(\iota_{n + 1}))$, so we have shown $F(\A)$ is exact. 

\qed

First installment of finding exact functors. 

\prop

Let $I$ be a set (index set). Consider the functors:
\begin{align*}
(R-Mod)^I &\to R-Mod \\
(m_i)_{i\in I} & \mapsto \prod_{i\in I}m_i \\
(f_i)_{i\in I} & \mapsto \prod_{i\in I}f_i, (\vec{m})_j = f_j(m_j) \\
\end{align*}
$\otimes_{i\in I}$ works the same as before. 

Both of them are exact .

\proof

Obvious

\qed

\subsection*{\underline{First installment re (2)}}

\rem 

Warning: Birge uses nonstandard notation. She says ``$X$ is a direct summand of $Y$" to mean $X\subseteq^{\bigoplus} Y$

\defn 

Let $A, B \fin R-mod, f \in \Hom_R(A, B)$. $f$ is called \underline{split} if $\ker f$ is a direct summand of $A$ and $\Im(f)$ is a direct summand of $B$, i.e. there exist $A', B' \fin R-Mod$ with $A = \ker(f)^{\oplus} A'$ and $B = \Im(f)^\oplus B'$.

\section*{Lecture 11, 2/6/23}

\underline{Convention:} Let $M, N \fin R-Mod$. We say that $N$ is a direct summand of $M$, written $N\subseteq^\oplus M$, if there exists $U \fin R-Mod$ such that $M = N^{\oplus} U$.

\defn

Let $\A\fin R-comp$. Then $\A$ is split if $f_n$ is split for all $n$.

\prop

Let $A, B, C$ be left $R$-modules, $f \in\Hom_R(A, B)$ and $h\in\Hom_R(B, C)$. Then the following conditions are equivalent:

\begin{enumerate}

\item The sequence $\exactshort{A}{f}{B}{g}{C}$ is split exact. 

\item There exists $f'\in\Hom_R(B, A)$ and $g'\in\Hom_R(C, B)$ such that $ff' + g'g = \Id_B$, and both triangles in the diagram below commute:

\[
\begin{tikzcd}
A\ar[r, "f"]\ar[d, equal]&\ar[dl, dotted, "f'"]B\ar[r, "g"]& C\ar[d, equal] \\
A & & C\ar[lu, dotted, "g'"] \\ 
\end{tikzcd}
\]

\end{enumerate}

\proof

\qed

\thm

Let $\A:\begin{tikzcd} \cdots\ar[r]& A_{n + 1}\ar[r, "f_{n + 1}"]& A_n\ar[r, "f_n"] & A_{n - 1}\ar[r] & \cdots \\ \end{tikzcd}\fin R-comp$. Then the following are equivalent:

\begin{enumerate}

\item For all additive functors $F:R-Mod\to S-Mod$ ($S$ any ring), the sequence $F(\A)$ is split exact. 

\item $\A$ is split exact.

\end{enumerate}

\proof

For $1\implies2$, apply $F = \Id_{R-Mod}$. 

For $2\implies1$, suppose 2. We will show 1 only in the case $\A = \exactshort{A}{f}{B}{g}{C}$ is split exact. Move to general $\A$ as in the proof of previous theorem. 

By the proposition, there exist maps $f'\in\Hom_R(B, A)$, $g'\in\Hom_R(C, B)$ with $\Id_B = ff' + g'g$. Let $F$ be an additive functor as in 1. Then
\begin{align*}
\Id_{F(B)} & = F(\Id_B) \\
			  & = F(f)F(f') + F(g')F(g) \\
\end{align*}
So by the proposition, the sequence
\[
\exactshort{F(A)}{F(f)}{F(B)}{F(g)}{F(c)}
\]
is split exact. 

\qed

\exm

There exist exact sequences 
\[
\exactshort{A}{f}{B}{g}{C}
\]
such that $B \cong A\oplus C$, but the sequence fails to be split.

Take $R = \Z$, let $n \geq 2, A = n\Z, B = \Z\oplus(\Z/n\Z)^{(\N)}$, and $C = (\Z/n\Z)^{(\N)}$. Then $A \oplus C \cong B$. 

Find $f \in \Hom_{\Z}(A, B), g \in \Hom_{\Z}(B, C)$, such that 
\[
\exactshort{A}{f}{B}{g}{C}
\]
is exact, but not split. 

\subsection*{\underline{Section 2: Projective Modules}}

Our aim is to characterize those $M\fin R-Mod$ for which $\Hom_R(M, -):R-Mod\to\Z-Mod$ is exact. 

\defn\,

\begin{enumerate}

\item $_RF$ is \underline{free} if $_RF \cong _RR^{(I)}$. 

It is known that $_RF$ is \underline{free} iff $F$ has an $R$-basis, that is, a linearly independent generating set. 

\item $_RP \fin R-Mod$ is called \underline{projective} if $P$ is isomorphic to a direct summand of a free module. 

\end{enumerate}

\exm\,

\begin{enumerate}

\item Vector spaces

\item Let $k$ be a field and $R = k\oplus k$ (ring product). Then $R$ admits projective modules that fail to be free. Let $e_1 = (1, 0), e_2 = (0, 1)$. Then $Re_1 = k\times\{0\}, Re_2 = \{0\}\times k$. We have $R  = Re_1\oplus Re_2$, so the $Re_i$ are projective. However, they are not free, because $\dim_k(R) = 2, \dim_k(Re_i) = 1$

\end{enumerate}

\rem\,

\begin{enumerate}

\item Let $(R_i)_{i\in I}$ be a family of left $R$-modules. Then $\bigoplus_{i\in I}P_i$ is projective if and only if $P_i$ is projective for all $i\in I$. 
\dbend

\item $\prod_{i\in I}P_i$ need not be projective for infinite $I$.

\item Let $R$ be a PID. The projective $R$-modules are precisely the free ones. Such $R$ include $R = \Z, R = k[x], k$ a field.  

\item $\Z^\N$ is \underline{NOT} free (proof to come), and hence not projective. 

\item (Serre's Conjecture) If $R = k[x_1, \dots, x_n]$, $k$ a field, is every projective $R$-module free? It turns out yes, and there are independent proofs by Suslin and \underline{Quillen} (Quillen got the fields medal).

\end{enumerate}

\section*{Lecture 12, 2/8/23}

\underline{True or false:} If $P\fin R-Mod$ is finitely generated projective, then there exists $n \in \N$ such that $P$ is isomorphic to a dirrect summand $_RR^n$? 

This is true. It is isomorphic to a direct summand of a free module $R^{(I)}$. Pick finite $I'\subseteq I$ with $x_k\in\oplus_{i\in I'}R_i$. Then $P \subseteq \oplus_{i\in I'}R \cong R^n$, with $n = |I'|$ and hence $P$ is isomorphic to a direct summand of $R^n$. 

Let $R = K[x, y]$. Then $P = (x)$ is not projective, despite being a submodule of $R$. 

In general: If $_RV\subseteq_RU\subseteq_RM$, and $V\subseteq^\oplus M$, then $V\subseteq^\oplus U$. Look up the ``modular law."

\thm

For $M\fin R-Mod$, the following are equivalent. 

\begin{enumerate}

\item $M$ is projective

\item $\Hom_R(M, -):R-mod\to\Z-mod$ is exact.

\item Whenever $f\in\Hom_R(B, C)$ is an epi and $g\in\Hom_R(M, C)$, then there exists a $\phi\in\Hom_R(M, B)$ with $f\circ\phi=g$. That is, there is a $\phi$ making the following diagram commutes. 
\[
\begin{tikzcd}
B\ar[r, twoheadrightarrow, "f"] & C \ar[r]&0 \\
& M \ar[ul, dotted, "\phi"'] \ar[u, "g"'] \\
\end{tikzcd}
\]

\item Every epi onto $M$ splits. That is, every surjection onto $M$ admits a section.

\end{enumerate}

For the following proof, we will abbreviate $\Hom_R(M, -)$ by $[M, -]$. 

\proof\,

\subsection{$(1)\implies(3)$}

We have $M \subseteq^\oplus _RF$, $F$ free on basis $(x_i)_{i\in I}$, $F = M\oplus N$. 

Let $f$ and $g$ be as in 3. That is, we have
\[
\begin{tikzcd}
B\ar[r, twoheadrightarrow, "f"] & C\\
& M\ar[u, "g"'] \\
& F\ar[u, "\pi"']\\
\end{tikzcd}
\]
where $\pi:F\to M$ is the projection along $N$, and $\iota:M\hookrightarrow F$ the embedding. There is a $\psi:F\to B$ which makes this commute: \[
\begin{tikzcd}
B\ar[r, twoheadrightarrow, "f"] & C\\
& M\ar[u, "g"'] \\
& F\ar[u, "\pi"']\ar[uul, dotted, "\psi"]\\
\end{tikzcd}
\]
defined by $\psi(x_i) = b_i$ if $f(b_i) = g\pi(x_i)$. This is well-defined because $F$ is free on the $x_i$, and the $b_i$ exist because $f$ is an epi Then, we have $f\circ\psi = g\circ\pi$, so $f\circ\psi\circ\iota = g\circ\pi\circ\iota$. But $\pi\circ\iota=\Id_M$, so $f\circ(\psi\circ\iota) = g$. But that means $\phi = \psi\circ\iota:M\to B$ makes the diagram commute: \[
\begin{tikzcd}
B\ar[r, twoheadrightarrow, "f"] & C\\
& M\ar[u, "g"'] \ar[ul, dotted, "\phi"']\\
& F\ar[u, "\pi"']\ar[uul, dotted, "\psi"]\\
\end{tikzcd}
\]

\subsection*{$(3)\implies(2)$}

Assume 3. Since $[M,-]$ is a left-exact functor, it suffices to show that $[M,-]$ takes epis to epis. 

Let $\begin{tikzcd} B\ar[r, "f"] & C\ar[r]&0\end{tikzcd}$ be exact. To show $[M, f]:[M, B]\to[M, C]$ is an epi, let $g \in [M, C]$. By assumption, there exists $\phi\in[M, B]$ with $f^*(\phi)=g$. But this means $\phi^*(f) = g$. 

\subsection*{$(2)\implies(4)$}

Assume 2. Let $f:N\twoheadrightarrow M$ be an epi. Then by assumption, $[M, f]:[M, B]\onto[M,C]$ is an epi. So there exists $\phi\in[M,N]$ with $f\circ\phi=\Id$. So the following diagram commutes:
\[
\begin{tikzcd}
M\ar[r, "\phi"]\ar[d, equals] & N \ar[dl, twoheadrightarrow, "f"'] \\
M & \\
\end{tikzcd}
\]
Hence $N = \Im(\phi)\oplus\ker(f)$, so $\ker(f)\subseteq^\oplus N$, and $f$ splits. 

\subsection*{$(4)\implies(1)$}

Assume 4. If $(m_i)_{i\in I}$ is a generating set for $M$, then $F = R^{(I)}\onto M$, given by $(r_i)\mapsto\sum_{i\in I}r_im_i$, (which will be a finite sum because all but finitely many $r_i$ are zero) is an epi. 

By assumption, $f$ splits, meaning $F = \ker(f)\oplus N$. Then $N \cong F/\ker(f)\cong M$. So $M$ is isomorphic to a direct summand of $F$. 

This completes the proof.

\qed

\subsection*{\underline{Further examples of projective modules and structure results}}

\exm\,

\begin{enumerate}

\item Let $R$ be a ring. All $M \fin R-Mod$ are projective if and only every submodule $U$ of any left $R$-module $N$ is a direct summand, i.e. $N = U \oplus V$. 

($=>$)
Let $_RU\subseteq _RN$. Take $M = N/U$ and epi $\pi:N\to N/U$. Since $N/U$ is projective, $\pi$ splits, meaning $U \subseteq^{\oplus}A$. 

($<=$)
Easy

Equivalently, if $R$ is semi-simple when viewed as a right module over itself, $R_R$.

\end{enumerate}

\section*{Lecture 13, 2/10/23}

\prop (Baer) $\Z^\N$ is not free. 

\proof (not by Baer) 

Assume $\Z^\N$ is free, i.e. there exists an isomorphism $\Z^\N\cong\Z^{(I)}$ for some $I$. Then $|I|>\aleph_0$. Pick a countable subset $I'\subset I$ such that $f(\Z^{(\N)}) \subseteq \Z^{(I')}$ and consider the map $\bar{f}$ induced by $f$, 
\[
\bar{f}:\Z^\N/\Z^{(\N)} \to \Z^{(I)}/\Z^{(I')}\cong\Z^{(I\setminus I')}
\]
\[
x + \Z^{(\N)} \mapsto f(x) + \Z^{(I')}
\]
In particular, $\Z^{(I\setminus I')}$ is nonzero and free, because $I$ is uncountable and $I'$ is countable. 

Now, if $F$ is a free abelian group, $F$ contains no nonzero elements which are divisible by $3^n$ for all $n\in\N$. 

Let $S = \{(z_n)_{n\in\N}\in\Z^\N \mid 3^n \mid z_n\text{ for }n\in\N\}$. Note $S$ is uncountable. 

Since $\Z^{(I')}$ is countable, there exists $s\in S$ with $f(s) \in \Z^{(I')}$ and thus $f(s) + \Z^{(I')}\neq0$ in $\Z^{(I\setminus I')}$.

Hence $\barf(s + \Z^{(\N)}) \neq 0\in\underbrace{\Z^{(I\setminus I')}}_{\text{free}}$.

Thus $\Z^{(I\setminus I')}$ contains nonzero elements divisible by $3^n$ for all $n$, a contradiction. 

Recall that projective means isomorphic to a summand of a free module $R^{(I)}$ for some $I$. 

\dbend\dbend\dbend


In general, projectives in $R$-Mod are \underline{not} direct sums of finitely generated modules. 

\thm (Kaplensky)

Let $P\fin R$-Mod be projective. Then $P$ is the sum of countably generated modules. 

\proof

Not enough time. 

\qed

\cor

If $R$ is local, then all projective $R$-modules are free.

\proof

\qed

\subsection*{\underline{Section 3:Projective resolutions and projective dimension}}

\underline{Idea:} 

Let $M\fin R$-mod. Approximate $f_0:P_0\twoheadrightarrow M\to 0$

\underline{Error:} $\ker(f_0)$. Next approximate $f_1:P_1\to\ker(f_0)\to0$, with $P_1$ projective. This results in a sequence

\[
\begin{tikzcd}
\cdots\ar[r]&P_2\ar[dr, twoheadrightarrow] \ar[rr, "f_2"]&&P_1\ar[rr, "f_1"] \ar[dr, twoheadrightarrow]&&P_0\ar[r, twoheadrightarrow, "f_0"]&M\ar[r]&0 \\ &
&\ker(f_1)\ar[ur]& &\ker(f_0)\ar[ur]& &\\
\end{tikzcd}
\]

\defn

Let $M\fin R$-Mod. A \underline{projective resolution of $M$} is an exact sequence
\[
\begin{tikzcd}
\cdots\ar[r]&P_2\ar[r, "f_2"]&P_1\ar[r, "f_1"]&P_0\ar[r, "f_0"]&M\ar[r]&0 \\
\end{tikzcd}
\]
where all $P_i$ are exact. 

Call such a projective resolution \underline{finite of length $n$} if $P_n\neq0$, but $P_m =0$ for $m > n$. 

\defn 

We define the \underline{projective dimension} of a module $M$ as follows. If there is no finite projective resolution, then it is $\oo$. Otherwise it is the smallest $n$ suuch there exists a projective resolution of length $n$. By convention, the projective dimension of the zero module is $-\oo$. 

We denote this by $\pdim M$. 

The \underline{(left) global dimension} of a ring $R$ is the defined as $\gldim R = \sup\{\pdim M \mid M\fin R-Mod\}$. There is also of course the right global dimension, where instead we consider $Mod-R$. 

\exm\,

\begin{itemize}

\item $\gldim\Z = 1$.

\item $\lgldim R = 0$ if and only if $R$ is semisimple, which happens if and only if $\rgldim R = 0$. 

\item 
\[
\begin{pmatrix}
K&\cdots&K \\
\vdots & \vdots \\
0 & \cdots & K \\
\end{pmatrix}  \not\supseteq \begin{pmatrix} 0 & K & \cdots K \\ 
\vdots & \ddots & \ddots & \vdots \\
0 & \cdots & \cdots & 0 \\
\end{pmatrix}
\]
not semisimple, so $\lgldim R \geq 1$. In fact, $=$ holds (argument later). 

\item This is Hilbert's Syzygy theorem. It says that if $K$ is a field, then 
\[
\gldim K[x_1, \dots, x_n] = n
\]
 

\item $R = \Z/p^n\Z$, $p$ prime, $n\geq2$. 
\[
\begin{tikzcd}
\Z/p^n\Z\ar[rr]\ar[dr]&& \underbrace{\Z/p^n\Z}_{P_0=_RR}\ar[r, twoheadrightarrow, "f_0"]&\underbrace{\Z/p^{n - 1}\Z}_{_RM}\ar[r]&0 \\
&p\Z/p^n\Z \ar[ur, hook] & \\
\end{tikzcd}
\]


\end{itemize}
















\end{document}