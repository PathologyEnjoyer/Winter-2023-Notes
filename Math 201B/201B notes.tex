\documentclass[x11names,reqno,14pt]{extarticle}
% Choomno Moos
% Portland State University
% Choom@pdx.edu


%% stupid experiment %%
%%%%%%%%%%%%% PACKAGES %%%%%%%%%%%%%

%%%% SYMBOLS AND MATH %%%%
\let\oldvec\vec
\usepackage{authblk}	% author block customization
\usepackage{microtype}	% makes stuff look real nice
\usepackage{amssymb} 	% math symbols
\usepackage{siunitx} 	% for SI units, and the degree symbol
\usepackage{mathrsfs}	% provides script fonts like mathscr
\usepackage{mathtools}	% extension to amsmath, also loads amsmath
\usepackage{esint}		% extended set of integrals
\mathtoolsset{showonlyrefs} % equation numbers only shown when referenced
\usepackage{amsthm}		% theorem environments
\usepackage{relsize}	%font size commands
\usepackage{bm}			% provides bold math
\usepackage{bbm}		% for blackboard bold 1

%%%% FIGURES %%%%
\usepackage{graphicx} % for including pictures
\usepackage{float} % allows [H] option on figures, so that they appear where they are typed in code
\usepackage{caption}
\usepackage{hyperref}
%\usepackage{titling}
\usepackage{tikz} % for drawing
\usetikzlibrary{shapes,arrows,chains,positioning,cd,decorations.pathreplacing,decorations.markings,hobby,knots,braids}
\usepackage{subcaption}	% subfigure environment in figures

%%%% MISC %%%%
\usepackage{enumitem} % for lists and itemizations
\setlist[enumerate]{leftmargin=*,label=\bf \arabic*.}

\usepackage{multicol}
\usepackage{multirow}
\usepackage{url}
\usepackage[symbol]{footmisc}
\renewcommand{\thefootnote}{\fnsymbol{footnote}}
\usepackage{lastpage} % provides the total number of pages for the "X of LastPage" page numbering
\usepackage{fancyhdr}
\usepackage{manfnt}
\usepackage{nicefrac}
%\usepackage{fontspec}
%\usepackage{polyglossia}
%\setmainlanguage{english}
%\setotherlanguages{khmer}
%\newfontfamily\khmerfont[Script=Khmer]{Khmer Busra}

%%% Khmer script commands for math %%%
%\newcommand{\ka}{\text{\textkhmer{ក}}}
%\newcommand{\ko}{\text{\textkhmer{ត}}}
%\newcommand{\kha}{\text{\textkhmer{ខ}}}

%\usepackage[
%backend=biber,
% numeric
%style=numeric,
% APA
%bibstyle=apa,
%citestyle=authoryear,
%]{biblatex}

\usepackage[explicit]{titlesec}
%%%%%%%% SOME CODE FOR REDECLARING %%%%%%%%%%

\makeatletter
\newcommand\RedeclareMathOperator{%
	\@ifstar{\def\rmo@s{m}\rmo@redeclare}{\def\rmo@s{o}\rmo@redeclare}%
}
% this is taken from \renew@command
\newcommand\rmo@redeclare[2]{%
	\begingroup \escapechar\m@ne\xdef\@gtempa{{\string#1}}\endgroup
	\expandafter\@ifundefined\@gtempa
	{\@latex@error{\noexpand#1undefined}\@ehc}%
	\relax
	\expandafter\rmo@declmathop\rmo@s{#1}{#2}}
% This is just \@declmathop without \@ifdefinable
\newcommand\rmo@declmathop[3]{%
	\DeclareRobustCommand{#2}{\qopname\newmcodes@#1{#3}}%
}
\@onlypreamble\RedeclareMathOperator
\makeatother

\makeatletter
\newcommand*{\relrelbarsep}{.386ex}
\newcommand*{\relrelbar}{%
	\mathrel{%
		\mathpalette\@relrelbar\relrelbarsep
	}%
}
\newcommand*{\@relrelbar}[2]{%
	\raise#2\hbox to 0pt{$\m@th#1\relbar$\hss}%
	\lower#2\hbox{$\m@th#1\relbar$}%
}
\providecommand*{\rightrightarrowsfill@}{%
	\arrowfill@\relrelbar\relrelbar\rightrightarrows
}
\providecommand*{\leftleftarrowsfill@}{%
	\arrowfill@\leftleftarrows\relrelbar\relrelbar
}
\providecommand*{\xrightrightarrows}[2][]{%
	\ext@arrow 0359\rightrightarrowsfill@{#1}{#2}%
}
\providecommand*{\xleftleftarrows}[2][]{%
	\ext@arrow 3095\leftleftarrowsfill@{#1}{#2}%
}
\makeatother

%%%%%%%% NEW COMMANDS %%%%%%%%%%

% settings
\newcommand{\N}{\mathbb{N}}                     	% Natural numbers
\newcommand{\Z}{\mathbb{Z}}                     	% Integers
\newcommand{\Q}{\mathbb{Q}}                     	% Rationals
\newcommand{\R}{\mathbb{R}}                     	% Reals
\newcommand{\C}{\mathbb{C}}                     	% Complex numbers
\newcommand{\K}{\mathbb{K}}							% Scalars
\newcommand{\F}{\mathbb{F}}                     	% Arbitrary Field
\newcommand{\E}{\mathbb{E}}                     	% Euclidean topological space
\renewcommand{\H}{{\mathbb{H}}}                   	% Quaternions / Half space
\newcommand{\RP}{{\mathbb{RP}}}                       % Real projective space
\newcommand{\CP}{{\mathbb{CP}}}                       % Complex projective space
\newcommand{\Mat}{{\mathrm{Mat}}}						% Matrix ring
\newcommand{\M}{\mathcal{M}}
\newcommand{\GL}{{\mathrm{GL}}}
\newcommand{\SL}{{\mathrm{SL}}}

\newcommand{\tgl}{\mathfrak{gl}}
\newcommand{\tsl}{\mathfrak{sl}}                  % Lie algebras; i.e., tangent space of SO/SL/SU
\newcommand{\tso}{\mathfrak{so}}
\newcommand{\tsu}{\mathfrak{sl}}


% typography
\newcommand{\noi}{\noindent}						% Removes indent
\newcommand{\tbf}[1]{\textbf{#1}}					% Boldface
\newcommand{\mc}[1]{\mathcal{#1}}               	% Calligraphic
\newcommand{\ms}[1]{\mathscr{#1}}               	% Script
\newcommand{\mbb}[1]{\mathbb{#1}}               	% Blackboard bold


% (in)equalities
\newcommand{\eqdef}{\overset{\mathrm{def}}{=}}		% Definition equals
\newcommand{\sub}{\subseteq}						% Changes default symbol from proper to improper
\newcommand{\psub}{\subset}						% Preferred proper subset symbol

% Categories
\newcommand{\catname}[1]{{\text{\sffamily {#1}}}}

\newcommand{\Cat}{{\catname{C}}}
\newcommand{\cat}[1]{{\catname{\ifblank{#1}{C}{#1}}}}
\newcommand{\CAT}{{\catname{Cat}}}
\newcommand{\Set}{{\catname{Set}}}

\newcommand{\Top}{{\catname{Top}}}
\newcommand{\Met}{{\catname{Met}}}
\newcommand{\PL}{{\catname{PL}}}
\newcommand{\Man}{{\catname{Man}}}
\newcommand{\Diff}{{\catname{Diff}}}

\newcommand{\Grp}{{\catname{Grp}}}
\newcommand{\Grpd}{{\catname{Grpd}}}
\newcommand{\Ab}{{\catname{Ab}}}
\newcommand{\Ring}{{\catname{Ring}}}
\newcommand{\CRing}{{\catname{CRing}}}
\newcommand{\Mod}{{\mhyphen\catname{Mod}}}
\newcommand{\Alg}{{\mhyphen\catname{Alg}}}
\newcommand{\Field}{{\catname{Field}}}
\newcommand{\Vect}{{\catname{Vect}}}
\newcommand{\Hilb}{{\catname{Hilb}}}
\newcommand{\Ch}{{\catname{Ch}}}

\newcommand{\Hom}{{\mathrm{Hom}}}
\newcommand{\End}{{\mathrm{End}}}
\newcommand{\Aut}{{\mathrm{Aut}}}
\newcommand{\Obj}{{\mathrm{Obj}}}
\newcommand{\op}{{\mathrm{op}}}

% Norms, inner products
\delimitershortfall=-1sp
\newcommand{\widecdot}{\, \cdot \,}
\newcommand\emptyarg{{}\cdot{}}
\DeclarePairedDelimiterX{\norm}[1]{\Vert}{\Vert}{\ifblank{#1}{\emptyarg}{#1}}
\DeclarePairedDelimiterX{\abs}[1]\vert\vert{\ifblank{#1}{\emptyarg}{#1}}
\DeclarePairedDelimiterX\inn[1]\langle\rangle{\ifblank{#1}{\emptyarg,\emptyarg}{#1}}
\DeclarePairedDelimiterX\cur[1]\{\}{\ifblank{#1}{\emptyarg,\emptyarg}{#1}}
\DeclarePairedDelimiterX\pa[1](){\ifblank{#1}{\emptyarg}{#1}}
\DeclarePairedDelimiterX\brak[1][]{\ifblank{#1}{\emptyarg}{#1}}
\DeclarePairedDelimiterX{\an}[1]\langle\rangle{\ifblank{#1}{\emptyarg}{#1}}
\DeclarePairedDelimiterX{\bra}[1]\langle\vert{\ifblank{#1}{\emptyarg}{#1}}
\DeclarePairedDelimiterX{\ket}[1]\vert\rangle{\ifblank{#1}{\emptyarg}{#1}}

% mathmode text operators
\RedeclareMathOperator{\Re}{\operatorname{Re}}		% Real part
\RedeclareMathOperator{\Im}{\operatorname{Im}}		% Imaginary part
\DeclareMathOperator{\Stab}{\mathrm{Stab}}
\DeclareMathOperator{\Orb}{\mathrm{Orb}}
\DeclareMathOperator{\Id}{\mathrm{Id}}
\DeclareMathOperator{\vspan}{\mathrm{span}}			% Vector span
\DeclareMathOperator{\tr}{\mathrm{tr}}
\DeclareMathOperator{\adj}{\mathrm{adj}}
\DeclareMathOperator{\diag}{\mathrm{diag}}
\DeclareMathOperator{\eq}{\mathrm{eq}}
\DeclareMathOperator{\coeq}{\mathrm{coeq}}
\DeclareMathOperator{\coker}{\mathrm{coker}}
\DeclareMathOperator{\dom}{\mathrm{dom}}
\DeclareMathOperator{\cod}{\mathrm{codom}}
\DeclareMathOperator{\im}{\mathrm{im}}
\DeclareMathOperator{\Dim}{\mathrm{dim}}
\DeclareMathOperator{\codim}{\mathrm{codim}}
\DeclareMathOperator{\Sym}{\mathrm{Sym}}
\DeclareMathOperator{\lcm}{\mathrm{lcm}}
\DeclareMathOperator{\Inn}{\mathrm{Inn}}
\DeclareMathOperator{\sgn}{sgn}						% sgn operator
\DeclareMathOperator{\intr}{\text{int}}             % Interior
\DeclareMathOperator{\co}{\mathrm{co}}				% dual/convex Hull
\DeclareMathOperator{\Ann}{\mathrm{Ann}}
\DeclareMathOperator{\Tor}{\mathrm{Tor}}


% misc symbols
\newcommand{\divides}{\big\lvert}
\newcommand{\grad}{\nabla}
\newcommand{\veps}{\varepsilon}						% Preferred epsilon
\newcommand{\vphi}{\varphi}
\newcommand{\del}{\partial}							% Differential/Boundary
\renewcommand{\emptyset}{\text{\O}}					% Traditional emptyset symbol
\newcommand{\tril}{\triangleleft}					% Quandle operation
\newcommand{\nabt}{\widetilde{\nabla}}				% Contravariant derivative
\newcommand{\later}{$\textcolor{red}{\blacksquare}$}% Laziness indicator

% misc
\mathchardef\mhyphen="2D							% mathomode hyphen
\renewcommand{\mod}[1]{\ (\mathrm{mod}\ #1)}
\renewcommand{\bar}[1]{\overline{#1}}				% Closure/conjugate
\renewcommand\qedsymbol{$\blacksquare$} 			% Changes default qed in proof environment
%%%%% raised chi
\DeclareRobustCommand{\rchi}{{\mathpalette\irchi\relax}}
\newcommand{\irchi}[2]{\raisebox{\depth}{$#1\chi$}}
\newcommand\concat{+\kern-1.3ex+\kern0.8ex}

% Arrows
\newcommand{\weak}{\rightharpoonup}					% Weak convergence
\newcommand{\weakstar}{\overset{*}{\rightharpoonup}}% Weak-star convergence
\newcommand{\inclusion}{\hookrightarrow}			% Inclusion/injective map
\renewcommand{\natural}{\twoheadrightarrow}				% Natural map
\newcommand{\oo}{\infty}

% Environments
\theoremstyle{plain}
\newtheorem{thm}{Theorem}[section]
%\newtheorem{lem}[thm]{Lemma}
\newtheorem{lem}{Lemma}
\newtheorem*{lems}{Lemma}
\newtheorem{cor}[thm]{Corollary}
\newtheorem{prop}{Proposition}
\newtheorem*{claim}{Claim}
\newtheorem*{cors}{Corollary}
\newtheorem*{props}{Proposition}
\newtheorem*{conj}{Conjecture}

\theoremstyle{definition}
\newtheorem{defn}{Definition}[section]
\newtheorem*{defns}{Definition}
\newtheorem{exm}{Example}[section]
\newtheorem{exer}{Exercise}[section]

\theoremstyle{remark}
\newtheorem*{rem}{Remark}

\newtheorem*{solnx}{Solution}
\newenvironment{soln}
    {\pushQED{\qed}\renewcommand{\qedsymbol}{$\Diamond$}\solnx}
    {\popQED\endsolnx}%

% Macros

\newcommand{\restr}[1]{_{\mkern 1mu \vrule height 2ex\mkern2mu #1}}
\newcommand{\Upushout}[5]{
    \begin{tikzcd}[ampersand replacement = \&]
    \&#2\ar[rd,"\iota_{#2}"]\ar[rrd,bend left,"f"]\&\&\\
    #1\ar[ur,"#4"]\ar[dr,"#5"]\&\&#2\oplus_{#1} #3\ar[r,dashed,"\vphi"]\&Z\\
    \&#3\ar[ur,"\iota_{#3}"']\ar[rru,bend right,"g"']\&\&
    \end{tikzcd}
}
\newcommand{\exactshort}[5]{
		\begin{tikzcd}[ampersand replacement = \&]
			0\ar[r]\&#1\ar[r,"#2"]\& #3 \ar[r,"#4"]\& #5 \ar[r]\&0
		\end{tikzcd}
}
\newcommand{\product}[6]{
		\begin{tikzcd}[ampersand replacement = \&]
			#1 \& #2 \ar[l,"#4"'] \\
			#3 \ar[u,"#5"] \ar[ur,"#6"']
		\end{tikzcd}
}
\newcommand{\coproduct}[6]{
		\begin{tikzcd}[ampersand replacement = \&]
			#1 \ar[r,"#4"] \ar[d,"#5"'] \& #2 \ar[dl,"#6"] \\
			#3
		\end{tikzcd}
}
%%%%%%%%%%%% PAGE FORMATTING %%%%%%%%%

\usepackage{geometry}
    \geometry{
		left=15mm,
		right=15mm,
		top=15mm,
		bottom=15mm	
		}

\usepackage{color} % to do: change to xcolor
\usepackage{listings}
\lstset{
    basicstyle=\ttfamily,columns=fullflexible,keepspaces=true
}
\usepackage{setspace}
\usepackage{setspace}
\usepackage{mdframed}
\usepackage{booktabs}
\usepackage[document]{ragged2e}
\usepackage{amsmath}
\pagestyle{fancy}{
	\fancyhead[L]{Fall 2022}
	\fancyhead[C]{201A - Real Analysis}
	\fancyhead[R]{John White}
  
  \fancyfoot[R]{\footnotesize Page \thepage \ of \pageref{LastPage}}
	\fancyfoot[C]{}
	}
\fancypagestyle{firststyle}{
     \fancyhead[L]{}
     \fancyhead[R]{}
     \fancyhead[C]{}
     \renewcommand{\headrulewidth}{0pt}
	\fancyfoot[R]{\footnotesize Page \thepage \ of \pageref{LastPage}}
}

\newcommand{\seq}[1]{_{#1 = 1}^\oo}
\newcommand{\barr}{{\bar{\R}}}
\DeclareMathOperator{\Vol}{Vol}
\title{220A - Groups}
\author{John White}
\date{Fall 2022}




\begin{document}

\section*{Lecture 1}

Let $(X, \mc{A}, \mu)$ be a measure space. Without any additional structure or information, we may define the Lebesgue integral $\int_Xf\,d\mu$ for $f$ an $\mc{A}-\mc{B}$ measurable function $f:X\to[-\oo,+\oo]$. 

We only have a few examples without any work. 

\exm

\begin{itemize}

\item For any set $X$, we can define the counting measure on $\mc{A} = 2^X$, which gives $\mu(A) = |A|$. If $X = \N$, then a measurable function is just a sequence $(f_n)$, and $\int_Xf\,d\mu = \sum f_n$

\item We can also define the Dirac mass $\delta_p$ for a fixed $p \in X$ by 
\[
\delta_p(E) = \begin{cases} 1 & p \in E \\ 0 & p \not\in E\\ \end{cases}
\]
We have $\int_Xf\,d\delta_p = f(p)$

\end{itemize}

To get another example of a measure we need to do some work. 

\underline{Problem:} We want a measure $\mu$ on $\R^n$ such that, for a rectangle, 
\[
\mu([a_1,b_1]\times\cdots\times[a_n, b_n]) = |a_1 - b_1|\cdots|a_n - b_n|
\]
Once it is defined on all rectangles, it is defined on the minimal $\sigma$-algebra containing them, which is the Borel $\sigma$-algebra. In other words, this condition will completely specify a measure on the Borel $\sigma$-algebra $\mc{B}_{\R^n}$

If $X = \R^n$, or a general metric space, or even a general topological space, then $\mc{B}(X)$ denotes the $\sigma$-algebra generated by the open subsets of $X$. 

\underline{Problem:} 

Suppose we have a distribution function $F:\R\to\R$, meaning $F$ is monotone, positive, and $\lim_{x\to-\oo}f(x) = 0, \lim_{x\to\oo}f(x) = 1$, and continuous from the right. We want a Borel measure $\mu$ such that $F(t) = \mu((-\oo, t])$. Such a measure, denoted by $\lambda_F$, is called a Lebesgues-Stieltjes measure. 

The corresponding integral is called a Lebesgue-Stieltjes integral.

If $F$ is smooth, then $\int_{\R}\phi\,d\lambda_F = \int_{-\oo}^{\oo}\phi(x)d\,F(x)$.

The measure we want on $\R^n$ is denoted by $\lambda^n$. 

\subsection*{\underline{The Carath\'eodory Construction}}

Suppose we have an outer measure $\gamma:2^X\to[0,\oo]$. This means $\gamma(\varnothing) = 0$, $A \subset B \implies \gamma(A) \leq \gamma(B)$ (monotone), and $\gamma(\cup\seq{i}E_i) \leq \sum\seq{i}\gamma(E_i)$ (subadditive).

We can define a set $S$ to be $\gamma$-measurable if for every testing set $T$, $\gamma(T) = \gamma(S \cap T) + \gamma(S^c \cap T)$.

\thm (Carath\'eodory Extension Theorem)

\begin{enumerate}

\item $\gamma(N) = 0 \implies N$ is measurable. 

\item The set of measurable sets forms a $\sigma$-algebra $\Gamma$. 

\item $\gamma$ restricted to $\Gamma$ forms a measure. 

\end{enumerate}

``Nothing in the above theorem can guarantee you that $\Gamma$ is not trivial, i.e. $\Gamma = \{\varnothing, X\}$. Nevertheless, this is a very useful guy" - Dennis. 

\defn (Lebesgue outer measure on $\R^n$)

Let $R$ be a rectangle in $\R^n$, that is $R = \prod_{i=1}^n[a_i, b_i]$. We have $\Vol(R) = |a_1 - b_1|\cdots|a_n - b_n|$. For any $E \subseteq\R^n$, we define
\[
\mu^*(E) \eqdef \inf\{\sum\seq{j}\Vol(R_j) \mid E \subseteq \cup\seq{j}R_j\}
\] 

\prop

$\mu^*$ is an outer measure on $\R^n$ such that $\mu^*(R) = \Vol(R)$ for all rectangles $R$. 

\proof

The first and second axioms are trivial, so we will just prove the subadditivity. Let $E$ be some set. By definition, for any $\varepsilon$, there is some cover $R_j$ by recrtangles such that 
\[
-\varepsilon + \sum\seq{j}\Vol(R_j) \leq \mu^*(E) \leq \sum\seq{j}\Vol(R_j)
\]
meaning that $\sum\seq{j}\Vol(R_j) \leq \mu^*(E) + \varepsilon$. So for each $E_k$, there is a sequence $R^k_j$ which covers $E_k$, such that $\sum\seq{j}\Vol(R^k_j) \leq \mu^*(E) + \frac{\varepsilon}{2^k}$. 

So $\{R^k_j\}_{j, k \in \N}$ forms a cover of $\cup\seq{j}E_j$. Thus

\begin{align*}
\mu^*(\cup\seq{k}E_k) & \leq \sum\seq{k}\sum\seq{j}\Vol(R_j^k) \\
& \leq \sum\seq{k}\left(\mu^*(E_k) + \frac{\varepsilon}{2^k}\right) \\
& = \sum\seq{k}\mu^*(E_k) + \varepsilon
\end{align*}
This is true for any positive $\varepsilon$. Taking the limit as $\varepsilon\to0$ gives the result. 

\qed

Now, fix a rectangle $R$. Note that $R$ itself forms a cover of $R$, so by the definition, $\mu^*(R) \leq \Vol(R)$. For $\varepsilon > 0$, we can take an almost-optimal cover $(R_j)$ such that $\sum\seq{j}\Vol(R_j) \leq \Vol(R) + \varepsilon$. We can rig it such that $|\Vol(R_j) - \Vol(R)| \leq \frac{\varepsilon}{2^j}$.  Because $R \subset\cup\seq{j}R_j$, and $R_j$ is an open cover, by compactness of $R$ there is a finite subcover, and the volume of $R$ is less than or equal to the sum of the volumes of these finitely many $R_j$. So the volume of $R$ is less than or equal to $\mu^*(R) + 2\varepsilon$. So $\Vol(R) = \mu^*(R)$.

\prop 

Every rectangle $R$ in $\R^n$ is Carath\'eodory measurable). 

\proof

I missed this lol. Apparently Dennis denotes $\mc{M}_{\lambda^*}$ by $\ms{L}^n$.

\qed

\defn

A set is said to be \underline{$G_\delta$} if it is the countable intersection of open sets. A set is said to be \underline{$F_\sigma$} if it is the countable union of closed sets. 

\thm

\begin{enumerate}

\item For all $E \in \ms{L}^n$, $\lambda^N(E) = \inf\{\lambda^n(O) \mid \text{open }O\supseteq E\}$. 

\item $E \in \ms{L}^n$ if and only if $E = H\setminus Z$, where $H$ is $G_\delta$, and $\lambda^*(Z) = 0$. 

\item $E \in \ms{L}^n$ if and only if $E = H \cup Z$, where $H$ is $F_\sigma$ and $\lambda^*(Z) = 0$. 

\item $\lambda^n(E) = \sup\{\lambda^n(C) \mid \text{closed }C \subseteq E\}$

\end{enumerate}

\proof

It suffices to prove the first statement, as the others will follow by passing to a complement. 

\qed

\defn

Suppose $X$ is a metric space. A measure on $X$ is a \underline{Radon measure} if it is Borel (meaning defined on a $\sigma$-algebra containing Borel sets), and for any Borel $E$, $\mu(E) = \inf\{\mu(O) \mid \text{open }O \supseteq E\}$, and for any compact $C\subseteq X$, $\mu(C)<\oo$. 

\thm (Riesz)

Let $X \subseteq \R^n$ be compact. Let $C(X)$ denote the vector space of all continuous functions on $X$. This admits a norm $\norm{f}_{C(X)} = \sup_X|f|$, making it a Banach space. Define $C^*(X) = \{\phi:C(X)\to\R, \phi$ is linear and continuous $\}$. 

For all $\phi \in C^*(X)$, there exists a Radon measure $\mu = \mu_{\phi}$, and a function $M:X\to\{\pm1\}$ which is Borel, such that
\[
\phi(f) = \int_Xf(x)M(x)\,d\mu_{\phi}(x)
\]

for all $f \in C(X)$. 












\end{document}