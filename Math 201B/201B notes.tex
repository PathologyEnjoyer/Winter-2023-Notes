\documentclass[x11names,reqno,14pt]{extarticle}
\input{preamble}
\usepackage[document]{ragged2e}
\usepackage{amsmath}
\pagestyle{fancy}{
	\fancyhead[L]{Spring 2023}
	\fancyhead[C]{201B - Real Analysis}
	\fancyhead[R]{John White}
  
  \fancyfoot[R]{\footnotesize Page \thepage \ of \pageref{LastPage}}
	\fancyfoot[C]{}
	}
\fancypagestyle{firststyle}{
     \fancyhead[L]{}
     \fancyhead[R]{}
     \fancyhead[C]{}
     \renewcommand{\headrulewidth}{0pt}
	\fancyfoot[R]{\footnotesize Page \thepage \ of \pageref{LastPage}}
}

\newcommand{\seq}[2][\oo]{_{#2 = 1}^#1}
\newcommand{\barr}{{\bar{\R}}}
\newcommand{\cupk}[1][\oo]{\cup\seq[#1]{k}}
\newcommand{\cupi}[1][\oo]{\cup\seq[#1]{i}}
\newcommand{\cupn}[1][\oo]{\cup\seq[#1]{n}}
\newcommand{\bigcupk}[1][\oo]{\bigcup\seq[#1]{k}}
\newcommand{\bigcupi}[1][\oo]{\bigcup\seq[#1]{i}}
\newcommand{\bigcupn}[1][\oo]{\bigcup\seq[#1]{n}}
\newcommand{\capk}[1][\oo]{\cap\seq[#1]{k}}
\newcommand{\capi}[1][\oo]{\cap\seq[#1]{i}}
\newcommand{\capn}[1][\oo]{\cap\seq[#1]{n}}
\newcommand{\bigcapk}[1][\oo]{\bigcup\seq[#1]{k}}
\newcommand{\bigcapi}[1][\oo]{\bigcup\seq[#1]{i}}
\newcommand{\bigcapn}[1][\oo]{\bigcup\seq[#1]{n}}
\newcommand{\Dmn}{D_\mu\nu}
\newcommand{\Dnm}{D_nu\mu}
\newcommand{\loc}{loc}
\DeclareMathOperator{\Vol}{Vol}
\DeclareMathOperator{\diam}{diam}
\DeclareMathOperator{\BV}{BV}
\DeclareMathOperator{\Monp}{Mon^+}
\DeclareMathOperator{\AC}{AC}
\DeclareMathOperator{\Lip}{Lip}
\title{220A - Groups}
\author{John White}
\date{Fall 2022}




\begin{document}

\section*{Lecture 1}

Let $(X, \mc{A}, \mu)$ be a measure space. Without any additional structure or information, we may define the Lebesgue integral $\int_Xf\,d\mu$ for $f$ an $\mc{A}-\mc{B}$ measurable function $f:X\to[-\oo,+\oo]$. 

We only have a few examples without any work. 

\exm

\begin{itemize}

\item For any set $X$, we can define the counting measure on $\mc{A} = 2^X$, which gives $\mu(A) = |A|$. If $X = \N$, then a measurable function is just a sequence $(f_n)$, and $\int_Xf\,d\mu = \sum f_n$

\item We can also define the Dirac mass $\delta_p$ for a fixed $p \in X$ by 
\[
\delta_p(E) = \begin{cases} 1 & p \in E \\ 0 & p \not\in E\\ \end{cases}
\]
We have $\int_Xf\,d\delta_p = f(p)$

\end{itemize}

To get another example of a measure we need to do some work. 

\underline{Problem:} We want a measure $\mu$ on $\R^n$ such that, for a rectangle, 
\[
\mu([a_1,b_1]\times\cdots\times[a_n, b_n]) = |a_1 - b_1|\cdots|a_n - b_n|
\]
Once it is defined on all rectangles, it is defined on the minimal $\sigma$-algebra containing them, which is the Borel $\sigma$-algebra. In other words, this condition will completely specify a measure on the Borel $\sigma$-algebra $\mc{B}_{\R^n}$

If $X = \R^n$, or a general metric space, or even a general topological space, then $\mc{B}(X)$ denotes the $\sigma$-algebra generated by the open subsets of $X$. 

\underline{Problem:} 

Suppose we have a distribution function $F:\R\to\R$, meaning $F$ is monotone, positive, and $\lim_{x\to-\oo}f(x) = 0, \lim_{x\to\oo}f(x) = 1$, and continuous from the right. We want a Borel measure $\mu$ such that $F(t) = \mu((-\oo, t])$. Such a measure, denoted by $\lambda_F$, is called a Lebesgues-Stieltjes measure. 

The corresponding integral is called a Lebesgue-Stieltjes integral.

If $F$ is smooth, then $\int_{\R}\phi\,d\lambda_F = \int_{-\oo}^{\oo}\phi(x)d\,F(x)$.

The measure we want on $\R^n$ is denoted by $\lambda^n$. 

\subsection*{\underline{The Carath\'eodory Construction}}

Suppose we have an outer measure $\gamma:2^X\to[0,\oo]$. This means $\gamma(\varnothing) = 0$, $A \subset B \implies \gamma(A) \leq \gamma(B)$ (monotone), and $\gamma(\cup\seq{i}E_i) \leq \sum\seq{i}\gamma(E_i)$ (subadditive).

We can define a set $S$ to be $\gamma$-measurable if for every testing set $T$, $\gamma(T) = \gamma(S \cap T) + \gamma(S^c \cap T)$.

\thm (Carath\'eodory Extension Theorem)

\begin{enumerate}

\item $\gamma(N) = 0 \implies N$ is measurable. 

\item The set of measurable sets forms a $\sigma$-algebra $\Gamma$. 

\item $\gamma$ restricted to $\Gamma$ forms a measure. 

\end{enumerate}

``Nothing in the above theorem can guarantee you that $\Gamma$ is not trivial, i.e. $\Gamma = \{\varnothing, X\}$. Nevertheless, this is a very useful guy" - Dennis. 

\defn (Lebesgue outer measure on $\R^n$)

Let $R$ be a rectangle in $\R^n$, that is $R = \prod_{i=1}^n[a_i, b_i]$. We have $\Vol(R) = |a_1 - b_1|\cdots|a_n - b_n|$. For any $E \subseteq\R^n$, we define
\[
\mu^*(E) \eqdef \inf\{\sum\seq{j}\Vol(R_j) \mid E \subseteq \cup\seq{j}R_j\}
\] 

\prop

$\mu^*$ is an outer measure on $\R^n$ such that $\mu^*(R) = \Vol(R)$ for all rectangles $R$. 

\proof

The first and second axioms are trivial, so we will just prove the subadditivity. Let $E$ be some set. By definition, for any $\varepsilon$, there is some cover $R_j$ by recrtangles such that 
\[
-\varepsilon + \sum\seq{j}\Vol(R_j) \leq \mu^*(E) \leq \sum\seq{j}\Vol(R_j)
\]
meaning that $\sum\seq{j}\Vol(R_j) \leq \mu^*(E) + \varepsilon$. So for each $E_k$, there is a sequence $R^k_j$ which covers $E_k$, such that $\sum\seq{j}\Vol(R^k_j) \leq \mu^*(E) + \frac{\varepsilon}{2^k}$. 

So $\{R^k_j\}_{j, k \in \N}$ forms a cover of $\cup\seq{j}E_j$. Thus

\begin{align*}
\mu^*(\cup\seq{k}E_k) & \leq \sum\seq{k}\sum\seq{j}\Vol(R_j^k) \\
& \leq \sum\seq{k}\left(\mu^*(E_k) + \frac{\varepsilon}{2^k}\right) \\
& = \sum\seq{k}\mu^*(E_k) + \varepsilon
\end{align*}
This is true for any positive $\varepsilon$. Taking the limit as $\varepsilon\to0$ gives the result. 

\qed

Now, fix a rectangle $R$. Note that $R$ itself forms a cover of $R$, so by the definition, $\mu^*(R) \leq \Vol(R)$. For $\varepsilon > 0$, we can take an almost-optimal cover $(R_j)$ such that $\sum\seq{j}\Vol(R_j) \leq \Vol(R) + \varepsilon$. We can rig it such that $|\Vol(R_j) - \Vol(R)| \leq \frac{\varepsilon}{2^j}$.  Because $R \subset\cup\seq{j}R_j$, and $R_j$ is an open cover, by compactness of $R$ there is a finite subcover, and the volume of $R$ is less than or equal to the sum of the volumes of these finitely many $R_j$. So the volume of $R$ is less than or equal to $\mu^*(R) + 2\varepsilon$. So $\Vol(R) = \mu^*(R)$.

\prop 

Every rectangle $R$ in $\R^n$ is Carath\'eodory measurable). 

\proof

I missed this lol. Apparently Dennis denotes $\mc{M}_{\lambda^*}$ by $\ms{L}^n$.

\qed

\defn

A set is said to be \underline{$G_\delta$} if it is the countable intersection of open sets. A set is said to be \underline{$F_\sigma$} if it is the countable union of closed sets. 

\thm

\begin{enumerate}

\item For all $E \in \ms{L}^n$, $\lambda^N(E) = \inf\{\lambda^n(O) \mid \text{open }O\supseteq E\}$. 

\item $E \in \ms{L}^n$ if and only if $E = H\setminus Z$, where $H$ is $G_\delta$, and $\lambda^*(Z) = 0$. 

\item $E \in \ms{L}^n$ if and only if $E = H \cup Z$, where $H$ is $F_\sigma$ and $\lambda^*(Z) = 0$. 

\item $\lambda^n(E) = \sup\{\lambda^n(C) \mid \text{closed }C \subseteq E\}$

\end{enumerate}

\proof

It suffices to prove the first statement, as the others will follow by passing to a complement. 

\qed

\defn

Suppose $X$ is a metric space. A measure on $X$ is a \underline{Radon measure} if it is Borel (meaning defined on a $\sigma$-algebra containing Borel sets), and for any Borel $E$, $\mu(E) = \inf\{\mu(O) \mid \text{open }O \supseteq E\}$, and for any compact $C\subseteq X$, $\mu(C)<\oo$. 

\thm (Riesz)

Let $X \subseteq \R^n$ be compact. Let $C(X)$ denote the vector space of all continuous functions on $X$. This admits a norm $\norm{f}_{C(X)} = \sup_X|f|$, making it a Banach space. Define $C^*(X) = \{\phi:C(X)\to\R, \phi$ is linear and continuous $\}$. 

For all $\phi \in C^*(X)$, there exists a Radon measure $\mu = \mu_+$, and a function $M:X\to\{\pm1\}$ which is Borel, such that
\[
\phi(f) = \int_Xf(x)M(x)\,d\mu(x)
\]
for all $f \in C(X)$. 

\proof

\qed

\section*{Lecture 2, 1/17/23}

Note: This is the first lecture with Davit. Davit will always use $\mu$ to refer to an \underline{outer} measure, not a measure. The book will be ``Measure theory and fine properties of functions." According to Davit, this is the correct book to be using. 

\defn 

Let $X$ be a nonempty set. A mapping $\mu:2^X\to[0,+\oo]$ is called a \underline{measure} if it satisfies the following 2 properties.

\begin{enumerate}
\item $\mu(\varnothing) = 0$. 
\item (Countable subadditivity and monotonicity) If $A, A_1, A_2, \dots \subseteq X$ and $A \subseteq \cup\seq{i}A_i$ then $\mu(A)\leq\sum\seq{i}\mu(A_i)$
\end{enumerate}

\rem From the second definition, we can automatically get monotonicity, i.e. if $A\subseteq B$, then $\mu(A)\leq\mu(B)$. This is because, as written, the second definition is a statement not just about $\cup\seq{i}A_i$, but about any subset of it. Indeed, let $A = A, A_1 = B$, and $A_k = \varnothing$ for $k\geq 2$. Then we have $\mu(A)\leq\mu(\cup\seq{i}A_i) = \mu(A\cup B)$. 

We will write ``$\mu$ is a measure on $X$" to mean that $\mu$ satisfies the above definition (that is, $\mu$ is an outer measure).

\defn

Let $X$ be a nonempty set and let $\mu$ be a measure on $X$. For a fixed set $C\subseteq X$, define the \underline{restriction measure} $\nu = \mu|_C$ by $\nu(A) = \mu|_A(A) = \mu(A\cap C)$.

\rem It is easy to prove that $\mu|_C$ is a measure on $X$. 

\defn (Carath\'eodory's condition). Let $X$ be a nonempty set and let $\mu$ be a measure on $X$. A subset $A\subseteq X$ is called \underline{$\mu$-measurable} if, for all subset $B\subseteq X$, we have
\[
\mu(B) = \mu(B \cap A) + \mu(B \setminus A)
\]

\rem $X$ and $\varnothing$ are easily seen to be $\mu$-measurable. 

\thm (Carath\'eodory extension theorem)

The collection of $\mu$-measurable sets on a set $X$ is a $\sigma$-algebra.

\thm Let $X$ be a nonempty set and let $\mu$ be a measure on $X$. Then the following holds: 

\begin{enumerate}
\item $\varnothing$ and $X$ are $\mu$-measurable.
\item $A \subseteq X$ is $\mu$-measurable if and only if $X \setminus A$ is $\mu$-measurable.
\item If $A\subseteq X$ is such that $\mu(A) = 0$, then $A$ is $\mu$-measurable. 
\item Let $C \subseteq X$. Then anything which is $\mu$-measurable is $\mu|_C$-measurable. 
\end{enumerate}

\rem

A measure is also finitely subadditive, which says that if $A \subseteq A_1 \cup \cdots \cup A_n$, then $\mu(A) \leq \sum_{i=1}^n\mu(A_i)$. So, to check that $\mu(B) = \mu(B \cap A) + \mu(B\setminus A)$, it will suffice to check
\[
\mu(B)\geq\mu(B\cap A) + \mu(B\setminus A)
\]

\proof

Part 1 is obvious. 

Suppose that $A$ is $\mu$-measurable. Then $\mu(B\cap A) = \mu(B \setminus A^c)$ and $\mu(B\cap A^c) = \mu(B\setminus A)$ so $\mu(B \cap A) + \mu(B\setminus A) = \mu(B \cap A^c) + \mu(B\setminus A^c)$. So $A$ is $\mu$-measurable if and only if $A^c$ is.

Suppose that $\mu(A) = 0$. Then $\mu(B\cap A) \leq \mu(A), \mu(B)$, so $\mu(B\cap A) = 0$ for any $B\subseteq X$. Now, $B\setminus A \subseteq B$, so by monotonicity $\mu(B\setminus A)\leq\mu(B)$. So $\mu(B \cap A) + \mu(B\setminus A) \leq \mu(B)$ for all $B \subseteq X$, so we are done. 

Let $A$ be $\mu$-measurable. Then for any $B\subseteq X$ we have 
\begin{align*}
\nu(B) & = \mu|_C(B) = \mu(B \cap C) \\
		 & = \mu((B\cap C) \cap A) + \mu((B\cap C)\setminus A) \\
		 & = \nu(B \cap A) + \mu((B\setminus A) \cap C)\\
		 & = \nu(B \cap A) + \nu(B\setminus A) \\
\end{align*}

\qed

\thm

Let $X$ be a nonempty set and let $\mu$ be a measure on $X$. 

Assume $A_1, A_2, \dots, A_n\subseteq X$ are $\mu$-measurable. Then
\begin{enumerate}
\item $\bigcup_{k=1}^n A_k$ and $\bigcap_{i=1}^nA_k$ are also $\mu$-measurable. 
\item If the $A_i$ are disjoint, then $\mu(\bigcup_{i=1}^nA_i = \sum_{i=1}^n\mu(A_i)$
\end{enumerate}

\proof

We prove part 2 first. Because each $A_i$ is measurable, 
\begin{align*}
\mu(\cup_{k=1}^nA_k) & = \mu((\cup_{k=1}^nA_k)\cap A_n) + (\mu(\cup_{k=1}^nA_k) \setminus A_n) \\
& = \mu(\cup_{i=1}^{n - 1}A_k) + \mu(A_n) = \cdots = \sum_{k=1}^n\mu(A_k) \\
\end{align*}

Now we prove part 1. Let $A, B \subseteq X$ be $\mu$-measurable and disjoint. Then for any $C\subseteq X$, $\mu(C) = \mu(C \cap A) + \mu(C\setminus A)$, and similarly for $B$. This is equal to 

\begin{align*}
\mu(C) & = \mu(C \cap A) + \mu((C\setminus A) \cap B) + \mu(C\setminus A \setminus B) \\
		 & = \mu(C\cap A) + \mu(C\cap B) + \mu(C\setminus(A\cup B)) + \mu(C\cap(A\cup B)) (?)\\
		 & = \mu(C \cap (A \cup B) \cap A) + \mu(C \cap (A \cup B) \setminus A) \\
		 & = \mu(C \cap A) + \mu(C\cap B) \\
		 & = \mu(C \cap (A \cup B)) + \mu(C\setminus(A \cup B))
\end{align*}

So $A \cup B$ is $\mu$-measurable. (I got a bit lost in the arithmetic, sorry)

Next, we show if $A, B\subseteq X$ are $\mu$-measurable, then $A \cap B$ is $\mu$-measurable. This is straightforward. We will continue next time. 

\section*{Lecture 3, 1/19/23}

We will continue our proof of the theorem. Assume $A, B \subseteq X$ are $\mu$-measurable. We aim to show that $A \cap B$ is also $\mu$-measurable. We need to show that, for any $C \subseteq X$, we have $\mu(C) = \mu(C\cap(A\cap B)) + \mu(C\setminus(A\cap B))$. Because $A$, $B$ are $\mu$-measurable, we have
\begin{align*}
\mu(C) &  = \mu(C \cap A) + \mu(C\setminus A) \\
		 & = \mu((C\cap A) \cap B) + \mu((C\cap A) \setminus B) + \mu(C\setminus A) \\
		 & = \mu(C\cap (A \cap B)) + \mu((C\cap A) \setminus B) + \mu(C\setminus A) \\
		 & \geq \mu(C\cap (A \cap B)) + \mu(C\setminus(A \cap B))
\end{align*}
The opposite inequality follows by subadditivity, so we have equality. 

By induction, we get that also $\cap_{k=1}^n A_k$ is $\mu$-measurable. For the union, we can get it using the fact that $\cup_{k=1}^nA_k = X \setminus \cap_{k=1}^nA_k$.

\qed

\rem If $A, B$ are $\mu$-measurable, then $A \setminus B$ is $\mu$-measurable. This follows from $A \setminus B = A \cap (X \setminus B)$

\thm 

Let $X$ be a nonempty set, and $\mu$ a measure on $X$. Assume $\{A_k\}\seq{k}\subseteq X$ are $\mu$-measurable. Then
\begin{enumerate}
\item If the $A_k$ are disjoint, then we have countable additivity:
\[
\mu\left(\cup\seq{k}A_k\right) = \sum\seq{k}\mu(A_k)
\]
\end{enumerate}
\item If $A_1 \subseteq A_2 \subseteq \cdots$, then 
\[
\lim_{k\to\oo}\mu(A_k) = \mu\left(\bigcup\seq{k}A_k\right)
\]
\item If $A_1 \supseteq A_2 \supseteq \cdots$, and $\mu(A_1) < \oo$, then 
\[
\lim_{k\to\oo}\mu(A_k) = \mu\left(\bigcap\seq{k}A_k\right)
\]


\proof

We have from before that if the $A_k$ are pairwise disjoint, then $\mu(\cup_{k=1}^nA_k) = \sum_{k=1}^n\mu(A_k)$ for any $n \in \N$. Because $\cup_{k=1}^nA_k \subseteq \cup\seq{k}A_k$, we must have that $\mu(\cup_{k=1}^nA_k) \leq \mu(\cup\seq{k}A_k)$. Using the previous fact, and passing to a limit, we have
\[
\sum\seq{k}\mu(A_k) \leq \mu(\cup\seq{k}A_k)
\]
The opposite equality is automatically true by the countable subadditivity of $\mu$, so we get equality. This completes the proof of 1. Now for part 2. 

Define $B_k = A_k \setminus A_{k - 1}$, where $A_0 \eqdef \varnothing$. We have $A_k = \cup_{i=1}^kB_i$. Note that the $B_i$ are disjoint. So we have
\[
\mu(A_k) = \sum_{i=1}^k\mu(B_i)
\]
So, in the limit, 
\[
\lim_{k\to\oo}\mu(A_k) = \lim_{k\to\oo}\sum_{i=1}^k\mu(B_i) = \sum\seq{i}\mu(B_i)
\]
So
\[
\mu(\cup\seq{i}B_i) = \mu(\cup\seq{k}A_k)
\]

Finally, let $A_1 \supseteq A_2 \supseteq \cdots, \mu(A_1)<\oo$. Define $B_k = A_1\setminus A_k$. This is decreasing sequence of $\mu$-measurable sets, so by the previous part, 
\[
\lim_{k\to\oo}\mu(B_k) = \mu(\cup\seq{k}B_k)
\]
So
\begin{align*}
\mu(B_k) = \mu(A_1\setminus A_k)  = \mu(A_1) - \mu(A_k) \implies \\
\lim_{k\to\oo}\mu(B_k) = \lim_{k\to\oo}(\mu(A_1) - \mu(A_k)) & = \mu(A_1) - \lim_{k\to\oo}\mu(A_k) \\
= \mu(\cup\seq{k}B_k) = \mu(\cup\seq{k}(A_1 \setminus A_k)) & = \mu(A_1 \setminus \cap\seq{k} A_k) \\
\geq \mu(A_1) - \mu(\cap\seq{k}A_k) & \\
\end{align*}

So $\lim_{k\to\oo}\mu(A_k) \leq \mu(\cap\seq{k}A_k)$. The opposite inequality follows easily by monotonicity. 

\qed

We are ready to prove the Carath\'eodory extension theorem. 

\proof

Let $A_1, A_2, \dots \subseteq X$ be $\mu$-measurable. It will suffice to prove that $\cup\seq{k}A_k$ is $\mu$-measurable. So we need to check that, for any $B$, 
\[
\mu(B) = \mu(B\cap (\cup\seq{k}A_k)) + \mu(B \setminus \cup\seq{k}A_k)
\]
Fix $B \subseteq X$, and consider $\nu = \mu|_B$. Recall this is defined as $\nu(C) = \mu(B \cap C)$. We would like 
\[
\nu(B) = \nu(\cup\seq{k}) + \nu(B \setminus \cup\seq{k}A_k)
\]
To this end, 
\begin{align*}
\nu(\cup\seq{k}A_k) & = \lim_{k\to\oo}\nu(\cup_{i=1}^kA_i) \\
\end{align*}
Without loss of generality, $\nu(B)<\oo$. If $\nu(B) = \oo$, then we are done trivially. 

As before, define $B_k = B\setminus \cup_{i=1}^kA_i$. Then
\[
\nu(B\setminus\cup\seq{k}A_k) = \lim_{k\to\oo}\nu(B\setminus\cup_{i=1}^kA_i)
\]
So, 
\begin{align*}
\nu(\cup\seq{k}A_k) + \nu(B\setminus\cup\seq{k}A_k) & = \lim_{k\to\oo}\left(\nu(\cup_{i=1}^kA_i) + \nu(B\setminus\cup_{i=1}^kA_i)\right) \\
& = \lim_{k\to\oo}\nu(B) = \nu(B) \\
\end{align*}
so we are done. 

\qed

\defn

Let $X$ be a nonempty set, and let $\mu$ be a measure on $X$. Then $\mu$ is said to be 
\begin{enumerate}
\item A \underline{regular measure} if, for any $A \subset X$, there exists a $\mu$-measurable $B\subseteq X$ such that $A \subseteq B$, and $\mu(A) = \mu(B)$. 
\item A \underline{Borel measure} if all Borel sets (i.e. the elements of the Borel $\sigma$-algebra) are measurable. This only applies if $X$ is also a topological space, of course. 
\item A \underline{Borel-regular measure} if $\mu$ is Borel, and for any $A \subseteq X$, there exists a Borel set $B \subseteq X$ such that $A \subseteq B$ and $\mu(A) = \mu(B)$. 
\item A \underline{Radon measure} if it is Borel-regular and $\mu(K)<\oo$ if $K$ is compact. 
\end{enumerate}

\rem

Note that being Borel and regular is weaker than being Borel-regular. 

\thm (Increasing sets \underline{for regular measures})
Let $X$ be a nonempty set, and let $\mu$ be a regular measure on $X$. Assume $A_1\subseteq A_2 \subseteq \cdots \subseteq X$. Then 
\[
\lim_{k\to\oo}\mu(A_k) = \mu\left(\bigcupk A_k\right)
\]

\rem The sets $A_k$ need not be $\mu$-measurable. 

\proof

For all $A_k$, there is a $C_k\subseteq X$ which is $\mu$-measurable, $A_k \subseteq C_k$, and $\mu(A_k) = \mu(C_k)$. Let $D_k = \cap_{i\geq k}C_i$. For $i \geq k$, we can see $A_k \subseteq A_i \subseteq C_i$. $A_k \subseteq \cup_{i\geq k}C_i = D_k$, then $\mu(A_k)\leq\mu(D_k)$. On the other hand, $D_k\subseteq C_k$, so $\mu(D_k)\leq\mu(C_k) = \mu(A_k)$. So
\begin{itemize}
\item $\mu(A_k) = \mu(D_k)$ 
\item $A_k \subseteq D_k$ 
\item $D_k$ is $\mu$-measurable and $D_1 \subseteq D_2 \subseteq \cdots $
\end{itemize}
So
\begin{align*}
\lim_{k\to\oo}\mu(A_k) & = \lim_{k\to\oo}\mu(D_k) \\
						      & = \mu\left(\bigcupk D_k\right) \\
								& \geq \mu\left(\bigcupk A_k\right) \\
\end{align*}
Because $\cupk A_k \subseteq \cupk D_k$, 
\[
\lim_{k\to\oo}\mu(A_k)\geq\mu\left(\bigcupk A_k\right) 
\]
But $A_k \subseteq \cupk A_k$, so the opposite inequality is also true, so we have equality. 

\qed

\section*{Lecture 4, 1/24/23}

\thm (Restriction and Radon measures)

Let $X$ be a topological space and let $\mu$ be a Borel-regular measure on $X$. Let $A \subseteq X$ be $\mu$-measurable with $\mu(A)<\oo$. Then the restriction measure $\nu = \mu|_A$ is Radon. 

\proof

First, $\nu$ is a finite measure, as $\nu(X) = \mu(A \cap X) = \mu(A) < \oo$ for any $C \subseteq X$. 

It is clear that $\nu$ is Borel, as $\mu$ is Borel. Next, we show $\nu$ is Borel-Regular. Without loss of generality, we may assume that $A$ is Borel, because $\mu$ is Borel-regular. Explicitly, we know there is a Borel set $B \subseteq X$ such that $A \subseteq B$ and $\mu(B) = \mu(A)$. We will show $\mu|_A = \mu|_B$. 

We have $\mu(B) = \mu(B \cap A) + \mu(B \setminus A) = \mu(A) + \mu(B \setminus A)$. So $\mu(B\setminus A) = 0$. 

So, for all $C \subseteq X$, 
\begin{align*}
\mu|_B(C) & = \mu(B \cap C)\\
			  & = \mu((B\cap C)\cap A) + \mu((B \cap C) \setminus A) \\
			  & = \mu(C \cap A) + \mu((B \cap C) \setminus A ) \\
			  & \leq \mu|_A(C) + \mu(B\setminus A) \\
			  & = \mu|_A(C) \\
\end{align*}
But $(A \cap C) \subseteq (B \cap C)$, so $\mu|_A(C) \leq \mu|_B(C)$, so we may conclude that $\mu|_A = \mu|_B$. 

So assume $A$ is Borel. Fix $C \subseteq X$. We need to prove that there exists a Borel $D\subseteq X$ such that $C\subseteq D$ and $\nu(C) = \nu(D)$. There exists a Borel $E \subseteq X$ such that $C \cap A \subseteq E$, and $\mu(C \cap A) = \mu(E)$. So $D = E \cup (X\setminus A)$ is Borel and $C \subseteq D$. 

So 
\begin{align*}
\nu(D) & = \mu((E\cup(X\setminus A))\cap A)\\ &  = \mu(E \cap A) \\ & \leq\mu(E) \\
& = \mu(C\cap A) \\
& = \nu(C) \\
\end{align*}
$C \subseteq D$ so $\nu(C)\leq\nu(D)$, so $\nu(C)=\nu(D)$.

\qed

\thm (Carath\'eodory Criterion for being Borel) 

Let $X$ be a metric space and let $\mu$ be a measure on $X$. Then $\mu$ is Borel if and only if, for all $A, B \subseteq X$ with $d(A, B) > 0$ (meaning $\inf\{d(a, b) \mid a \in A, b \in B\} > 0$), 
\[
\mu(A \cup B) = \mu(A) + \mu(B)
\]

\proof

\subsection*{$=>$}

Suppose $\mu$ is Borel. We will use $\bar{B}$ to denote the closure of $B$. Then $d(A,\bar{B}) = d(A, B) > 0$. By measurability of $\bar{B}$, 
\[
\mu(A \cup B) = \mu((A \cup B) \cap \bar{B}) + \mu((A \cup B) \setminus \bar{B}) = \mu(B) = \mu(A)
\]

\subsection*{$<=$}

Suppose that, for $A, B$ with $d(A, B) > 0$, $\mu(A \cup B) = \mu(A) + \mu(B)$. We will show that this implies $\mu$ is Borel. Let us show that every closed subset $C\subseteq X$ is $\mu$-measurable. So we have to prove that for every $A \subseteq X$, 
\[
\mu(A) = \mu(A \cap C) + \mu(A\setminus C)
\]
We have $\leq$ trivially. Assume $\mu(A)<\oo$; otherwise, this equality holds trivially. 

Define for every $n \in \N$ the set $C_n = \{x\in X \mid d(x, C) \leq \frac{1}{n}\}$. We can see $d(A\setminus C_n, C) \geq \frac{1}{n} > 0$. So
\begin{align*}
\mu({(A\setminus C_n)\cup (A \cap C)}_{\subseteq A}) & = \mu(A \setminus C_n) + \mu(A \cap C) \\
& \leq \mu(A)\\
\end{align*}
SO $\mu(A \setminus C_n) + \mu(A \cap C) \leq \mu(A)$ for all $n \in \N$. We will prove that $\lim_{n\to\oo}\mu(A\setminus C_n) = \mu(A \setminus C)$. 

Consider the annuli $R_n = \{x \in A \mid \frac{1}{n + 1} < d(x, C) \leq \frac{1}{n}$. We have
\[
(A\setminus C_1) \bigcupn R_n \subseteq A \setminus C
\]
$C$ is closed, so in fact we have equality above. Why? If a point belongs to $A \setminus C$, then it does not belong to $C$, so $d(x, C) > 0$. So there is an $n\in\N$ such that $x \in R_n$ or $x \in A \setminus C_1$. We have
\begin{align*}
\mu\left(\bigcup_{k=0}^nR_{2k + 1}\right) & = \sum_{k=0}^n\mu(R_{2k + 1})  \leq \mu(A) \\
\mu\left(\bigcup_{k=1}^nR_{2k}\right) & = \sum_{k=1}^n\mu(R_{2k}) \leq \mu(A) \\
\end{align*}

So $\sum\seq{n}\mu(R_n) \leq 2\mu(A)<\oo$, so $\lim_{n\to\oo}(\sum_{k=n}^\oo\mu(R_k)) = 0$. So $(A \setminus C_n) \bigcup_{k=n}^\oo R_k = A \setminus C$

So by subadditivity, 
\[
\mu(A\setminus C) \leq \mu(A\setminus C_n) + \sum_{k=n}^\oo\mu(R_k)
\]
So as $n\to\oo$, 
\[
\mu(A \setminus C) \leq \liminf_{n\to\oo}\mu(A\setminus C_n) \leq \mu(A \setminus C)
\]
This completes the proof. 

It is time for our third section. 

\subsection*{\underline{Approximation by open, closed, and compact sets}}

\thm Let $\mu$ be a Borel measure on $\R^n$, and let $B \subseteq \R^n$ be a Borel set. 

\begin{enumerate}

\item If $\mu(B)<\oo$, then for any $\varepsilon>0$, there exists a closed $C \subseteq B$ such that $\mu(B\setminus C) < \varepsilon$. 

\item If $\mu$ is a Radon measure, then for all $\varepsilon>0$, there exists an open $U \supseteq B$ such that $\mu(U\setminus B) < \varepsilon$. 

\end{enumerate}

\proof

\begin{enumerate}

\item Let $\nu = \mu|_B$, a finite measure on $\R^n$. 

Define the collection $\ms{F} = \{A\subseteq\R^n\mid A$ is $\mu$-measurable and for all $\varepsilon > 0, $ there exists a closed $C \subseteq  A$ such that $\mu(A\setminus C) < \varepsilon \}$

Our goal is to show that $\ms{B}_{\R^n}\subseteq\ms{F}$. Davit uses ``$\sigma_B$" to indicate the Borel $\sigma$-algebra. 

By previous discussion, $\ms{F}$ contains all closed sets. 

Now, if $A_1, A_2, \dots \in \ms{F}$, then $\bigcapk A_k\in\ms{F}$. For all $A_k$, there exists a closed $C_k \subseteq A_k$, such that $\nu(A_k\setminus C_k) < \frac{\varepsilon}{2^k}$. Then by subadditivity, 
\[
\nu(\bigcapk A_k \setminus \bigcapk C_k) \leq \nu(\bigcupk(A_k \setminus C_k)) \leq \sum\seq{k}\nu(A_k\setminus C_k) < \varepsilon
\]
and $C = \bigcapk C_k$ is closed. 


\end{enumerate}

\section*{Lecture 5, 1/26/23}

\thm 

Let $\mu$ be a Borel measure on $\R^n$ and let $B \subseteq \R^n$ be a Borel set. 
\begin{enumerate}

\item If $\mu(B)<\oo$, then for all $\varepsilon>0$, there exists a closed $C \subseteq B$ such that $\mu(B\setminus C) < \varepsilon$. 

\item If $\mu$ is a Radon measure, then for all $\varepsilon>0$, there exists an open $U \subseteq \R^n$ such that $B \subseteq U$ and $\mu(U\setminus B)<\varepsilon$. 

\end{enumerate}

\proof
\,
\begin{enumerate}

\item Let $\nu = \mu|_B$ be a finite Borel measure on $\R^n$.  Define the collection
\[
\ms{F} = \{A\subseteq \R^n:A \mu-\text{measurable and for all }\varepsilon>0,\exists C \subseteq A, C\text{ closed, }\nu(A\setminus C)<\varepsilon\} 
\]
We want to show $\mc{B} \in \ms{F}$, where $\mc{B}$ is the Borel set. 

\subsubsection*{Step 1:} $\ms{F}$ contains all closed sets

\subsubsection*{Step 2:} If $A_1, A_1, \dots, A_k \in \ms{F}$, then for all $A_k$, there exists a closed $C_k$ such that $\nu(A_k\setminus C_k)<\frac{\varepsilon}{2^k}$. Thus, 
\begin{align*}
\nu\left(\bigcapk A_k\setminus \bigcapk C_k\right) & \leq \nu\left(\bigcupk(A_k\setminus C_k)\right) \\
& \leq \sum_{k=1}^\oo\nu(A_k\setminus C_k) < \varepsilon \\
\end{align*}
Furthermore, $\capk C_k$ is closed. Thus $\ms{F}$ is closed under countable intersections. 

\subsubsection*{Step 3:} We want to show countable unions belong to $\ms{F}$. If $A_1, A_2, \dots, A_k, \dots \in \ms{F}$, then for all $A_k$, there is a closed $C_k$ such that $\nu(A_k\setminus C_k)<\frac{\varepsilon}{2^k}$. However, we do not know if $\cupk C_k$ is closed. Note that $\nu(\cupk A_k \setminus \cupk C_k) = \lim_{m\to\oo}\nu(\cup_{k=1}^m A_k \setminus \cup_{k=1}^m C_k) < \varepsilon$. So there is an $m \in \N$ such that $\nu(\cup_{k=1}^m A_k \setminus \cup_{k=1}^m C_k) < \varepsilon$. Furthermore $C = \cupk[m]C_k$ is closed. 

\subsubsection*{Step 4:} In the homework, we showed that every open set is the countable union of closed balls. Since $\ms{F}$ contains all closed sets, and is closed under countable unions, $\ms{F}$ contains all open sets. 

\subsubsection*{Step 5:} Consider the subset $G \subseteq\ms{F}$ given by $G = \{A\in\ms{F}\mid A^c\in\ms{F}\}$. We claim that $G$ is a $\sigma$-algebra. Going through the axioms, 
\begin{enumerate}[label=(\roman*)]

\item Clearly, $\varnothing\in G$. 

\item If $A \in G, A^c \in G$. 

\item If $A_1, A_2, \dots, A_k, \dots \in G$, then $\cupk A_k \in G$. Why? $\cupk A_k \in \ms{F}$ and $\R^n\setminus\capk(\R^n\setminus A_k) \in\ms{F}$, since each $\R^n\setminus A_k \in \ms{F}$, and $\ms{F}$ is closed under countable intersections. 

\end{enumerate}

\subsubsection*{Step 6:} Since the complement of an open sets is a closed set, and since $\ms{F}$ contains all open and closed sets, all open sets are contained in the $\sigma$-algebra $G$. Thus the Borel sets are contained in $G$, implying that they are contained in $\ms{F}$. 

Note: Part 1 requires that $X$ be a seperable metric space. 

\item For all $m \in \N$, denote $U_m = B_m(0) = \{x\in\R^n\mid\norm{x}<m\}$. 


Note that $\mu(U_m\setminus B) \leq \mu(U_m) < \oo$. Thus there exists a closed $C_m \subseteq U_m\setminus B$ such that $\mu((U_m\setminus B)\setminus C_m)<\frac{\varepsilon}{2^m}$. 

Note that $B \cap U_m \subseteq (U_m\setminus C_m)$, which is an open set. 

Thus, $\mu((U_m\setminus C_m)\setminus(B\cap U_m)_ = \mu((U_m\setminus C_m)\setminus B_ < \frac{\varepsilon}{2}$. 

Define $U = \cup\seq{m}(U_m\setminus C_m)$, which is an open set. 

Thus $B = \cup\seq{m}(B\cap U_m) \subseteq \cup\seq{m}(U_m\setminus C_m) = U$. 

Furtheremore,
\begin{align*}
\mu(U\setminus B) & = \mu(\cup\seq{m}(U_m\setminus C_m) \setminus \cup\seq{m}(B \cap U_m)) \\
& \leq \mu(\cup\seq{m}(U_m\setminus C_m \setminus B \cap U_m)\\
& \leq \sum\seq{m}\mu(U_m\setminus C_m\setminus B\cap U_m) \\
& < \varepsilon \\
\end{align*}

Note: Part 2 requires that for all $r>0$, for all $x \in X$, $\mu(B_r(x))<\oo$. 




\end{enumerate}

\qed

\thm (Approximation by compact and open sets). 

Let $\mu$ be a Radon measure on $\R^n$. Then 
\begin{enumerate}

\item For all $A \subseteq\R^n$, $\mu(A) = \inf\{\mu(U) \mid A \subseteq U, U$ open $\}$. 

\item For all $\mu$-measurable $A \subseteq\R^n$, we have $\mu(A) = \sup\{\mu(K) \mid K\subseteq A, K$ compact $\}$. 

\end{enumerate}

\proof

Shortly. 

Note: If $\mu$ is the Lebesgue measure, we define the outer measure as
\begin{align*}
\mu(A) & = \inf\{\sum\seq{k}(b_k - a_k) \mid A \subseteq\cupk(a_k, b_k) \} \\
& = \inf\{\sum\seq{k}(b_k - a_k)\mid A \subseteq \cupk\, [a_k, b_k), [a_i, b_i) \cap [a_j, b_j) = \varnothing\text{ if } i\neq j\} \\
\end{align*}

\rem Let $X$ be a topological space, and let $\mu$ be a measure on $X$. If $A \subseteq X$ is such that for all $\varepsilon>0$, there exists a $\mu$-measurable $A_\varepsilon\subseteq A$, such that $\mu(A\setminus A_\varepsilon)<\varepsilon$. Then $A$ is $\mu$-measurable. 

\proof of remark. 

Take $\varepsilon = \frac{1}{k}$. By part 1, there exists $A_k \subseteq A$ such that $0\leq\mu(A)-\mu(A_k)<\frac{1}{k}$. 

Let $b = \cupk A_k$, and note that $B \subseteq A$, $B$ is $\mu$-measurable, and $\mu(A)-\frac{1}{k} \leq \mu(A_k) \leq \mu(B) $ for all $k \in \N$. 

This implies that $\mu(B) = \mu(A), B \subseteq A$. 

Thus, $\mu(A) = \mu(A\cap B) + \mu(A\setminus B) = \mu(B) + \mu(A\setminus B)$. 

Since $\mu(A) = \mu(B)$, this implies that $\mu(A\setminus B) = 0$, i.e. $A\setminus B$ is $\mu$-measurable. 

So $A = B \cup (A \setminus B)$ is $\mu$-measurable. 

\qed

We are now prepared for a proof of the theorem. 

\proof

\begin{enumerate}

\item If $\mu(A)=\oo$, then this is trivial. So assume $\mu(A)<\oo$. If $A$ is Borel, we use part 2 of theorem 1. 

We have an open $U_\varepsilon\subseteq\R^n$ such that $A \subseteq U_\varepsilon$, and $\mu(U_\varepsilon\setminus A)<\varepsilon$. 

Thus $\mu(A)\leq\mu(U_\varepsilon)\leq\mu(A) + \mu(U_\varepsilon\setminus A) < \mu(A) + \varepsilon$. 

So we are odne for Borel sets. If $A$ is not Borel, then there exists a Borel $B \subseteq \R^n$ such that $A \subseteq B$ and $\mu(A) = \mu(B)$. 

Note: For the above proof, we used a previous theorem, and the fact that $\mu$ is Borel-regular. 

\item Lets prove that $\mu(A) = \sup\{\mu(C)\mid C \subseteq A, C$ closed $\}$. 
We have two cases:

\begin{enumerate}[label=(\roman*)]

\item $\mu(A)<\oo$. In this case, consider $\nu = \mu|_A$. $\nu$ is a finite Radon measure. We apply part 1 to $A^c$. There exists an open $U_\varepsilon\subseteq\R^n$ such that $\R^n\setminus A \subseteq U_\varepsilon$ and $\nu(U_\varepsilon)<\nu(A^c) + \varepsilon = \varepsilon$. 

Set $C_\varepsilon = U_\varepsilon^c$, which is closed. Note that $C_\varepsilon\subseteq A$ and $\mu(A\setminus C_\varepsilon) = \mu(A \cap U_\varepsilon)= \nu(U_\varepsilon) < \varepsilon$. 

Since $C \subseteq A$, $\mu(C_\varepsilon)\leq\mu(A)$. By countable subadditivity, $\mu(A)\leq\mu(C_\varepsilon) + \mu(A\setminus C_\varepsilon) < \mu(C_\varepsilon) + \varepsilon$. 

\item Next time!

\end{enumerate}

\end{enumerate}

\qed

\section*{Lecture 6, 1/31/23}

\rem 

Let $\mu$ be a Borel measure on $X$. If for every $A \subseteq X$ one has $\mu(A) = \inf\{\mu(B) \mid A \subseteq B, B$ Borel $\}$, then $\mu$ has to be Borel-regular. 

\proof of remark. 

Take $B_k\supseteq A$ such that $\mu(B_k) < \mu(A) + \frac{1}{k}$. Let $B = \capk B_k$. Note $B$ is Borel, $A \subseteq B$, and $\mu(A)\leq\mu(B)\leq\mu(B_k)<\mu(A) + \frac{1}{k}$. So $\mu(A) = \mu(B)$. 

\qed 

\proof of theorem from last time. 

We showed that if $\mu(A)<\oo$, then $\mu(A) = \sup\{\mu(C)\mid C \subseteq A, C$ closed $\}$. 

If $\mu(A) = \oo$, write $\R^n = \cupk R_k$, where $R_k = \{x\in\R^n\mid k \leq \norm{x} < k + 1\}$. Thus $A = \cupk A \cap R_k$. 

For all $k, \mu(A\cap R_k) \leq \mu(R_k) < \oo$. So there exists $C_k \subseteq A\cap R_k$ such that $\mu(C_k) > \mu(A\cap R_k) - \frac{1}{2^k}$. 

Thus $\mu(\cupk C_k) = \sup\seq{k}\mu(C_k) \geq \sum\seq{k}*\mu(A\cap R_k) - \frac{1}{2^k}) \geq \mu(A) - 2 = \oo$. 

This implies that $\lim_{m\to\oo}\mu(\cupk[m]C_k = \mu(\cupk C_k) = \oo$. 

This proves the theorem for closed sets. 

Now we prove the theorem for compact sets. 

Case 1: $\mu(A)<\oo$. For all $\varepsilon > 0$, there exists a closed set $C_\varepsilon\subseteq A$ such that $\mu(C_\varepsilon) > \mu(A) - \varepsilon$. Consider $K_m = C_\varepsilon\cap B_m$, where $B_m = \{x\in\R^n \mid \norm{x}\leq m\}$. 

Note $K_m$ is compact. 

Thus $\lim_{m\to\oo}\mu(K_m) = \mu(\cupk K_m) = \mu(C_\varepsilon) > \mu(A) - \varepsilon$. 

Case 2: $\mu(A) = \oo$. For all $m \in \N$, there exists a closed $C_m\subseteq A$ such that $\mu(C_m)\geq m$. Apply the same procedure. 

\qed

\subsubsection*{\underline{Covering theorems} (Vitali's and Besicovitch)}

Notation: we will work in $\R^n$. Closed balls will be denoted by $B$. For a given closed ball $B = B_r(x) = \{y\in\R^n \mid \norm{x - y} \leq r\}$, $\hat{B} = cB = \{y\in\R^n \mid \norm{x - y} \leq cr\} = B_{cr}(x)$. 

\defn

Let $A \subseteq \R^n$ and let $\ms{F} = \{B\subseteq\R^n\}$ be a family of balls. 
\begin{enumerate}
\item $\ms{F}$ is a \underline{cover of $A$} if $A \subseteq \cup_{B\in\ms{F}}B$. 

\item $\ms{F}$ is a \underline{fine cover of $A$} if for all $x \in A$ and $\varepsilon>0$, there exists a $B\in\ms{F}$ such that $x \in B$ and $\diam(B) < \varepsilon$. Alternatively, for all $x \in X$, $\inf\{\diam(B) \mid x \in B\} = 0$. 

\end{enumerate}

\thm (Vitali's Covering Theorem)

Let $\ms{F}$ be a collection of nondegenerate closed balls in $\R^n$ with diameters uniformly bounded, i.e. $\sup\{\diam(B) \mid B \in \ms{F}\} < \oo$. Then there exists a subcollection of countably many disjoint balls $\{\hat{B}_i\}\seq{i}$, such that $\cup_{b\in B}B \subseteq \cupi\hat{B}_i$

\proof 

Denote $D = \sup\{\diam(B) \mid B \in \ms{F}\}$ and consider $\ms{F}_k = \{B\in\ms{F}\mid\frac{D}{2^k} < \diam(B) \leq \frac{D}{2^{k - 1}}\}$. Let $G_1 \subseteq \ms{F}_1$ be a maximal disjoint subcollection of balls in $\ms{F}$ (we can produce this easily with Zorn's Lemma). It will be maximal in the sense that if we add another element, it will not be a disjoint set. 

Assume $G_1, \dots, G_{k - 1}$ have been chose. 

Let $G_k$ bew a maximal disjoint subcollection in $\ms{F}$ such taht the balls at $G_k$ do not intersect with the balls in $\cup_{i=1}^{k - 1}G_i$.

Set $G = \cupk G_k \subseteq \ms{F}$. Let $B \in \ms{F}_m$, i.e. $\frac{D}{2^m} < \diam(B) \leq \frac{D}{2^{m - 1}}$. 

Because $G_m$ is maximal, there exists $\bar{B} \in \cupi[m]G_i$ such that $B \cap\bar{B} = \varnothing$. Thus $\diam(\bar{B}) \geq \frac{D}{2^m} \geq \frac{1}{2}\diam(B)$. Thus $B \subseteq \hat{\bar{B}}$. 

\section*{Lecture 7, 2/2/23}

\cor

Let $A \subseteq \R^n$, and let the collection $\ms{F}$ of nondegenerate closed balls be a fine cover of $A$ such that $\sup\{\diam(B)\mid B\in\ms{F}\} < \oo$. 

Then for any finite number of balls $B_1, B_2, \dots, B_m\in\ms{F}$, one has
\[
A\setminus\cupi[m]B_i \subseteq \cup_{B\in G\setminus\{B_1, \dots, B_m\}}\hat{B}
\]
where $G$ is the disjoint collection of balls guaranteed by Vitali's theorem. 

\proof

Assume $x \in A\setminus\cupi[m]B_i$. Then $x\not\in\cupi[m]B_i$, so $d(x, \cupi[m]B_i) > 0$, as $B_i$ is closed for all $i$, and the finite union of closed sets is closed. 

Let $d = d(x, \cupi[m]B_i)$. Because $x \in A$, there exists a ball $B = B_r(y) \in \ms{F}$ such that $x\in B_r(y)$ and $2r<d$. This gives us that $B \cap \cupi[m]B_i = \varnothing$. By the construction of $G$, there exists $\bar{B} \in G$ such that $B \cap \bar{B} = \varnothing$ and $B \subseteq \hat{\bar{B}}$. Now, $\bar{B}\neq B_i, i = 1, 2, \dots, m$, and therefore $x \in B \subseteq\hat{\bar{B}}\subseteq \cup_{\bar{\bar{B}}\in G\setminus\{B_1, \dots, B_n\}}\hat{\bar{\bar{B}}}$. 

\thm (Filling open sets with closed balls)

Let $U \subseteq \R^n$ be open, and let $\delta > 0$. Then there is a countable collection $G$ of nondegenerate closed, disjoint balls, such that 
\[
\sup\{\diam(B) \mid B \in G\} \leq \delta
\]
and $\ms{L}^n(U\setminus\cupi B_i) = 0$, where $\ms{L}^n$ denotes $n$-dimensional Lebesgue measure. Here, $G = \{B_i\}\seq{i}$. 

\proof
\,

\subsubsection*{Case 1: $\ms{L}^n(U) < \oo$}

Consider the collection of nondegenerate closed balls $\ms{F} = \{ B \subseteq U \mid \diam(B)\leq\delta\}$. Because $U$ is open, $\cup_{B\in\ms{F}}B = U$. By Vitali's covering theorem, there exists a countable family $G$ of disjoint balls such that $U\subseteq\cupi\hat{B_i}$. So
\[
\ms{L}^n(U)\leq\ms{L}^n(\cupi\hat{B}_i) \leq \sum\seq{i}\ms{L}^n(\hat{B_i})
\]
By countable subadditivity, $\ms{L}^n(U) \leq 5^n\sum\seq{i}\ms{L}^n(B_i) = 5^n\ms{L}^n(\cupi B_i)$. So
\[
\ms{L}^n(U\setminus\cupi B_i) = \ms{L}^n(U) - \ms{L}^n(\cupi B_i) \leq (1 - \frac{1}{5^n})\ms{L}^n(U)
\]
Now
\[
\lim_{m\to\oo}\ms{L}^n(U\setminus\cupi[m]B_i) = \ms{L}^n(U\setminus\cupi B_i) \leq (1 - \frac{1}{5^n})\ms{L}^n(U)
\]
So there exists an index $m_1 \in \N$ such that $\ms{L}^n(U \setminus \cup_{i=1}^{m_1} B_i ) \leq (1 - \frac{1}{2\cdot 5^n})\ms{L}^n(U)$. 

Consider $U_2 = U \setminus \cup_{i=1}^{m_1}B_i$ and the new collection $\ms{F}_i = \{B\mid B \subseteq U_2, \diam(B)\leq\delta\}$. Then 
\[
\ms{L}^n(U_2) \leq q\cdot\ms{L}^n(U) < \oo
\]
So there exist disjoint closed $B_{m_{1 + 1}},\cdots, B_{m_2} \in \ms{F}_2$ such that
\[
\ms{L}^n(U_2 \setminus \cup_{i=m_1 + 1}^{m_2}B_i) \leq q\ms{L}^n(U)
\]
So $\ms{L}^n(U\setminus\cup_{i=1}^{m_2})\leq q^2\ms{L}^n(U)$. 

\underline{$k$-th step}

There are disjoint balls $B_1, B_2, \dots, B_{m_k} \subseteq U$ such that
\[
\ms{L}^n(U\setminus \cup_{i=1}^{m_k}) \leq q^k\ms{L}^k(U)
\]
So
\[
\ms{L}^n(U \setminus \cupi B_i) \leq \ms{L}^n(U\setminus\cup_{i=1}^{m_k}) \leq q^k\ms{L}^n(U)
\]
The above is true for every $k$, so it follows that $\ms{L}^n(U \setminus \cupi B_i) = 0$. $G = \{B_i\}\seq{i}$. This completes the proof in the case of $\ms{L}^n(U)<\oo$. 

\subsubsection*{Case 2, $\ms{L}^n(U) = \oo$}

Consider $U_m = U \cap \{x\in \R^n: m - 1 < |x| < m\}$, $m =1, 2, \dots$. We know $\ms{L}^n(\del B_r(x)) = 0$ for all $B_r(x) \subseteq \R^n$ (this will be a homework problem). 

\qed

``You can go look at the proof of Besicovitch in the book , but to be honest I never read that proof." - Davit


\thm (Besicovitch's covering theorem) 

There exists a number $N_n$ that depends only on the space dimension $n$, with the following property. 

If $\ms{F}$ is any collection of nondegenerate closed balls in $\R^n$, with 
\[
\sup\{\diam(B)\mid B \in \ms{F}\}<\oo
\]
and $A = \{x \mid \exists B_r(x) \in \ms{F}$ (the centers of the balls). 

Then there exists $N_n$ countable collections $G_1, G_2, \dots, G_{N_n}$, each of which are disjoint (as in, the balls in each collection are disjoint. This does not mean $G_i \cap G_j = \varnothing$) in $\ms{F}$ such that
\[
A \subseteq \cup_{i=1}^{N_n}(\cup_{B\in G_i}B)
\]

\proof

In the book

\qed 

\thm (More on filling open sets with Balls)

Let $\mu$ be a Borel measure on $\R^n$, and let $\ms{F}$ be any collection of nondegenerate closed balls. Let $A = \{x\mid \exists B_r(x) \in \ms{F}$ (again, the set of centers). 

Assume $\mu(A)<\oo$ (we do not assume $A$ is $\mu$-measurable) and $\inf\{r:B_r(a) \in \ms{f}\} = 0$ for any $a \in A$. 

Then for every open set $U \subseteq \R^n$, there exists a countable collection $G$ of disjoint balls in $\ms{F}$ such that
\begin{enumerate}

\item $\cup_{B\in G}B\subseteq U$ 

\item $\mu(A \cap U \setminus \cup_{B\in G}B) = 0$. 

\end{enumerate}

\proof

Consider the collection $\ms{F}_1 = \{B \mid B \in \ms{F}, B \subseteq U, \diam(B)\leq 1\}$. 

$A \cap U = \{x \mid \exists B_r(x) \in \ms{F}_1\}$. 

Apply the theorem to $\ms{F}_1$. Then there exist $G_1, G_2, \dots, G_{N_n}$ countable collections of disjoint balls (each) in $\ms{F}_1$ such that
\[
A \cap U \subseteq \cup_{i=1}^{N_n}(\cup_{B\in G_1}B)
\]
Then $\mu(A \cap U) \leq \sum_{i=1}^{N_n}\mu\left((A \cap U) \cap (\cup_{B\in G_i}B)\right)$. So there exists an index $k\in\{1, 2, \dots, N_n\}$ such that
\[
\mu\left((A \cap U) \cap (\cup_{B\in G_k}B)\right) \geq \frac{1}{N_n}\mu(A \cap U)
\]

Write $G_k = \{B_i\}\seq{i}$. Then
\[
\mu\left((A \cap U) \cap (\cupi B_i)\right) \geq \frac{1}{N_n}\mu(A \cap U)
\]
Let $q = 1 - \frac{1}{2N_n}$. 

$\mu(A \cap U) \leq \sup_{i=1}^{N_n}\mu\left((A \cap U) \cap (\cup_{B\in G_iB}\right)$. 

There exists an index $k \in \{1, 2, \dots, N_n\}$ such that
\[
\mu\left((A \cap U) \cap (\cup_{B\in G_k}G)\right) \geq \frac{1}{N_n}\mu(A \cap U)
\]

\section*{Lecture 8, 2/7/23}

We continue with the proof. Some review. 

$\mu$ is a Borel measure on $\R^n$. $\ms{F} = \{B\mid B \subseteq \R^n, B$-nondegenerate$\}$. $A = \{a\in\R^n\mid \exists B_r(a)\in\ms{F}\}$. Assume $\mu(A)<\oo$. Assume $\inf\{r\mid B_r(a)\in\ms{F}\} =0$ for all $a\in A$.

Then for all open $U\subseteq\R^n$, there exists $G = \{B_i\}\seq{i}\subseteq\ms{F}$ collection of disjoint balls such that $\mu(A \cap U \setminus \cupi B_i) = 0$, $\cupi B_i \subseteq U$.  

Let $\ms{F}_1 = \{B\in\ms{F}\mid B\subseteq U, \diam(B)\leq1\}$. 

New set of centers $= A \cap U$. By Besicovitch, there exist $N_n$ collections $G_1, G_2, \dots, G_{N_N}\subseteq\ms{F}_1$ such that $A \cap U \subseteq\cupi\cup_{B\in G_i}B$. 

Then $\mu(A \cap U) \leq \sum_{i=1}^{N_n}\mu(\cup_{B\in G_i}(B\cap A\cap U))$.

So there exists an index $k\in\{1, 2, \dots, N_n\}$ such that
\[
\mu((A\cap U) \cap (\cup_{B\in G_k}B))\geq\frac{1}{N_n}\mu(A \cap U)
\]
Let $\nu = \mu|_A$ - Borel.
Let $G_k = \{B_i\}\seq{i}$. Then $\mu((A \cap U) \cap (\cupi B_i)) = \nu(U \cap \cupi B_i) = \lim_{n\to\oo}\nu(U \cap \cupi[m]B_i)$. So there exists $m_1 \in \N$ such that 
\[
\nu(U\cap \cup_{i=1}^{m_1}B_i) = \mu((A\cap U)\cup_{i=1}^{m_1}B_i) \geq \frac{1}{2N_n}\mu(A \cap U)
\]
So 
\begin{align*}
\mu(A \cap U \setminus \cup_{i=1}^{m_i}B_i & = \mu(A \cap U) - \mu(A \cap U \cap \cup_{i=1}^{m_1}B_i) \\
& \leq \underbrace{(1 - \frac{1}{2N_n})}_{0<q<1}\mu(A \cap U) \\
\end{align*}

Let $U_2 = U \setminus \cup_{i=1}^{m_1}B_i$, $\ms{F}_2 = \{B\in\ms{F}_1 \mid B \subseteq U_2, \diam(B)\leq1\}$. 

\underline{$k$th step}

We have $B_{m_{n - 1} + 1}, \dots, B_{m_k}\in\ms{F}_k$ such that
\[
\mu(A \cap U \setminus \cup_{i=1}^{m_k}) \leq q^k\mu(A \cup U)
\]

Let $G = \{B_i\}$. 



\qed

\subsection*{\underline{Differentiation of Radon Measures}}

\defn

Let $\mu$ and $\nu$ be Radon measures on $\R^n$. Define, for any $x \in \R^n$, 
\underline{the upper derivative of $\mu$ with respect to $\nu$} by
\[
\bar{D}_{\mu}\nu(x) = \begin{cases} \limsup_{r\to0^+}\frac{\nu(B_r(x))}{\mu(B_r(x))} & \text{ if }\mu(B_r(x)) > 0, \forall r > 0 \\ +\oo & \mu(B_r(x)) = 0\text{ for some }r\\
\end{cases}
\]
The \underline{lower derivative} is defined similarly:
\[
\underline{D}_{\mu}\nu(x) = \begin{cases} \liminf_{r\to0^+}\frac{\nu(B_r(x))}{\mu(B_r(x))} & \text{ if }\mu(B_r(x)) > 0, \forall r > 0 \\ +\oo & \mu(B_r(x)) = 0\text{ for some }r\\
\end{cases}
\]

$\bar{D_\mu}\nu, \underline{D}_\mu\nu:\R^n\to[0,\oo]$. These are sometimes also called the \underline{upper/lower density}. 

We say that \underline{$\nu$ is differentiable with respect to $\mu$ at the point $x$} if $\bar{D}_\mu\nu(x) = \underline{D}_\mu\nu(x) < \oo$

This leads to several questions/goals: 

\begin{enumerate}

\item Study the set where $\underline{D}_\mu\nu(x) = \bar{D}_\mu\nu(x) < \oo$. Is it $\mu$-a.e.?

\item Do we have $\nu(B) = \int_BD_\mu\nu(x)\,d\mu$ for Borel sets $B \subseteq\R^n$?

\end{enumerate}

For question 1, the answer is yes, $\mu$-almost everywhere in $\R^n$. The answer to question 2 is also yes, subject to the additional condition $\nu<<\mu$. 

\lem

Let $\mu$ and $\nu$ be Radon measures on $\R^n$. Let $0 < \alpha < +\oo$. 

\begin{enumerate}[label=(\roman*)]

\item If $A \subseteq \{x\in\R^n\mid\underline{D}_\mu\nu(x)\leq\alpha\}$, then $\nu(A)\leq\alpha\mu(A)$

\item If $A \subseteq \{x\in\R^n\mid\bar{D}_\mu\nu(x)\geq\alpha\}$, then $\nu(A)\geq\alpha\mu(A)$. 

\end{enumerate}

\proof

We can assume without loss of generality that $\mu(\R^n),\nu(\R^n)<\oo$. This is because $\nu(A \cap B_R(0)) \leq \alpha\mu(A \cap B_R(0))$, and as $R\to+\oo$, the left converges to $\nu(A)$, and the right converges to $\alpha\mu(A)$, $B_R(0) = \{|x|<R\}$. 

Fix $\varepsilon>0$. 

\begin{enumerate}[label=(\roman*)]

\item Let $U$ be any open set such that $A \subseteq U$. Consider the collection of closed balls $\ms{F} = \{B_r(x) \subseteq U \mid x\in A,\nu(B_r(x)) \leq (\alpha + \varepsilon)\mu(B_r(x)), \diam(B)\leq1\}$

For any $a \in A,$ we have $\inf\{r\mid B_r(a)\in\ms{F} \} = 0$ because $\underline{D}_\mu\nu(x)\leq\alpha$. Set of centers $=A$. 

By the theorem we just proved, there exists a countable collection $G = \{B_i\}\seq{i}$ of disjoint balls in $\ms{F}$ such that
\[
\nu(A\cap U \setminus\cupi B_i) = 0
\]
So
\[
\nu(A)\leq\nu(\cupi B_i) + \nu(A\setminus \cupi B_i) = \nu(\cupi B_i)
\]
By disjointness, this is equal to 
\begin{align*}
\sum\seq{i}\nu(B_i) & \leq\sum\seq{i}(\alpha+\varepsilon)\mu(B_i) \\
& = (\alpha + \varepsilon)\mu(\cupi B_i) \\
& \leq (\alpha + \varepsilon)\mu(U) \\
\end{align*}

So $\nu(A)\leq(\alpha+\varepsilon)\mu(U)$ for all $\varepsilon > ), U \supseteq A$. 

In the limit as $\varepsilon\to0^+$, we have $\nu(A)\leq\alpha\mu(U)$ for all $U \supseteq A$. So $\nu(A) \leq \alpha\inf\{\mu(U)\mid A \subseteq U, U$-open$\} = \alpha\mu(A)$. 

This completes the proof of 1. Proof of 2 is similar? 

\qed

\end{enumerate}

\thm Let $\mu$ and $\nu$ be Radon measures on $\R^n$. Then

\begin{enumerate}[label=(\roman*)]

\item $\nu$ is differentiable with respect to $\mu$ almost everywhere in $\R^n$. 

\item $\underline{D}_\mu\nu(x) = \bar{D}_\mu\nu(x)<\oo$ $\mu$-almost everywhere in $\R^n$. 

\item $D_\mu\nu$ is $\mu$-measurable. 

\end{enumerate}

\proof

Let $I = \{x\in\R^n \mid \bar{D}_\mu\nu(x) = \oo\}$. 

With the lemma we have just proven, it is easy to see that $\mu(I) = 0$. 

Assume $\mu(\R^n),\nu(\R^n) < \oo$. Fix any $\alpha>0$. Then $I = \{x\in\R^n\mid\bar{D}_\mu\nu(x)\geq\alpha\}$. So $\nu(I)\geq\alpha\mu(I)$, so
\[
\mu(I)\leq\frac{1}{\alpha}\nu(I)\leq\frac{\nu(\R^n)}{\alpha}
\]
As let let $\alpha\uparrow+\oo$, we get $\mu(I)<\oo$. 

\section*{Lecture 9, 2/9/23}

\thm

Let $\mu$ and $\nu$ be Radon measures on $\R^n$. Then 

\begin{enumerate}

\item $D_\mu\nu(x)$ exists and is finite $\mu$-almost everywhere. 

\item $D_\mu\nu(x)$ is $\mu$-measurable. 

\end{enumerate}

\proof

Let $I = \{x \in \R^n \mid \bar{D}_\mu\nu(x) = \oo\}$. We know that $\mu(I) = 0$, so we just have to prove 2. Assume within the ball $B_R(0) = \{x\in\R^n\mid|x|<R\}, D_\mu\nu(x)$ exists $\mu$-almost everywhere in $B_R(x)$. 

Define $X_{INE}(R) = \{x\in B_R(0)\mid $either $D_\mu\nu(x) = \oo$ or $D_\mu\nu(x) $ DNE $\}$. 

Note $X_{INE}(R)\subseteq B_R(x), X_{INE}(1) \subseteq X_{INE}(2) \subseteq \cdots \subseteq X_{INE}(m)\subseteq\cdots$. 

Now $\mu(X_{INE}(m)) = 0$ for all $m \in \N$, thus $\mu(X_{INE}(\oo))\leq\sum_{m=1}^\oo\mu(X_{INE}(m)) = 0$. Since $X_{INE}$ is a null set, it is measurable.

Assume $\nu(\R^n), \mu(\R^n)<\oo$. Let $X_{NE} = \{x\in\R^n \mid D_\mu\nu(x)\, $DNE$\} \implies \mu(X_{NE}) = 0$.

For $0 < a < b < \oo$, define $J(a, b) = \{x\in\R^n\mid \underline{D}_\mu\nu(x) \leq a$ or $\bar{D}_{\mu}\nu(x)\geq b\}$. 

Then $X_{NE} \subseteq I \cup (\cup_{0<r,q<\oo}J(r, q))$. 
$\mu(X_{NE})\leq\mu(I) + \sum_{0<r, q< \oo}\mu(J(r, q))$. 

By a previous lemma, $\nu(J(a, b)) \leq a\mu(J(a, b)), \nu(J(a, b)) \geq b \mu(J(a, b))$. 

Since $a\mu(J(a, b)) \geq b\mu(J(a, b))$ for all $a < b$, $\mu(J(a, b)) = 0$ for all $0<a<b<\oo$. So $\mu(X_{NE}) = 0$. 

\qed ?

Idea: Express $D_\mu\nu(x)$ as the limit of a sequence of $\mu$-measurable functions. 

\claim

\[
D_\mu\nu(x) = 
\begin{cases} \lim_{r\to0^+}\frac{\nu(B_r(x))}{\mu(B_r(x))} & x \in \R^n\setminus N \\ \oo\text{ or not defined in }N & \text{ where }\mu(N) = 0 \\
\end{cases}
\]
\proof

For fixed $r > 0$, define $f_r(x):\R^n\to[0,\oo]$ by 
\[
f_r(x) = 
\begin{cases}
\frac{\nu(B_r(x))}{\mu(B_r(x))} & \text{ if }\mu(B_r(x))>0 \\
\oo & \text{ otherwise}. \\
\end{cases}
\]

\claim 

For any $r > 0$, the function $f_r(x)$ is $\mu$-measurable. 

\claim

For every $r > 0$, the function $g_r(x):\R^n\to[0,\oo)$ defined by $g_r(x) = \mu(B_r(x))$ is upper semicontinuous, i.e. if $x_k \to x$, then $\limsup g_r(x_k) \leq g_r(x)$. 

\proof of second claim

Let $x_k \to x$ be a convergent sequence. 

Define $\varphi(x) = \chi_{B_r(x)}$ and note $\limsup_{x_k\to x}\varphi(x_k) \leq\varphi(x)$. 

$\limsup_{x_k\to x}\chi_{B_r(x_k)}(y) \leq \chi_{B_r(x)}(y)$. 

If $y\in B_r(x)$ then $\chi_{B_r(x)}(y) = 1$, so we're done. 

If $y\not\in B_r(x)$ then $|x - y| = r + \delta$ where $\delta > 0$ so there exists $K \in \N$ such that for all $k \geq K$, $|x_k - x| < \delta \implies y\not\in B_r(x_k)$. 

So $\chi_{B_r(x_k)}(y) = 0 \implies \limsup_{x_k\to x}\chi_{B_r(x)}(y) = 0$. 

Note $\liminf_{x_k\to x}(1 - \chi_{B_r(x_k)}(y)) \geq 1 - \chi_{B_r(x)}(y)$. 

Thus $\int_{B_r(x)}\liminf_{x_k\to x}(1 - \chi_{B_r(x_k)}(y))\,d\mu(y) \geq \int_{B_r(x)}1 - \chi_{B_r(x)}(y)$

By Fatou's lemma, 
\[
\liminf_{k\to\oo}\int_{B_r(x)}(1 - \chi_{B_r(x_k)}(y))\,d\mu(y) \geq \int_{B_r(x)}(1 - \chi_{B_r(x)}(y))
\]

Thus $\liminf_{k\to\oo} f(\mu(B_{2r}(x))) - \mu(B_r(x_k)) \geq \mu(B_{2r}(x)) - \mu(B_r(x))$

So $\limsup_{k\to\oo}\mu(B_r(x_k)) \leq \mu(B_r(x))$

\qed

\proof of first claim

Denote $I_r = \{x\in\R^n \mid \mu(B_r(x)) = 0 \}$. 

$I_r \subseteq I$, $\mu(I) = 0$, so $\mu(I_r) = 0$ for all $r > 0$. 

Furthermore, if $0 < r_1 < r_2$, $I_{r_2} \subset I_{r_1} \subset I$. 

\qed

\claim

\[
D_\mu\nu(x) = 
\begin{cases}
\lim_{k\to\oo}\frac{\nu(B_{\frac{1}{k}}(x))}{\mu(B_{\frac{1}{k}}(x))} & \mu(B_{\frac{1}{k}}(x) > 0 \\ 
\oo & \mu(B_{\frac{1}{k}}(x)) = 0 \\
\end{cases}
\] 

If $D_\mu\nu(x)<\oo$ $\mu$-almost everywhere, then $D_\mu\nu(x)$ is $\mu$-measurable. 

\proof

\qed

\defn

Let $\mu$ and $\nu$ be Borel measures on $\R^n$. Then

\begin{enumerate}

\item $\nu$ is \underline{absolutely continuous with respect to $\mu$} if $\mu(A) = 0\implies\nu(A) = 0$ for all $A \subseteq \R^n$. We write $\nu<<\mu$. 

\item $\mu$ and $\nu$ are \underline{mutually singular} if there exists a Borel set $B \subseteq \R^n$ such that $\nu(B) = \mu(\R^n\setminus B) = 0$. We write $\nu\perp\mu$. 

\end{enumerate}

\thm (Radon-Nikodym)

Let $\mu$ and $\nu$ be Radon measures on $\R^n$ such that $\nu<<\mu$. Then for any $\mu$-measurable set $A \subseteq \R^n$, one has

\begin{enumerate}

\item $\nu(A) = \int_AD_\mu\nu(x)\,d\mu$ 

\item For any $f:\R^n\to\R$ $\mu$-measurable, one has $\int_Af(x)\,d\nu = \int_Af(x)D_\mu\nu(x)\,d\mu$. 

\end{enumerate}

\proof

For the first part, observe that if $A$ is $\mu$-measurable, then $A$ is also $\nu$-measurable. Why? If there exists a Borel set $B\subseteq\R^n$ such that $A\subseteq B, \mu(A) = \mu(B)$, then $\mu(B\setminus A) = 0$. Since $\nu<<\mu$, $\nu(B\setminus A) = 0$, so $B \setminus A$ is $\nu$-measurable. So $A = B\setminus(B\setminus A)$. so $A$ is $\nu$-measurable. 

\cor

If $f$ is $\mu$-measurable, then it is $\nu$-measurable. 

\section*{Lecture 10, 2/21/23}

\claim

If $A \subseteq \R^n$ is $\mu$-measurable, then $A$ is also $\nu$-measurable. If $f:\R^n\to\R$ is $\mu$ measurable, then $f$ is also $\nu$-measurable. 

\proof\,

Fix $t > 1$ ($t$-truncation argument)

Consider the sets
\[
A_m = \{x\in A\mid t^m \leq D_\mu\nu(x) < t^{m + 1}\},\,\,m\in\Z
\]
Let $A = \cup_{m\in\Z}A_m \cup (A \cap \tilde{I}) \cup (A \cap \{x\mid D_\mu\nu(x) \text{ does not exist}\} \cup (A \cap Z)$, where 
\begin{align*}
\tilde{I} & \eqdef \{x\in\R^n\mid D_\mu\nu(x) = +\oo\} \\
Z & \eqdef \{x\in\R^n\mid D_\mu\nu(x) = 0\} \\
\end{align*}

Note $\nu((A \cap \tilde{I}) \cup (A \cap \{x\mid D_\mu\nu(x) DNE ) \cup (A \cap Z))= 0$, so $\nu(A) = \nu(\cup_{m\in\Z}A_m) = \sum_{m\in\Z}\nu(A_m)$.

Why is this set $\nu$-null? 

First, note $\mu(\{x\mid D_\mu\nu(x) DNE\}) = 0$ implies $\nu(\{x\mid D_\mu\nu(x) DNE\}) = 0$. 

$\tilde{I}\subseteq I$, and $\mu(I) = 0$, so $\mu(\tilde{I}) = 0$, so $\mu(\tilde{I} \cap A) = 0$, so $\nu(\tilde{I}\cap A) = 0$.

Fix $\alpha > 0$. $\nu(C_\alpha)\leq\alpha\mu(C_\alpha)$. So $\nu(Z)\leq\nu(C_\alpha)\leq\alpha\mu(C_\alpha)\leq\alpha\mu(\R^n)$, which goes to $0$. 

So $\nu(Z) = 0$. 

Thus $\nu(\tilde{I}) = \int_{\tilde{I}} D_\mu\nu(x)\,d\mu, \nu(Z) = \int_ZD_\mu\nu(x)\,d\mu = 0$.

Now $t^n\mu(A_m) \leq \nu(A_m) \leq t^{m + 1}\mu(A_)$. 

So $\sum_{m\in\Z}t^m\mu(A_m) \leq \sum_{m\in\Z}\nu(A_m) \leq \sum_{m\in\Z}t^{m + 1}\mu(A_m) =t \sum_{m\in\Z}t^m\mu(A_m)$. 


So $\int_{A_m}t^m\,d\mu \leq \int_{A_m}D_\mu\nu(x)\,d\mu \leq \int_{A_m}t^{m + 1}\,d\mu$. 

By monotone convergence theorem,$\sum_{m\in\Z}\int_{A_m}D_\mu\nu(x)\,d\mu = \int_{\cup A_m}D\mu\nu(x)\,d\mu$. 

So $\sum_{m\in\Z}t^n\mu(A_m) \leq \int_{\cup A_m}\Dmn(x)\,d\mu \leq t\sum_{m\in\Z}t^m\mu(A_m)$. 

So $\frac{1}{t}\int_{\cup A_m}\Dmn(A)\,d\mu \leq \nu(A) \leq t\int_{\cup A_m}\Dmn(x)\,d\mu, \forall t > 1$. 

So $\frac{1}{t}\int_A\Dmn(x)\,d\mu \leq \nu(A) \leq t \int_{A}\Dmn(x)\,d\mu$. 

We let $t\to 1$ to get $\nu(A) = \cup_A\Dmn(x)\,d\mu$. 

We now prove that $\int_Af(x)\,d\nu = \int_Af(x)\Dmn(x)\,d\mu$. 

Let $f^+(x) = \max(f(x), 0)$ and $f^-(x) = -\min(0, f(x))$. Note $f = f^+ - f^-$. 

Assume $f(x) \geq 0$ for all $x \in\R^n$. 

\lem 

Let $X$ be a nonempty set, and let $\mu$ be a measure on $X$, and let $f:X\to[0,\oo]$ be $\mu$-measurable. Then there exist $\mu$-measurable sets $\{A_k\}\seq{k}$ such that 
\[
f(x) = \sum_{k=1}^\oo \frac{1}{k}\chi_{A_k}(x)
\]

\rem The $\{A_k\}$ need not be disjoint.

\proof 

Define $A_1 = \{x\mid f(x)\geq1\}$. 

Assuming $A_1, A_2, \dots, A_{k - 1}$ are well-defined, define $A_k = \{x\in X \mid f(x) \geq \frac{1}{k} + \sum_{i=1}^{k - 1}\frac{1}{i}\chi_{A_i}(x)\}$

By construction/induction, each $A_k$ is $\mu$-measurable. 

We claim $f(x) = \sum_{k=1}^\oo\frac{1}{k}\chi_{A_k}(x)$. 

\begin{enumerate}

\item Fix $n\in\N$. We claim $f(x) \geq \sum_{k=1}^n\frac{1}{k}\chi_{A_k}(x)$. Let $m\in\{1,\dots,n\}$ be the largest integer such that $x \in A_m$. Then 
\[
f(x) \geq \frac{1}{m} + \sum_{k=1}^{m - 1}\frac{1}{k}\chi_{A_k}(x) = \sum_{k=1}^m\frac{1}{k}\chi_{A_k}(x) \leq \sum_{k=1}^n\frac{1}{k}\chi_{A_k}(x)
\]

\item If $f(x) = \oo$, $x \in A_k$ for all $k\in\N$ so $\sum_{k=1}^m\frac{1}{k}\chi_{A_k}(x) = \sum_{k=1}^\oo\frac{1}{k} = \oo$. If $f(x) < \oo$, then there exists a sequence of naturals $\{n_\ell\}$ such that $x\not\in A_{n_\ell}$. So 
\[
f(x) - \sum_{k=1}^{n_\ell - 1}\frac{1}{k}\chi_{A_k}(x) < \frac{1}{n_\ell}, \,\,\ell = 1, 2, \dots .
\]
We let $\ell\to\oo$ to get $f(x) = \sum_{k=1}^{\oo}\frac{1}{k}\chi_{A_k}(x)$

\end{enumerate}

We are now ready to finish the proof. 

Let $f_m = \sum_{k=1}^{m}\frac{1}{k}\chi_{A_k}(x)$, $A_k \subseteq A$. 

Then $\int_Af_m(x)\,d\nu = \int_Af_m(x)\Dmn(x)\,d\nu(x)$. 

By MCT, 
\[
\int_Af(x)\,d\mu = \int_Af(x)\Dmn(x)\,d\nu
\]

\qed

\thm (Lebesgue Decomposition Theorem)

Let $\mu$ and $\nu$ be Radon measures on $\R^n$. Then there exists Radon measures $\nu_{ac},\nu_s$ on $\R^n$ such that $\nu = \nu_{ac} + \nu_s$, and 
\begin{enumerate}

\item $\nu_{ac} << \mu, \nu_s \perp \mu$ 

\item $\Dmn_s(x) = 0$ for $\mu$-almost every $x \in \R^n$. Further, for all Borel $B \subseteq\R^n$, we have 
\[
\nu(B) = \int_B\Dmn_{ac}(x)\,d\mu + \nu_s(B)
\]

\end{enumerate}

\proof

Consider the collection $\ms{F} = \{A\subseteq \R^n\setminus A) = 0, A$ Borel $\}$. Let $m = \inf\{nu(A) \mid A \in \ms{F}\}$.





\section*{Lecture 11, 2/23/23}

\thm (Lebesgue Decomposition Theorem). 

Let $\mu$ and $\nu$ be Radon measures on $\R^n$. Then there exists $\nu_{ac}, \nu_s$ Radon measures on $\R^n$ such that $\nu = \nu_{ac} + \nu_s$, 
\begin{enumerate}

\item $\nu_{ac}<<\nu_{s}, \nu_s\perp\mu$

\item $\Dmn_s(x) = 0$ for $\mu$-almost every $x \in\R^n$, and for all Borel $B \subseteq \R^n$, 
\[
\nu(B) = \int_B\Dmn_{ac}(x)\,d\mu + \nu_s(B)
\]

\end{enumerate}

\proof

First, assume $\mu,\nu$ are finite. Consider the collection $\ms{F} = \{A\subseteq\R^n \mid \mu(\R^n\setminus A) = 0, A$ Borel$\}$. 

Let $m = \inf\{\nu(A)\mid A\in\ms{F}\}$. 

Note $0 \leq m < \oo$, so we can choose $A_k\in\ms{F}$ so that $m \leq \nu(A_k) < m + \frac{1}{k}$. 

Define $B = \capk A_k$. By continuity from above, $\nu(B) = m$. Furthermore, $B$ is Borel. Note $\mu(\R^n\setminus B) = \mu(\R^N\setminus \capk A_k) = \mu(\cupk(\R^n\setminus A_k)) \leq \sum\seq{k}\mu(\R^n\setminus A_k) = 0$. 

So $B \in \ms{F}$.

Note $\mu$ is supported on $B$. Define $\nu_{ac} = \nu|_B$, $\nu_s = \nu_{\R^n\setminus B}$. Obviously, $\nu_s\perp\nu_{ac}$. 

We claim $\nu_{ac}<<\mu$. Towards contradiction, suppose there is a $C \subseteq\R^n$ such that $\mu(C) = 0$ but $\nu_{ac}>0$.

Without loss of generality, assume $C$ is Borel. Further, assume $C\subseteq B$. Then $B \setminus C \in \ms{F}$, since $\mu(\R^n\setminus(B\setminus C)) \leq \mu(\R^n\setminus B) + \mu(C) = 0$. Furthermore, $\nu(B\setminus C) = \nu(B) - \nu_{ac}(C) < \nu(B)$, a contradiction.

Now we show $\Dmn_s(x) = 0$ for $\mu$-almost every $x \in \R^n$. 

For every $\alpha>0$, set $C_\alpha = \{x\in\R^n\mid \bar{\Dmn_s}(x) \geq\alpha\}$.

By lemma, $\nu_s(C_\alpha) \geq \alpha\mu(C_\alpha)$. Clearly, $C_\alpha = (C_\alpha \cap B) \cup (C_\alpha \setminus B)$. Note $\mu(C_\alpha\setminus B) = 0$, $\mu(C_\alpha \cap B) = 0$. So $\mu(C_\alpha) = 0$. 

Since $\{x\mid \Dmn_s(x) > 0\} = \cupk C_{\frac{1}{k}}$, $\mu(\{x\mid \Dmn_s(x)>0\}) = 0$.

Let $C_m = \{x\mid|x|<m\}$. There exists $B_m\subseteq C_m$ such that
\begin{itemize}
\item $\mu(C_m\setminus B_m) = 0$
\item $\nu_{ac}^m = \nu_{C_m} < < \mu$
\item $\nu_s^m = \nu_{C_m\setminus B_m}\perp\mu$
\item $\Dmn_{ac}^m(x) = 0$ for $\mu$-almost every $x \in C_m$
\end{itemize}
Set $B = \cupi B_i$, $\nu_{ac} = \nu_B, \nu_s = \nu_{\R^n\setminus B}$

\begin{enumerate}

\item $\mu(\R^n\setminus B) = \mu(\cupk C_k \setminus \cupk B_k) \leq \mu(\cupk(C_k\setminus B_k)) \leq \sum\seq{k}\mu(C_k\setminus B_k) = 0$

\item We claim $\nu_s\perp\mu$, $\nu_{ac}<<\mu$. If $\mu(A) = 0$, $\mu(A \cap C_m) = 0$ for all $m\in\N$. So $\nu_{ac}(A) = \nu_{ac}(\cupk C_k \cap A) \leq \sum\seq{k}\nu_{ac}(C_k\cap A) = \sum\seq{k}\nu(B\cap C_k \cap A)$.

Fix: New $\bar{B_m} = \cupk[m]B_k$. This will give us $ = \sum\seq{k}\nu_{ac}^k(C_k \cap A) = 0$.

\qed ?

\end{enumerate}

\subsection*{\underline{Differentiation of Integrals, Lebesgue's Point}}

\defn

\[
L^1(\R^n,\mu) \eqdef \{f:\R^n\to[-\oo,\oo] \mid f\text{ is }\mu-\text{measurable }, \int_{\R^n}|f|\,d\mu < \oo\}
\]
\[
L^1_{\loc}(\R^n, \mu) \eqdef \{f:\R^n\to[-\oo,\oo] \mid f\text{ is } \mu-\text{measurable }, \int_K|f|\,d\mu < \oo\text{ for all $K$ compact}\}
\]

\thm (Lebesgue-Besicovitch differentiation theorem)

Let $\mu$ be a Radon measure on $\R^n$ and let $f\in L^1_{\loc}(\R^n, \mu)$. Then 
\[
\lim_{r\to0^+}\frac{1}{\mu(B(x, r))}\int_{B(x, r)}f(y)\,dy = f(x)
\]
for $\mu$-almost every $x \in \R^n$.

\proof

Consider $f^+ = \max(f, 0)$ and $f^- = -\min(0, f)$. Note $f = f^+ - f^-$, $f^+$ and $f^-$ are $\mu$-measurable, $f^+, f^- \in L^1_{\loc}(\R^n,\mu)$. 

Define $\nu^+,\nu^-:2^{\R^n}\to[0,\oo]$ as follows: 
\begin{align*}
\nu^{\pm}(B) & \eqdef \int_Bf^{\pm}(x)\,d\mu, \text{ where $B$ is Borel}\} \\
\nu^{\pm}(A) & \eqdef \inf\{\nu^{\pm}(B) \mid A \subseteq B, B\text{ Borel}\} \\
\end{align*}

\lem 

$\nu^{pm}$ are Radon measures with 
\begin{itemize}

\item $\nu^{\pm} << \mu$
\item $\Dmn^{\pm}(x) = f^{\pm}(x)$ for $\mu$-almost every $x \in \R^n$.

\end{itemize}

\lem Let $g \in L^1_{\loc}(\R^n,\mu)$, where $\mu$ is a Borel measure on $\R^n$. Then
\begin{align*}
\nu(B) & = \int_Bg(x)\,d\mu, B\text{ Borel} \\
\nu(A) & = \inf\{\nu(B)\mid A \subseteq B, B \text{ Borel} \}\\
\end{align*}
is a Radon measure, $\nu<<\mu, \Dmn(x) = g(x)$ for $\mu$-almost every $x\in\R^n$. 

\section*{Lecture 12, 2/28/23}

\lem Let $\mu$ be a Borel measure on $\R^n$ and let $f:\R^n\to[0,\oo]$ be such that $f\in L^1_{\loc}(\R^n,\mu)$. Define $\nu(B) = \int_Bf\,d\mu$ for all $B$ Borel, $\nu(A) = \inf\{\nu(B)\mid A \subseteq B, B \text{ Borel}\}$. Then:
\begin{enumerate}

\item $\nu$ is a Radon measure on $\R^n$ (even if $\mu$ is just Borel)
\item If $\mu$ is Borel-regular, then $\nu<<\mu$ and $\nu(A) = \int_Af\,d\mu$ for all $A \subseteq\R^n$ $\mu$-measurable.
\item If $\mu$ is Radon, then $\Dmn(x) = f(x)$ for $\mu$-almost every $x \in \R^n$. 

\end{enumerate}

\proof\,

\begin{enumerate}

\item $\nu$ is a Borel-regular measure on $\R^n$, by homework 2, problem 2. 

\item Assume $A$ is $\mu$-measurable. 

Then there exists a Borel set $B$ such that $A \subseteq B, \mu(A) = \mu(B)$. So $\mu(B\setminus A) = \mu(B) - \mu(A) = 0$. 

$\nu(B) = \int_Bf\,d\mu = \int_Af\,d\mu + \int_{B\setminus A}f\,d\mu = \int_Af\,d\mu$

$\nu(A) = \inf\{\nu(\tilde{B})\mid A\subseteq\tilde{B}, \tilde{B} \text{ Borel}\}$. 

So $\int_{\tilde{B}}f\,d\mu = \int_{\tilde{B}\setminus A}f\,d\mu + \int_Af\,d\mu \geq \int_Af\,d\mu = \nu(B)$. So $\nu(B) = \inf\{``"\} = \nu(A)$. 

Assume $\mu(A) = 0$. Then $\nu(A) = \int_Af\,d\mu = 0$, so $\nu<<\mu$. 

\item We need a lemma


\end{enumerate}

\lem 

Let $X$ be a nonempty set, and let $\mu$ be a measure on $X$. Assume $f \in L^1_{\loc}(X,\mu)$. If $\int_Af(x)\,d\mu = 0$ for all $A$ $\mu$-measurable, then $f \equiv 0$ $\mu$-almost everywhere. 

\proof

Assume $f$ is nonnegative $\mu$-almost everywhere. Let $A_n = \{x\in X \mid f(x) \geq\frac{1}{n}\}$. For all $n$, $A_n$ is $\mu$-measurable. Note $\{x\in X \mid f(x) > 0 \} = \cupn A_n$. We have
\[
0 = \int_{A_n}f\,d\mu \geq \int_{A_n}\frac{1}{n}\,d\mu = \frac{1}{n}\mu(A_n) \geq 0
\]
So $\mu(A_n) = 0$. Thus $\mu(\{x\in X \mid f(x) > 0 \}) = 0$. 

Now consider any $f\in L^1_{\loc}(X,\mu)$. Let $f = f^+ - f^-$. Both $f^+, f^- = 0$ $\mu$-almost everywhere. 

\qed

We now proceed with the proof. For all $A \subseteq \R^n$ $\mu$-measurable and $K$ compact, we have $\int_{A\cap K}f\,d\mu = \nu(A\cap K)$, and by the Radon-Nikodym theorem this equals $\int_{A\cap K}\Dmn(x)\,d\mu$. Thus $\int_{A\cap K}\Dmn(x) - f(x)\,d\mu = 0$. So by lemma 2, $\Dmn(x) = f(x)$ $\mu$-almost everywhere. Take $K = B_1, B_2, \dots $. 

\thm (Lebesgue-Besicovitch Theorem):

Let $\mu$ be a Radon measure on $\R^n$ and let $f\in L^1_{\loc}(\R^n,\mu)$. Then $\lim_{r\to0^+}\frac{1}{\mu(B(x, r))}\int_{B(x, r)}f\,d\mu = f(x)$ $\mu$-almost everywhere.

\proof

Write $f = f^+ - f^-$ and note $f^{\pm}\in L^1_{\loc}(\R^n,\mu)$. Define $\nu^{\pm}(B) = \int_Bf^{\pm}\,d\mu$ for all $B$ Borel, and as in lemma 1, $\nu^{\pm}(A) = \inf\{\nu^{\pm}(B)\mid A \subseteq B, B$ Borel $\}$. 

Then $\nu^{\pm}<<\mu, \nu^{\pm}$ are Radon. Note that $\Dmn^{\pm}(x) = f^{\pm}(x)$ $\mu$-almost everywhere. So
\begin{align*}
\lim_{r\to0^+}\frac{1}{\mu(B(x, r))}\int_{B(x, r)}f\,d\mu & = \lim_{r\to0^+}\frac{1}{\mu(B(x, r))}\int_{B(x, r)}(f^+ - f^-)\,d\mu \\
& = \lim_{r\to0^+}\left(\nu^+(B(x, r)) - \nu^-(B(x, r))\right) \\
& = \lim_{r\to0^+}\frac{\nu^+(B(x, r))}{\mu(B(x, r))} - \lim_{r\to0^+}\frac{\nu^-(B(x, r))}{\mu(B(x, r))} \\
& = \Dmn^+(x) - \Dmn^-(x) \\
& = f^+(x) - f^-(x) \\
& = f(x) \mu-a.e. \\
\end{align*}

\qed

\thm Under the conditions of the previous theorem, we have 
\[
\lim_{r\to0^+}\frac{1}{\mu(B(x, r))}\int_{B(x, r)}|f(y) - f(x)|\,d\mu = 0
\]
$\mu$-almost everywhere. 

\proof

We have
\begin{align*}
|\frac{1}{\mu(B(x, r))}\int_{B(x, r)}f(y)\,d\mu - f(x)| & = |\frac{1}{\mu(B(x, r))}\int_{B(x, r)}f(y) - f(x)\,d\mu| \\
& \leq \frac{1}{\mu(B(x, r))}\int_{B(x, r)}|f(y) - f(x)|\,d\mu \\
& \to 0 \\
\end{align*}

And then the previous theorem finishes the proof. 

\section*{Lecture 13, 3/2/23}

\thm Let $\mu$ be a Radon measure on $\R^n$ and let $f \in L^1_{\loc}(\R^n,\mu)$. Then 
\[
\underbrace{\lim_{r\to0^+}\frac{1}{\mu(B(x, r))}\int_{B(x, r)}|f(y) - f(x)|\,d\mu}_{*} = 0
\]
for $\mu$-almost every $x \in \R^n$. 

\defn Points $x\in\R^n$ that satisfy the above equation are called 

\underline{Lebesgue points of $f$ (for the measure $\mu$)}

\proof

Let $\Q = \{r_i\}\seq{i}\subseteq\R$ be an enumeration of the rationals. We know $\Q$ is dense in $\R$. 

For every $i \in \N$, $\lim_{r\to0^+}\frac{1}{\mu(B(x, r))}\int_{B(x, r)}|f(y) - f(r_i)|\,d\mu = |f(x) - r_i|$ is satisfied $\mu$-almost everywhere. 

Let $A_i\subseteq \R^n$ be points where the equation is satisfied. Then $\mu(\R^n\setminus A_i) = 0$. 

We have 
\[
\phi_i(x) = |f(x) - r_i|\in L^1_{\loc}(\R^n,\mu)
\]
because 
\[
|\phi_i(x)| \leq |f(x) + |r_i| \in L^1_{\loc}(\R^n,\mu)
\]
Set $A = \capi A_i$. We claim that $*$ is satisfied for all $x \in A$. Note
\[
\mu(\R^n\setminus A) = \mu(\R^n\setminus\capi A_i) = \mu(\cupi (R^n\setminus A_i)) \leq \sum\seq{i}\mu(\R^n\setminus A_i) = 0
\]
Fix $\varepsilon > 0$. Choose $r_m$ such that $|f(x) - r_m| < \varepsilon$. 

Then 
\[
|f(y) - f(x)| \leq |f(y) - r_m| + |f(x) - r_m|
\]
so
\[
\frac{1}{\mu(B(x, r))}\int_{B(x, r)}|f(y) - f(x)|\,d\mu \leq 2|f(x) - r_m| < 2\varepsilon
\]
We take the limit as $\varepsilon\to0$ to get
\[
\limsup_{r\to0^+}\frac{1}{\mu(B(x, r))}\int_{B(x, r)}|f(y) - f(x)|\,d\mu = 0
\]
thus
\[
\lim_{r\to0^+}\frac{1}{\mu(B(x, r))}\int_{B(x, r)}|f(y) - f(x)|\,d\mu =0
\]

\qed

\defn 

A measure $\mu$ has \underline{the doubling property} if there exists a constant $C>0$, such that
\[
0< \mu(B(x, 2r)) \leq C\mu(B(x, r)) < \oo
\]

\rem

Note that $\mu = \lambda^n$ has the doubling property: there exists $c_1, c_2$ such that $\mu(B(x, c_1r)) \leq c_2\mu(B(x, r))$ for all $r > 0, x \in \R^n$. Simply choose $c_1 = 2, c_2 = 2^n$. Because $c_1 > 1$, there exists a $k\in\N$ such taht $c_1^k\geq2$, $\mu(B(x, 2r)) \leq \mu(B(x, c_1^kr)) \leq \cdots \leq c_1^k\mu(B(x, r))$

\thm Let $\mu$ be a Radon measure on $\R^n$ with the doubling property. Choose $f \in L^1_{\loc}(\R^n,\mu)$. Then $\lim_{r\to0^+}\frac{1}{\mu(B(x, r))}\int_{B(x, r)}|f(y) - f(x)|\,d\mu = 0$ for $\mu$-almost every $x \in \R^n$. 

\proof

\begin{align*}
\frac{1}{\mu(B(x, r))}\int_{B(z(r), r)}|f(y) - f(x)|\,d\mu & \leq \frac{1}{\mu(B(x, r))}\int_{B(x, 2r)}|f(y) - f(x)|\,d\mu \\
& \leq \frac{c_1^k}{\mu(B(z(r), 2r))}|f(y) - f(x)|\,d\mu \\
\end{align*}

This goes to 0 as $r\to 0$. 

\thm (Density)

Let $\mu$ be a Radon measure on $\R^n$ and let $E \subseteq \R^n$ be $\mu$-measurable. Then
\begin{enumerate}

\item $\lim_{r\to0^+}\frac{\mu(E \cap B(x, r))}{\mu(B(x, r))} = 1$ for $\mu$-almost every $x \in E$
\item $\lim_{r\to0^+}\frac{\mu(E \cap B(x, r))}{\mu(B(x, r))} = 0$ for $\mu$-almost every $x \in E^c$.

\end{enumerate}

\proof

Apply previous theorem to $f(x) = \chi_{E}(x)$.

Choose $R_1 < R_2 < \cdots < R_k < \cdots$ such that $\lim_{k\to\oo}R_k = \oo$, $\mu9\{x\mid||x| = R_i\}) = 0$. Then do the analysis for the sets $a-k = \{x\mid R_k < |x| < R_{k + 1}\}$, $A_0 = \{x\mid |x| < R\}$. Let $\nu_s = \sum\nu_s^i + \nu|_{\cupk\{x\mid |x| = R_k\}}, \nu_{ac} =\sum\nu_{ac}^i$. 

\qed

\lem

Let $\mu$ be a Radon measure on $\R^n$. Fix $x\in\R^n$. Then $\mu(\del(B(x, r)) = 0$ for $r \in \R^n\setminus\{r_1, r_2, \dots\}$. 

\proof

It is sufficient to prove the statement within any fixed ball $B(x, r), R> 0$. Then take $R = 1, 2, \dots, \R^n = \cup\seq{r}B(x, r)$.

Define $A_k = \{r\in[0,R]\mid \mu(\del B(x, r)) \geq \frac{1}{k}\}, k = 1, 2, \dots $. Note $\{r\in[0,R]\mid \mu(\del B(x, r)) > >0 \} = \cupk A_k$. 

$\oo > \mu(B(x, r)) \geq |A_k|\frac{1}{k}, |A_k| \leq k\mu(B(x, r))$

\section*{Lecture 14, 3/7/23}

\subsection*{\underline{Increasing, BV, AV, and Lipschitz Functions}}

\begin{itemize}

\item BV: Bounded Variation

\item AC: Absolutely Continuous 

\end{itemize}

Almost everywhere differentiability of nondecreasing functions. 

\thm Let $f:\R\to\R$ be increasing. Then $f'(x)$ exists and is finite $\lambda$-almost everywhere. Moreover, $f':\R\to\R$ is Lebesgue-measurable $(f'(x)\geq0)$ and 
\[
\int_af'(x)\,dx \leq f(b) - f(a) 
\]
for all $-\oo<b<a<\oo$. 

\lem Let $f:\R\to\R$ be increasing. Then the set of discontinuous points of $f$ is at most countable. 

\proof

Since $f$ is increasing, for all $x \in \R$, there exist $x+, x-$ such that $f(x+), f(x-)$ are finite, and 
\begin{itemize}

\item $f(x-) = \lim_{\varepsilon\to0^+}f(x - \varepsilon) \leq f(x)$ 

\item $f(x+) = \lim_{\varepsilon\to0^+}f(x + \varepsilon) \leq f(y)$ for all $y > x$. 

\item $f(x-) \leq f(x) \leq f(x+)$

\end{itemize}

$\lim_{y\to x}f(y)$ exists if $f(x-) = f(x+) = f(x)$. If $f(x+) > f(x-)$, then $f$ is discontinuous at $x$. The intervals $(f(x-), f(x+))$ are disjoint for all discontinuities $x$. Each contains a unique rational, giving an injection to a countable set. So there are at most countably many discontinuities. 

\qed

Given $f:\R\to\R$ increasing, define $g:\R\to\R$ as follows:
\[
g(x) = \begin{cases} f(x) & f\text{ continuous at }x \\ f(x+) & \text{otherwise} \end{cases}
\]
In other words, $g(x) = f(x+)$. 

\lem Let $f:\R\to\R$ be increasing. Let $g(x) = f(x+)$ for all $x \in \R$. Then, if $g'(x)$ exists, then $f'(x)$ exists, and $f'(x) = g'(x)$. 

\proof:

Homework assignment

\qed

\thm Let $f:\R\to\R$ be increasing and right-continuous. Then $f'(x)$ exists and is finite $\lambda$-almost everywhere. Moreover, $f:\R\to\R$ is Lebesgue-measurable $(f'(x)\geq0)$ and 
\[
\int_a^bf'(x)\,dx \leq f(b) - f(a)
\]
for all $-\oo<b<a<\oo$.

\proof

$\nu(A) = \inf\{\sum\seq{i}(f(b_i) - f(a_i)): A \subseteq \cupi(a_i, b_i]\}$ is a Radon measure on $\R$. Furthermore, $\nu(a, b] = f(b) - f(a)$ for all $a < b$, so $\nu(\{a\}) = f(a) - f(a')$ for all $a \in \R$.

By the Lebesgue Decomposition Theorem, $\nu = \nu_{ac} + \nu_s$, where 
\begin{itemize}

\item $\nu_{ac}<<\lambda, \nu_s\perp\lambda$ 

\item $D_{\lambda}\nu_s(x) = 0$ $\lambda$-almost everywhere in $\R$. 

\item $D_{\lambda}\nu_{ac}(x)$ and $D_{\lambda}\nu_s(x)$ are nonnegative, $\lambda$-measurable. 

\item $\nu_{ac} = \int_AD_\lambda\nu_{ac}(x)\,d\lambda = \int_AD_\lambda\nu_{ac}(x)\,dx$ for all $\lambda$-measurable $A \subseteq \R$. 

\end{itemize}

I.e. there exists $B \subseteq \R$ Borel such that $\nu_{ac} = \nu|_B, \lambda(\R\setminus B) = 0, \nu_s = \nu|_{(\R\setminus B)}$

Now we need a proposition (PROOF NOT COMPLETE). 

\prop Let $A \subseteq \R$ be the set of all points $x \in \R$, where 
\begin{itemize}

\item $f(x+) = f(x-)$ 

\item There exists $D_\lambda\nu_{ac}(x)<\oo, D_\lambda\nu_s(x) = 0$

\end{itemize}

Then $f'(x)$ exists and equals $D_\lambda\nu_{ac}(x)$. 

\proof

By definition, we have
\begin{align*}
f'(x+) & = \lim_{h\to0^+}\frac{f(x + h) - f(x)}{h} \\ 
&  = \lim_{h\to0^+} \frac{\nu((x, x + h])}{h} \\ 
& = \lim_{h\to0^+} \frac{\nu_{ac}((x, x + h]) + \nu_s((x, x + h])}{h} \\
& = \lim_{h\to0^+}\frac{\nu_{ac}((x, x + h])}{h} + \lim_{h\to0^+}\frac{\nu_s((x, x + h])}{h} \\
\end{align*}
Observe 
\[
0\leq\lim_{h\to0^+}\frac{\nu_s((x, x + h])}{h} \leq 2\lim_{h\to0^+}\frac{\nu_s((x - h, x + h])}{h} = 2D_\lambda\nu_s(x) = 0
\]
So $\lim_{h\to0^+}\frac{\nu_s((x, x + h])}{h} = 0$. Further, 
\[
\lim_{h\to0^+}\frac{\nu_{ac}((x, x + h])}{h} = \lim_{h\to0^+}\frac{\int_{(x, x + h]}D_\lambda\nu_{ac}(t)\,dt}{h} = \lim_{h\to0^+}\frac{\int_{[x, x + h]}D_\lambda\nu_{ac}(t)\,dt}{h} = D_\lambda\nu_{ac}(x)
\]
So $f'(x+) = D_\lambda\nu_{ac}(x) = 0$. A similar computation shows that $f(x-) = D_\lambda\nu_{ac}(x)$.

\qed

We proceed to the proof of the theorem. 

\proof

We now need to show $\int_a^bf'(x)\,dx \leq f(b) - f(a)$. Indeed, 
\begin{align*}
\int_a^bf'(x)\,dx & = \int_a^bD_\lambda\nu_{ac}(x))\,dx \\ 
& = \nu_{ac}((a, b]) \\
& \leq \nu((a, b]) \\
& = f(b) - f(a) \\
\end{align*}

\qed

We now show the theorem from the beginning. 

\proof

Let $g(x) = f(x+)$. Note that $\int_a^bf'(x)\,dx = \int_a^bg'(x)$ by lemma, and by previous theorem this is less than or equal to $g(b) - g(a)$ for all $a < b$. 

Furthermore, $g(a) = f(a+) \geq f(a)$. Choose $(b_n)$ such that $g(b_n) = f(b_n)$, and $b_n\uparrow b$. 

$\int_a^{b_n}f'(x)\,dx = \int_a^{b_n}g'(x)\,dx \leq g(b_n) - g(a) \leq f(b_n) - f(a) \leq f(b) - f(a)$

\qed

\exm 

Here are some examples of strict inequality: $f(x) = \begin{cases} 0 & x < 1 \\ 1 & x \geq 1 \end{cases} = \chi_{[1,\oo)}$
$f'(x) = 0$ for all $x \in (0, 1)$, and $f(1) - f(0) = 1$. 

We don't even need a jump point, cf the Devil's staircase. 

\section*{Lecture 15, 3/9/23}

\subsection*{\underline{BV (Bounded Variation) Functions}}

\defn

Let $a, b \in \R, a < b$. A \underline{partition} of $[a, b]$ is a finite sequence 
\[
P:a = x_0 < x_1 < \cdots < x_n = b
\]
For $f:[a,b]\to\R$, define 
\[
\vee(f, P) = \sum_{i=1}^{n - 1}|f(x_{i + 1}) - f(x_i)|
\]
and call it the variation of $f$ at $P$. Define
\[
\vee_a^b(f) = \sup_P\vee(f, P)
\]
and call it the total variation of $f$ over $[a, b]$. The function $f$ is said to be of bounded variation in $[a, b]$ if $\nu_a^b(f)<\oo$. The family of all such functions is denoted by $BV[a, b]$. 

\thm 

Assume $f, g \in BV[a, b]$. Then 
\begin{enumerate}

\item $f\pm g \in BV[a, b]$ and $\vee_a^b(f\pm g) = \vee_a^bf\pm\vee_a^bg$

\item For all $c \in \R$, $cf \in BV[a,b]$ and $\vee_a^b(cf) = |c|\vee_a^b(f)$

\item $f$ is bounded in $[a, b]$ 

\item If $|f(x)|, |g(x)| < M$ for all $x\in[a, b]$, then $\vee_a^b(fg) \leq M(\vee_a^b(f) + \vee_a^b(g))$ 

\item $\vee_a^b(f) \geq |f(b) - f(a)|$

\item $f \in BV[a, b] \iff f \in BV[a,c], f \in BV[c, b]$ with $\vee_a^c(f) + \vee_c^b(f) = \vee_a^b(f)$ for all $c\in[a, b]$. 

\item The function $v(x) = \vee_a^x(f):[a, b]\to \R$ is increasing and $v(x) \geq |f(x) - f(a)|$

\end{enumerate}

\proof

\begin{enumerate}

\item Triangle inequality

\item By definition 

\item For all $x \in [a, b]$, $|f(x) - f(a)| + |f(x) - f(b)| \leq \vee_a^b(f)$ so $|f(x)| \leq |f(a)| + \vee_a^b(f)<\oo$

\item For all $P:a=x_0 < x_1 < \cdots < x_n = b$, 
\begin{align*}
\vee_a^b(fg) & = \sum_{i=0}^{n - 1}|f(x_{i + 1})g(x_{i + 1}) - f(x_i)g(x_i)| \\ 
& = \sum_{i=0}^{n - 1}|g(x_{i + 1})(f(x_{i + 1}) - f(x_i)) + f(x_i)(g(x_{i + 1}) - g(x_i))| \\
& \leq \sum_{i=0}^{n - 1}|g(x_{i + 1})(f(x_{i + 1}) - f(x_i))| + \sum_{i=0}^{n - 1}|f(x_i)(g(x_{i + 1}) - g(x_i))| \\
& \leq M\sum_{i=0}^{n - 1}|f(x_{i + 1}) - f(x_i)| + M \sum_{i=0}^{n - 1}|g(x_{i + 1}) - g(x_i)| \\
& = M\vee_a^b(f) + M\vee_a^b(g)
\end{align*}

\item Triangle Inequality

\item Given a partition $P$ of $[a, b]$, by adding one (thus finitely many) points to $P$, the variational sum $\vee(f, P)$ may only increase. 

So, assume $f\in BV[a, b]$. We prove that $\vee_a^c(f) + \vee_c^b(f) = \vee_a^b(f)$ for all $c \in (a, b)$. 

First, observe that $f\in BV[a, c]$ and $f \in BV[c, b]$. $\vee_a^c(f) + \vee_c^b(f) \leq \vee_a^b(f)$. 

Fix $\varepsilon>0$. There exist $P:a = x_0 < x_1 < \cdots < x_n = b$ such that $\vee(f, p) > \vee_a^b(f)-\varepsilon$. 

$\vee_a^c(f) + \vee_c^b(f) \geq \vee(f, P_1) + \vee(f, P_2) \geq \vee_a^b(f, P) > \vee_a^b(f)-\varepsilon$

We take the limit as $\varepsilon\to0$ to get the desired result. 

For the opposite inequality, choose $P_1$ for $[a, c]$ and $P_2$ for $[c, b]$ such that 
\[
\vee(f, P_1) > \vee_a^c(f)-\varepsilon, \vee(f, P_2) > \vee_a^b(f)-\varepsilon
\]
Then $\vee_a^b(f) \geq \vee(f, P_1 \cup P_2) = \vee(f, P_1) + \vee(f, P_2) > \vee_a^c(f) + \vee_c^b(f) - 2\varepsilon$. 

\item If $x < y$, $x, y \in [a, b]$, $v(y) = \vee_a^y(f) = \vee_a^x(f) + \vee_x^y(f) \geq \vee_a^x(f) = v(x)$. $v(y) - f(x) = \vee_x^y(f) \geq |f(y) - f(x)|$.

\end{enumerate}

\thm 

For any $f \in\BV[a, b]$, there exists $f_1, f_2:[a,b]\to\R$ increasing such that $f = f_1 - f_2$

\rem

Denote $\Monp[a, b] = \{f:[a,b]\to\R\mid f$ increasing in $[a, b]\}$. 

$\BV[a,b]\subseteq\Monp[a,b]-\Monp[a, b]$
If $f \in \Monp[a,b]$, then $\vee_a^b(f) = f(b) - f(a)$.

So $\Monp[a,b]\subseteq\BV[a,b]$. 

On the other hand, we have $\Monp[a,b]-\Monp[a,b]\subseteq\BV[a,b]$. In sum, we have $\BV[a,b] = \Monp[a,b] - \Monp[a, b]$

\proof

$f(x) = v(x) - (v(x) - f(x))$ for all $x \in [a, b]$. 

Define $f_1 = v, f_2 = v - f$. If $x < y$, $v(y) - f(y) \geq v(x) - f(x), v(y) - v(x) \geq f(y) - f(x)$.

\qed

\thm For any $f \in \BV[a,b]$, $f'(x)$ exists and is finite almost everywhere in $[a,b]$. Moreover, $f'(x)$ is Lebesgue-measurable.

\proof

$f = f_1 - f_2$, both increasing on $[a,b]$. We have $f_1'(x)$ and $f_2'(x)$ exists and are finite almost everywhere in $[a,b]$ and are Lebesgue-measurable. Thsu $f'(x) = f_1'(x) - f_2'(x)$ almost everywhere on $[a,b]$.

\qed

\subsection*{\underline{Absolutely Continuous Functions: AC[a,b]}}

\defn

Let $f:[a,b]\to\R$. Then $f$ is \underline{absolutely continuous} on $[a, b]$ if, for all $\varepsilon>0$, there exists $\delta>0$ such that for all $\{(a_i, b_i)\}\seq{i} \subseteq [a, b]$ disjoint with $\lambda(\cupi(a_i,b_i))<\delta$, $\sum\seq{i}|f(b_i) - f(a_i)| < \varepsilon$. 

We write $f \in \AC[a,b]$

\exm

\begin{enumerate}

\item $f(x) = cx$

\item $f(x) \in C^1[a,b]$

\item $\Lip[a,b] = \{f:[a,b]\to\R \mid \exists M$ such that $|f(x) - f(y)| \leq M|x - y|$ for all $x, y \in [a, b]\}$. 

\end{enumerate}

If $f\in C^1[a,b]$ then $|f'(x)|\leq M$ for all $x \in [a, b]$. 

By MVT, $|\frac{f(y) - f(x)}{y - x}| = |f'(z)| \leq M$

\section*{Lecture 16, 3/14/23}

\rem If we take just one interval in the definition of absolute continuity, then we reproduce uniform continuity. 

Class 1: Lipschitz functions on $[a, b]$ are absolutely continuous  on $[a, b]$. 

Class 2: Continuously differentiable functions on $[a, b]$, called $C^1[a,b]$. 

\prop 

$C^1[a, b]\subseteq \Lip[a,b]\subseteq \AC[a,b]$

\proof\,

\subsection*{$C^1[a,b]\subseteq\Lip[a,b]$}

Choose $f \in C^1[a,b]$. Since $f'(x)\in C[a,b]$, $f'(x)$ is bounded by Weierstrass Theorem. 

By MVT, for all $x, y \in [a, b]$, $\frac{f(x) - f(y)}{x - y} = f'(z)$, for some $z \in [x, y]$. 

Because $|f'(z)| \leq M$, $|f(x) - f(y)| \leq M|y - x|$

So $f$ is Lipschitz on $[a, b]$.

\subsection*{$\Lip[a,b]\subseteq\AC[a,b]$}

Choose $f \in \Lip[a,b]$. Then there exists $M$ such that for all $x, y \in[a, b]$, $|f(x) - f(y)| \leq M|x - y|$ for all $x, y \in [a,b]$. 

Let $\{(a_i, b_i)\}\seq{i} \subseteq[a,b]$ be disjoint intervals. 

Then $\sum\seq{i}|f(b_i) - f(a_i)| \leq \sum\seq{i}M|b_i - a_i| = M \sum\seq{i}|b_i - a_i|$

So we can choose in the definition of absolutely continuous $\delta = \frac{\varepsilon}{M}$. 

So $\sum\seq{i}(b_i - a_i) < \delta = \frac{\varepsilon}{M} \implies \sum\seq{i}|f(b_i) - f(a_i)| < \varepsilon$. 

\qed

Our goal is to prove the following theorem:
\thm

Absolutely continuous functions are almost everywhere differentiable. Namely, if $f \in \AC[a,b]$, then $f'(x)$ and is finite almost everywhere in $[a, b]$. Moreover, 
\[
\int_a^bf'(x)\,dx = f(b) - f(a)
\]
where $f'(x)$ is Lebesgue-integrable in $[a, b]$.

\rem The fundamental theorem of calculus therefore holds for absolutely continuous functions. 

\proof 

\prop $\AC[a,b]\subseteq\BV[a,b]$

\proof

Chose $f \in \AC[a,b]$.

Then, choosing $\varepsilon = 1$ in the definition, there exists $\delta>0$ such that for all $\{(a_i,b_i)\}\seq{i}$ disjoint, $\sum\seq{i}(b_i - a_i) < \delta$, then $\sum\seq{i}|f(b_i) - f(a_i)| < 1$. 

We divide $[a, b]$ into $N$ equal parts, where $N$ is an integer large enough that $\frac{b - a}{N} < \delta$. Let the dividing points be $y_1, \dots, y_n$.

Let $P:a = x_0 < x_1 \cdots < x_n = b$ be any partitions. Then
\begin{align*}
\vee(f, P) & \leq \vee(f, P \cup \{y_1, \dots, y_n\}) \\
			  & = \vee_a^{y_1}(f, P_1) + \vee_{y_1}^{y_2}(f, P_1) + \cdots + \vee_{y_{N - 1}}^{y_N}(f, P_N) \\ 
			  & < 1 + 1 + \cdots + 1 \\
			  & = N \\
\end{align*}

So $\vee_a^b(f) = \sup_{P}\vee(f, P) \leq N < \oo$. Thus $f \in \BV[a, b]$. 

This completes the proof of the claim. 

\cor $f \in \AC[a,b]$ implies $f'(x)$ exists and is finite almost everywhere in $[a, b]$. 

\qed

Question: Why do we have $\int_a^bf'(x)\,dx = f(b) - f(a)$? 

\prop If $f \in \AC[a,b]$, then the function $v(x) = \vee_a^x(f)$ is continuous in $[a, b]$. 

\proof

We evaluate $|v(x) - v(y)|$.

Assume $a \leq x \leq y \leq b$. 

Then $|v(x) - v(y)| = v(y) - v(x) = \vee_a^y(f) - \vee_a^x(f) = \vee_x^y(f) = \sup_{P}\vee_x^y(f, P)$. 

Because $f \in \AC[a,b]$, let $\varepsilon>0$ be fixed and $\delta>0$ the corresponding $\delta$. If $|x - y| = y - x < \delta$, then $\vee_x^y(f, P) = \sum_{i=1}^n|f(x_i) - f(x_{i + 1})|<\varepsilon$

Thus $\vee_x^y(f) = \sup_P\vee_x^y(f, P) < \varepsilon$

Thus $v(x) = \vee_a^x(f)$ is continuous. 

Now suppose $\{(a_i, b_i)\}\seq{i} \subseteq[a,b]$ such that $\sum\seq{i}(b_i - a_i) < \delta$, then 
\[
\sum\seq{i}|v(b_i) - v(a_i)| = \sum\seq{i}\vee_{a_i}^{b_i}(f) < \varepsilon
\]
(follow the same proof)

\prop\,
\begin{enumerate}

\item $\AC[a,b]\pm\AC[a,b] \subseteq \AC[a,b]$

\item $c\AC[a,b] = \AC[a, b], c \neq 0$ 

\item $\AC[a,b] \cdot \AC[a,b] \subseteq \AC[a,b]$

\end{enumerate}

\proof

David skipped it

\rem If $f \in \AC[a,b]$, then $f \in C[a, b]$. 

Assume now that $f$ is absolutely continuous. Then $f \in \BV[a,b]$, $f(x) = f(x) - (v(x) - f(x)), x \in [a, b]$ where $v(x) = \vee_a^x(f)$ for all $x \in [a,b]$.

We know that $v(x)$ and $v(x) - f(x)$ are monotone increasing on $[a, b]$. 

By previous proposition, we know that $v \in \AC[a,b]$.

Because $f \in \AC[a,b]$, $v - f \in \AC[a,b]$. 

Let $f = f_1 - f_2$, where $f_1$ and $f_2$ are absolutely continuous and increasing. So it remains to prove the FTOC for increasing $\AC$ functions.

\section*{Lecture 17, 3/16/23}

\thm

Increasing absolutely continuous functions are almost everywhere differentiable. Namely, if $f \in \AC[a,b]$, then $f'(x)$ exists and is finite almost everyhwere in $[a, b]$. Moreover, 
\[
\int_a^bf'(x)\,dx = f(b) - f(a)
\]
where $f'(x)$ is Lebesgue-integrable in $[a, b]$. 

\proof

Assume $f:[a,b]\to\R$ is increasing and absolutely continuous. So $f \in \BV[a,b]$ and thus $f'(x)$ exists and is finite $\mu$-almost everywhere. We need to show $\int_a^bf'(x)\,dx = f(b) - f(a)$.

Claim: The Radon measure $\nu$ generated by $f$ is absolutely continuous with respect to the Lebesgue Measure.

Recall:

We extend $f$ to $\R$ as
\[
f(x) = \begin{cases} f(x) & x\in[a,b] \\ f(a) & x < a \\ f(b) & x > b \\ \end{cases}
\]

Of course, since $f$ is absolutely continuous, it is right continuous. Thus $\nu$ is a Radon measure, where
\[
\nu(A) = \inf\{\sum\seq{i}(f(b_i) - f(a_i)) \mid A \subseteq \bigcup\seq{i}[a_i,b_i), [a_i, b_i)\text{ disjoint}\}
\]


For all $[c, d]\subseteq\R$, $\nu([c, d]) = f(d) - f(c)$.

Assume $A \subseteq \R$ has zero Lebesgue measure. 

We want to show $\nu(A) = 0$. 

We have $\lambda(a)= \inf\{\sum\seq{i}(b_i - a_i) \mid A\subseteq \cup\seq{i}[a_i, b_i), [a_i, b_i)$ disjoint $\}$

Because $f$ is absolutely continuous in $[a, b]$, and we extended it as constant to the left and right of $[a,b]$, then $f$ is also in $\AC$ in $\R$. 

For all $\varepsilon>0$, there exists $\delta>0$ such that if $\sum\seq{i}(b_i - a_i)<\delta$, then $\sum\seq{i}|f(b_i) - f(a_i)| < \varepsilon$, where $[a_i, b_i)\subseteq\R$ are disjoint. 

Fix $\varepsilon>0$, and choose $\delta$ as above. 

Since $\lambda(A) = 0$, there exists a cover of $A$, $\{[a_i, b_i)\}$ such that $\sum\seq{i}(b_i - a_i)<\delta$. Then by choice of $\delta$, we have $\nu(A)\leq\sum\seq{i}(f(b_i) - f(a_i)) < \varepsilon$.

We take the limit as $\varepsilon\to0$ to conclude that $\nu(A) = $. So $\nu<<\lambda$

By the Radon-Nikodyn theorem, $\nu[a, b] = \int_a^bD_{\lambda}\nu(x)\,dx = \int_a^bf'(x)\,dx$, since $f'(x) = D_{\lambda}\nu_{ac}(x) = D_{\lambda}\nu(x)$ for almost every $x \in [a, b]$. 

However, $\nu[a,b] = nu[a,b) + \nu\{b\} = f(b) - f(a) + 0 \implies f(b) - f(a) = \int_a^bf'(x)\,dx$

Summary:

Let $f \in \AC[a,b]$. Then $f \in \BV[a,b]$. So $f = f_1 - f_2$, where $f_1, f_2 \in \BV[a,b]$, and are increasing in $[a,b]$. Morevoer, $f_1 - f_2 \in \AC[a,b]$. We proved that
\[
\int_a^bf_1'(x)\,dx = f_1(b) - f_1(a), \int_a^b f_2'(x)\,dx = f_2(b) - f_2(a)
\]
where $f_1', f_2'$ exist and are finite for almost every $x\in[a,b]$. 

Thus 
\[
\int_a^bf'(x)\,dx = \int_a^bf_1'(x)\,dx - \int_a^bf_2'(x)\,dx = f_1(b) - f_1(a) - (f_2(b) - f_2(a)) = f(b) - f(a)
\]

\qed

\thm

Let $f \in \AC[a,b]$. Then $f'(x)$ exists and is finite almost everywhere in $[a, b]$. Moreover, $f'(x)$ is summable in $[a, b]$ (not only integrable) and the FTOC holds, i.e. $\int_a^bf'(x)\,dx = f(b) - f(a)$.

\thm (Characterization/integral representation of $\AC$ functions): A function $f:[a,b]\to\R$ is absolutely continuous in $[a,b]$ iff there exists $g(x) \in L^1[a,b]$ such that $f(x) = f(a) + \int_a^xg(t)\,dt$ for all $x \in [a, b]$

\rem

In that case, we have $g(t) = f'(t)$ for almost every $t \in [a,b]$.

\proof

\subsection*{$=>$}

Assume $f \in \AC[a,b]$. Then we know $f'(x)\in L^1[a,b]. $

As $f \in \AC[a,x]$ for all $x\in[a,b[$, by  Newton-Leibniz we have $int_a^xf'(t)\,dt = f(x) - f(a)$ for all $x \in [a, b]$. 

Chose $g(t) = f'(t) \in L^1[a, b]$. 

\subsection*{$<=$}

Now assume there exists $g(t) \in L^1[a,b]$ and that $f(x) = f(a) + \int+a^xg(t)\,dt$. 

We need to show that $f \in \AC[a,b]$, and $f'(t) = g(t)$ for almost every $x \in [a, b]$.

$f \in [a, b]$: Follows from absolute continuity of the integral. 

Choose $\varepsilon>0$. There exists $\delta>0$ such that if $A \subseteq [a, b]$ is Lebesgue-measurable, $\lambda(A)<\delta$, then $\int_A|g(t)|\,dt < \varepsilon$. 

If now $\{[a_i, b_i)\} \subseteq [a, b]$ are disjoint with $\sum\seq{i}(b_i-a_i)<\delta$, then 
\begin{align*}
\sum\seq{i}|f(b_i) - f(a_i)| & = \sum\seq{i}|\int_a^{b_i}g(t)\,dt - \int_a^{a_i}g(t)\,dt| \\
& = \sum\seq{i}|\int_{a_i}^{b_i}g(t)\,dt| \\
& \leq \sum\seq{i}\int_{a_i}^{b_i}|g(t)\,dt \\
& = \int_{{\cupi\,[a_i,b_i)}}|g(t)\,dt < \varepsilon \\
\end{align*}
Thus $f \in \AC[a,b]$. 

$f'(x) = g(x)$ for almost every $x \in [a,b]$: Because $g\in L^1[a,b]$, then $x$ is a Lebesgue point of $g$ for almost every $x \in [a, b]$. 

\defn

A \underline{Lebesgue point} is a point $x$ such that 
\[
\lim_{r\to0^+}\frac{1}{r}\int_x^{x + r}g(t)\,dt = g(x), \,\lim_{r\to0^+}\frac{1}{r}\int_{x - r}^xg(t)\,dt = g(x)
\]
Thus $\lim_{r\to0^+}\frac{1}{r}\int_x^{x + r}g(t)|,dt = \lim_{r\to0^+}\frac{1}{r}((f(x + r)) - f(x)) = f'(x+) = g(x)$

Similarly, $\lim_{r\to0^+}\frac{1}{r}\int_{x-r}^xg(t)\,dt = f'(x-) = g(x)$. 

Thus $f'(x+) = f'(x-) = g(x)$ for almost every $x \in [a,b]$. So $f'(x) = g(x)$ for almost every $x \in [a, b]$. 

\qed

\exm A function that is increasing, BV, but not AC

Cantor's function.

The classical "middle thirds" Cantor set is constructed as follows: 
\begin{enumerate}

\item Divide $[0,1]$ into 3 equal parts, remove middle open interval. 

\item Divide remaining intervals into 3 equal parts, remove middle open intervals, 
\item Continue infinitely many times.

\end{enumerate}

The remaining part of $[0,1]$ is the Cantor set. This set has the following properties:

\begin{enumerate}

\item $\lambda(C) = 0$

\item $C \subseteq [0, 1]$ is compact, because $C = [0,1]\setminus\cupi(a_i, b_i)$, where $(a_i,b_i) \subseteq [0, 1]$ are disjoint open intervals. 

\item $C$ is nonempty. In particular, $\frac{1}{3}\in C$.

\end{enumerate}

Cantor function: From the construction of the cantor set for all $x \in C$, $x$ has the form $x = \sum\seq{n}\frac{2a_n}{3^n}$, where $a_n \in \{0,1\}$. Every number $x \in [0, 1]$ can be written in base 3 as $x = \sum\seq{k}\frac{b_k}{3^k}, b_k \in \{0,1,2\}.$

This representation in base 3 is unique. 

The points $x$ that belong to the Cantor set have representation in base 3 coefficients 1 or 2.

\[
C(x) = \begin{cases} \sum\seq{n} \frac{a_n}{2^n}, & x = \sum\seq{n}\frac{2a_n}{3^n} \in C \\ \sup_{y\leq x, y \in C}C(y) & x \not\in C \\ \end{cases}
\]

\thm (Properties of the Cantor function): The Cantor function has the following properties:

\begin{enumerate}

\item Continuous on $[0,1]$

\item Increasing on $[0,1]$

\item $C'(x) = 0$ for almost every $x \in [0, 1]$

\item $C(x)$ is not absolutely continuous in $[0,1]$.

\end{enumerate}

\proof\,

\begin{enumerate}

\item By definition, if we have a sequence $y_k = \sum\seq{n}\frac{2a_n^{(k)}}{3^n}$ converges to $x = \sum\seq{n}\frac{2b_n}{3^n}$, then we should have for all $N \in \N$, there exists $K_0 \in \N$ such that $a_n^{(k)} = b_n$ if $k \geq K_0$, $n = 1, \dots, N$. 

Thus 
\[
|C(y_k) - C(x)|=|\sum\seq{n}\frac{2a_n^{(k)}}{3^n} - \sum\seq{n}\frac{2b_n}{3^n}| = |\sum\seq{n}\frac{2(a_n^{(k)} - b_n)}{3^n} \leq \sum_{n = N + 1}^\oo \frac{4}{3^n} < \frac{4}{3^{N + 1}}
\]

We take the limit as $N \to \oo$ to get $|C(y_k) - C(x)| \to 0$

So the Cantor function is continuous on the Cantor set. 

\item By definition, $C(x)$ increases on the Cantor set. Now we show $C(x)$ increases on all of $[0, 1]$. 

Choose $x < y$. If $x, y \in C$, we are done. 

If $x \in C, C(y) = \sup_{z\leq y, z \in C}C(z) \geq C(x)$, because $x \in C, x \leq y$. 

If $x \not\in C$ and $y \in C,$ $C(x) = \sup_{z\leq x, z \in C}C(z) \leq \sup_{z\leq y, z \in C}C(z) \leq C(y)$, because $C(x)$ increases on $C$

If $x\not\in C$, $y\not\in C$, $C(x) = \sup_{z\leq x, z \in C}C(z) \leq \sup_{z\leq y, z \in C}C(z) = C(y)$

\item[1] Continued.

We want to show $C(x)$ is continuous on $[0,1]$. Assume $x_0 \in [0,1]$, need to prove $\lim_{x\to x_0}C(x) = C(x_0)$

\paragraph{Case 1:} Suppose $x_0\not\in C$. Thus $x_0 \in (a_k, b_k)$, where $(a_k, b_k)$ is an interval removed at some step in the construction of $C$. 

$C(x)$ is constant on any removed interval, so it is continuous on these (open) removed intervals. 

\paragraph{Case 2:} Suppose $x_0 \in C$. Assume $x_n \to x_0$, $x_n\not\in C$. Because $C(x)$ is increaseing, without loss of generality we can assume $x_n$ is increasing. If there exists $y_n \in C$ such that $x_n < y_n < x_0$, then we havec $C(x_0) = \lim_{n\to\oo}C(y_n)$. 

For all $n \in \N$, there exists $N_n\in\N$ such that $x_0 > x_{N_n} > y_n$. 

Thus $C(x_0) > C(x_{N_n}) \geq C(y_n) \to C(x_0)$

Thus $C(x_n)\to C(x_0)$ as $n\to\oo$.

If no such sequence $y_n$ exists, then there exists $N \in \N$ such that $(x_N, x_0) \subseteq [0,1]\setminus C$. Since $x_0 \in C$, $x_0$ iss the right endpoint of a removed interval. 

Thus $C(x)$ is left-continuous at $x_0$. 

In the same way, we prove that $C(x)$ is right-continuous. 

Therefore $C(x)$ is continuous on $[0, 1]$. 

\item $C'(x) = 0$ for almost every $x \in [0, 1]$. 

For every removed interval $(a_k, b_k) \subset [0, 1], C(x)$ is constant on $(a_k, b_k)$. 

Thus $C'(x) = 0$ for all $x \in [0, 1]\setminus C$. 

Since $\lambda([0,1]\setminus C_ = 1, C'(x) = 0$ for almost every $x \in [0, 1]$. 

\item $C(x)\not\in\AC[0,1]$. 

This follows from the fact that $0, 1 \in C$. 

It is easy to show $C(0) = 0, C(1) = 1$. 

Thus $\int_0^1C'(x)\,dx = 0 < C(1) - C(0),$ therefore $C\not\in\AC[0,1]$

\end{enumerate}

\qed





\end{document}