\documentclass[x11names,reqno,14pt]{extarticle}
\input{preamble}
\usepackage[document]{ragged2e}
\usepackage{amsmath}
\pagestyle{fancy}{
	\fancyhead[L]{Fall 2022}
	\fancyhead[C]{201A - Real Analysis}
	\fancyhead[R]{John White}
  
  \fancyfoot[R]{\footnotesize Page \thepage \ of \pageref{LastPage}}
	\fancyfoot[C]{}
	}
\fancypagestyle{firststyle}{
     \fancyhead[L]{}
     \fancyhead[R]{}
     \fancyhead[C]{}
     \renewcommand{\headrulewidth}{0pt}
	\fancyfoot[R]{\footnotesize Page \thepage \ of \pageref{LastPage}}
}

\newcommand{\seq}[1]{_{#1 = 1}^\oo}
\newcommand{\barr}{{\bar{\R}}}
\DeclareMathOperator{\Vol}{Vol}
\title{220A - Groups}
\author{John White}
\date{Fall 2022}




\begin{document}

\section*{Lecture 1}

Let $(X, \mc{A}, \mu)$ be a measure space. Without any additional structure or information, we may define the Lebesgue integral $\int_Xf\,d\mu$ for $f$ an $\mc{A}-\mc{B}$ measurable function $f:X\to[-\oo,+\oo]$. 

We only have a few examples without any work. 

\exm

\begin{itemize}

\item For any set $X$, we can define the counting measure on $\mc{A} = 2^X$, which gives $\mu(A) = |A|$. If $X = \N$, then a measurable function is just a sequence $(f_n)$, and $\int_Xf\,d\mu = \sum f_n$

\item We can also define the Dirac mass $\delta_p$ for a fixed $p \in X$ by 
\[
\delta_p(E) = \begin{cases} 1 & p \in E \\ 0 & p \not\in E\\ \end{cases}
\]
We have $\int_Xf\,d\delta_p = f(p)$

\end{itemize}

To get another example of a measure we need to do some work. 

\underline{Problem:} We want a measure $\mu$ on $\R^n$ such that, for a rectangle, 
\[
\mu([a_1,b_1]\times\cdots\times[a_n, b_n]) = |a_1 - b_1|\cdots|a_n - b_n|
\]
Once it is defined on all rectangles, it is defined on the minimal $\sigma$-algebra containing them, which is the Borel $\sigma$-algebra. In other words, this condition will completely specify a measure on the Borel $\sigma$-algebra $\mc{B}_{\R^n}$

If $X = \R^n$, or a general metric space, or even a general topological space, then $\mc{B}(X)$ denotes the $\sigma$-algebra generated by the open subsets of $X$. 

\underline{Problem:} 

Suppose we have a distribution function $F:\R\to\R$, meaning $F$ is monotone, positive, and $\lim_{x\to-\oo}f(x) = 0, \lim_{x\to\oo}f(x) = 1$, and continuous from the right. We want a Borel measure $\mu$ such that $F(t) = \mu((-\oo, t])$. Such a measure, denoted by $\lambda_F$, is called a Lebesgues-Stieltjes measure. 

The corresponding integral is called a Lebesgue-Stieltjes integral.

If $F$ is smooth, then $\int_{\R}\phi\,d\lambda_F = \int_{-\oo}^{\oo}\phi(x)d\,F(x)$.

The measure we want on $\R^n$ is denoted by $\lambda^n$. 

\subsection*{\underline{The Carath\'eodory Construction}}

Suppose we have an outer measure $\gamma:2^X\to[0,\oo]$. This means $\gamma(\varnothing) = 0$, $A \subset B \implies \gamma(A) \leq \gamma(B)$ (monotone), and $\gamma(\cup\seq{i}E_i) \leq \sum\seq{i}\gamma(E_i)$ (subadditive).

We can define a set $S$ to be $\gamma$-measurable if for every testing set $T$, $\gamma(T) = \gamma(S \cap T) + \gamma(S^c \cap T)$.

\thm (Carath\'eodory Extension Theorem)

\begin{enumerate}

\item $\gamma(N) = 0 \implies N$ is measurable. 

\item The set of measurable sets forms a $\sigma$-algebra $\Gamma$. 

\item $\gamma$ restricted to $\Gamma$ forms a measure. 

\end{enumerate}

``Nothing in the above theorem can guarantee you that $\Gamma$ is not trivial, i.e. $\Gamma = \{\varnothing, X\}$. Nevertheless, this is a very useful guy" - Dennis. 

\defn (Lebesgue outer measure on $\R^n$)

Let $R$ be a rectangle in $\R^n$, that is $R = \prod_{i=1}^n[a_i, b_i]$. We have $\Vol(R) = |a_1 - b_1|\cdots|a_n - b_n|$. For any $E \subseteq\R^n$, we define
\[
\mu^*(E) \eqdef \inf\{\sum\seq{j}\Vol(R_j) \mid E \subseteq \cup\seq{j}R_j\}
\] 

\prop

$\mu^*$ is an outer measure on $\R^n$ such that $\mu^*(R) = \Vol(R)$ for all rectangles $R$. 

\proof

The first and second axioms are trivial, so we will just prove the subadditivity. Let $E$ be some set. By definition, for any $\varepsilon$, there is some cover $R_j$ by recrtangles such that 
\[
-\varepsilon + \sum\seq{j}\Vol(R_j) \leq \mu^*(E) \leq \sum\seq{j}\Vol(R_j)
\]
meaning that $\sum\seq{j}\Vol(R_j) \leq \mu^*(E) + \varepsilon$. So for each $E_k$, there is a sequence $R^k_j$ which covers $E_k$, such that $\sum\seq{j}\Vol(R^k_j) \leq \mu^*(E) + \frac{\varepsilon}{2^k}$. 

So $\{R^k_j\}_{j, k \in \N}$ forms a cover of $\cup\seq{j}E_j$. Thus

\begin{align*}
\mu^*(\cup\seq{k}E_k) & \leq \sum\seq{k}\sum\seq{j}\Vol(R_j^k) \\
& \leq \sum\seq{k}\left(\mu^*(E_k) + \frac{\varepsilon}{2^k}\right) \\
& = \sum\seq{k}\mu^*(E_k) + \varepsilon
\end{align*}
This is true for any positive $\varepsilon$. Taking the limit as $\varepsilon\to0$ gives the result. 

\qed

Now, fix a rectangle $R$. Note that $R$ itself forms a cover of $R$, so by the definition, $\mu^*(R) \leq \Vol(R)$. For $\varepsilon > 0$, we can take an almost-optimal cover $(R_j)$ such that $\sum\seq{j}\Vol(R_j) \leq \Vol(R) + \varepsilon$. We can rig it such that $|\Vol(R_j) - \Vol(R)| \leq \frac{\varepsilon}{2^j}$.  Because $R \subset\cup\seq{j}R_j$, and $R_j$ is an open cover, by compactness of $R$ there is a finite subcover, and the volume of $R$ is less than or equal to the sum of the volumes of these finitely many $R_j$. So the volume of $R$ is less than or equal to $\mu^*(R) + 2\varepsilon$. So $\Vol(R) = \mu^*(R)$.

\prop 

Every rectangle $R$ in $\R^n$ is Carath\'eodory measurable). 

\proof

I missed this lol. Apparently Dennis denotes $\mc{M}_{\lambda^*}$ by $\ms{L}^n$.

\qed

\defn

A set is said to be \underline{$G_\delta$} if it is the countable intersection of open sets. A set is said to be \underline{$F_\sigma$} if it is the countable union of closed sets. 

\thm

\begin{enumerate}

\item For all $E \in \ms{L}^n$, $\lambda^N(E) = \inf\{\lambda^n(O) \mid \text{open }O\supseteq E\}$. 

\item $E \in \ms{L}^n$ if and only if $E = H\setminus Z$, where $H$ is $G_\delta$, and $\lambda^*(Z) = 0$. 

\item $E \in \ms{L}^n$ if and only if $E = H \cup Z$, where $H$ is $F_\sigma$ and $\lambda^*(Z) = 0$. 

\item $\lambda^n(E) = \sup\{\lambda^n(C) \mid \text{closed }C \subseteq E\}$

\end{enumerate}

\proof

It suffices to prove the first statement, as the others will follow by passing to a complement. 

\qed

\defn

Suppose $X$ is a metric space. A measure on $X$ is a \underline{Radon measure} if it is Borel (meaning defined on a $\sigma$-algebra containing Borel sets), and for any Borel $E$, $\mu(E) = \inf\{\mu(O) \mid \text{open }O \supseteq E\}$, and for any compact $C\subseteq X$, $\mu(C)<\oo$. 

\thm (Riesz)

Let $X \subseteq \R^n$ be compact. Let $C(X)$ denote the vector space of all continuous functions on $X$. This admits a norm $\norm{f}_{C(X)} = \sup_X|f|$, making it a Banach space. Define $C^*(X) = \{\phi:C(X)\to\R, \phi$ is linear and continuous $\}$. 

For all $\phi \in C^*(X)$, there exists a Radon measure $\mu = \mu_{\phi}$, and a function $M:X\to\{\pm1\}$ which is Borel, such that
\[
\phi(f) = \int_Xf(x)M(x)\,d\mu_{\phi}(x)
\]

for all $f \in C(X)$. 












\end{document}