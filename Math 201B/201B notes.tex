\documentclass[x11names,reqno,14pt]{extarticle}
\input{preamble}
\usepackage[document]{ragged2e}
\usepackage{amsmath}
\pagestyle{fancy}{
	\fancyhead[L]{Spring 2023}
	\fancyhead[C]{201B - Real Analysis}
	\fancyhead[R]{John White}
  
  \fancyfoot[R]{\footnotesize Page \thepage \ of \pageref{LastPage}}
	\fancyfoot[C]{}
	}
\fancypagestyle{firststyle}{
     \fancyhead[L]{}
     \fancyhead[R]{}
     \fancyhead[C]{}
     \renewcommand{\headrulewidth}{0pt}
	\fancyfoot[R]{\footnotesize Page \thepage \ of \pageref{LastPage}}
}

\newcommand{\seq}[1]{_{#1 = 1}^\oo}
\newcommand{\barr}{{\bar{\R}}}
\newcommand{\cupk}{\cup\seq{k}}
\newcommand{\cupi}{\cup\seq{i}}
\newcommand{\cupn}{\cup\seq{n}}
\newcommand{\bigcupk}{\bigcup\seq{k}}
\newcommand{\bigcupi}{\bigcup\seq{i}}
\newcommand{\bigcupn}{\bigcup\seq{n}}
\newcommand{\capk}{\cap\seq{k}}
\newcommand{\capi}{\cap\seq{i}}
\newcommand{\capn}{\cap\seq{n}}
\newcommand{\bigcapk}{\bigcap\seq{k}}
\newcommand{\bigcapi}{\bigcap\seq{i}}
\newcommand{\bigcapn}{\bigcap\seq{n}}
\DeclareMathOperator{\Vol}{Vol}
\title{220A - Groups}
\author{John White}
\date{Fall 2022}




\begin{document}

\section*{Lecture 1}

Let $(X, \mc{A}, \mu)$ be a measure space. Without any additional structure or information, we may define the Lebesgue integral $\int_Xf\,d\mu$ for $f$ an $\mc{A}-\mc{B}$ measurable function $f:X\to[-\oo,+\oo]$. 

We only have a few examples without any work. 

\exm

\begin{itemize}

\item For any set $X$, we can define the counting measure on $\mc{A} = 2^X$, which gives $\mu(A) = |A|$. If $X = \N$, then a measurable function is just a sequence $(f_n)$, and $\int_Xf\,d\mu = \sum f_n$

\item We can also define the Dirac mass $\delta_p$ for a fixed $p \in X$ by 
\[
\delta_p(E) = \begin{cases} 1 & p \in E \\ 0 & p \not\in E\\ \end{cases}
\]
We have $\int_Xf\,d\delta_p = f(p)$

\end{itemize}

To get another example of a measure we need to do some work. 

\underline{Problem:} We want a measure $\mu$ on $\R^n$ such that, for a rectangle, 
\[
\mu([a_1,b_1]\times\cdots\times[a_n, b_n]) = |a_1 - b_1|\cdots|a_n - b_n|
\]
Once it is defined on all rectangles, it is defined on the minimal $\sigma$-algebra containing them, which is the Borel $\sigma$-algebra. In other words, this condition will completely specify a measure on the Borel $\sigma$-algebra $\mc{B}_{\R^n}$

If $X = \R^n$, or a general metric space, or even a general topological space, then $\mc{B}(X)$ denotes the $\sigma$-algebra generated by the open subsets of $X$. 

\underline{Problem:} 

Suppose we have a distribution function $F:\R\to\R$, meaning $F$ is monotone, positive, and $\lim_{x\to-\oo}f(x) = 0, \lim_{x\to\oo}f(x) = 1$, and continuous from the right. We want a Borel measure $\mu$ such that $F(t) = \mu((-\oo, t])$. Such a measure, denoted by $\lambda_F$, is called a Lebesgues-Stieltjes measure. 

The corresponding integral is called a Lebesgue-Stieltjes integral.

If $F$ is smooth, then $\int_{\R}\phi\,d\lambda_F = \int_{-\oo}^{\oo}\phi(x)d\,F(x)$.

The measure we want on $\R^n$ is denoted by $\lambda^n$. 

\subsection*{\underline{The Carath\'eodory Construction}}

Suppose we have an outer measure $\gamma:2^X\to[0,\oo]$. This means $\gamma(\varnothing) = 0$, $A \subset B \implies \gamma(A) \leq \gamma(B)$ (monotone), and $\gamma(\cup\seq{i}E_i) \leq \sum\seq{i}\gamma(E_i)$ (subadditive).

We can define a set $S$ to be $\gamma$-measurable if for every testing set $T$, $\gamma(T) = \gamma(S \cap T) + \gamma(S^c \cap T)$.

\thm (Carath\'eodory Extension Theorem)

\begin{enumerate}

\item $\gamma(N) = 0 \implies N$ is measurable. 

\item The set of measurable sets forms a $\sigma$-algebra $\Gamma$. 

\item $\gamma$ restricted to $\Gamma$ forms a measure. 

\end{enumerate}

``Nothing in the above theorem can guarantee you that $\Gamma$ is not trivial, i.e. $\Gamma = \{\varnothing, X\}$. Nevertheless, this is a very useful guy" - Dennis. 

\defn (Lebesgue outer measure on $\R^n$)

Let $R$ be a rectangle in $\R^n$, that is $R = \prod_{i=1}^n[a_i, b_i]$. We have $\Vol(R) = |a_1 - b_1|\cdots|a_n - b_n|$. For any $E \subseteq\R^n$, we define
\[
\mu^*(E) \eqdef \inf\{\sum\seq{j}\Vol(R_j) \mid E \subseteq \cup\seq{j}R_j\}
\] 

\prop

$\mu^*$ is an outer measure on $\R^n$ such that $\mu^*(R) = \Vol(R)$ for all rectangles $R$. 

\proof

The first and second axioms are trivial, so we will just prove the subadditivity. Let $E$ be some set. By definition, for any $\varepsilon$, there is some cover $R_j$ by recrtangles such that 
\[
-\varepsilon + \sum\seq{j}\Vol(R_j) \leq \mu^*(E) \leq \sum\seq{j}\Vol(R_j)
\]
meaning that $\sum\seq{j}\Vol(R_j) \leq \mu^*(E) + \varepsilon$. So for each $E_k$, there is a sequence $R^k_j$ which covers $E_k$, such that $\sum\seq{j}\Vol(R^k_j) \leq \mu^*(E) + \frac{\varepsilon}{2^k}$. 

So $\{R^k_j\}_{j, k \in \N}$ forms a cover of $\cup\seq{j}E_j$. Thus

\begin{align*}
\mu^*(\cup\seq{k}E_k) & \leq \sum\seq{k}\sum\seq{j}\Vol(R_j^k) \\
& \leq \sum\seq{k}\left(\mu^*(E_k) + \frac{\varepsilon}{2^k}\right) \\
& = \sum\seq{k}\mu^*(E_k) + \varepsilon
\end{align*}
This is true for any positive $\varepsilon$. Taking the limit as $\varepsilon\to0$ gives the result. 

\qed

Now, fix a rectangle $R$. Note that $R$ itself forms a cover of $R$, so by the definition, $\mu^*(R) \leq \Vol(R)$. For $\varepsilon > 0$, we can take an almost-optimal cover $(R_j)$ such that $\sum\seq{j}\Vol(R_j) \leq \Vol(R) + \varepsilon$. We can rig it such that $|\Vol(R_j) - \Vol(R)| \leq \frac{\varepsilon}{2^j}$.  Because $R \subset\cup\seq{j}R_j$, and $R_j$ is an open cover, by compactness of $R$ there is a finite subcover, and the volume of $R$ is less than or equal to the sum of the volumes of these finitely many $R_j$. So the volume of $R$ is less than or equal to $\mu^*(R) + 2\varepsilon$. So $\Vol(R) = \mu^*(R)$.

\prop 

Every rectangle $R$ in $\R^n$ is Carath\'eodory measurable). 

\proof

I missed this lol. Apparently Dennis denotes $\mc{M}_{\lambda^*}$ by $\ms{L}^n$.

\qed

\defn

A set is said to be \underline{$G_\delta$} if it is the countable intersection of open sets. A set is said to be \underline{$F_\sigma$} if it is the countable union of closed sets. 

\thm

\begin{enumerate}

\item For all $E \in \ms{L}^n$, $\lambda^N(E) = \inf\{\lambda^n(O) \mid \text{open }O\supseteq E\}$. 

\item $E \in \ms{L}^n$ if and only if $E = H\setminus Z$, where $H$ is $G_\delta$, and $\lambda^*(Z) = 0$. 

\item $E \in \ms{L}^n$ if and only if $E = H \cup Z$, where $H$ is $F_\sigma$ and $\lambda^*(Z) = 0$. 

\item $\lambda^n(E) = \sup\{\lambda^n(C) \mid \text{closed }C \subseteq E\}$

\end{enumerate}

\proof

It suffices to prove the first statement, as the others will follow by passing to a complement. 

\qed

\defn

Suppose $X$ is a metric space. A measure on $X$ is a \underline{Radon measure} if it is Borel (meaning defined on a $\sigma$-algebra containing Borel sets), and for any Borel $E$, $\mu(E) = \inf\{\mu(O) \mid \text{open }O \supseteq E\}$, and for any compact $C\subseteq X$, $\mu(C)<\oo$. 

\thm (Riesz)

Let $X \subseteq \R^n$ be compact. Let $C(X)$ denote the vector space of all continuous functions on $X$. This admits a norm $\norm{f}_{C(X)} = \sup_X|f|$, making it a Banach space. Define $C^*(X) = \{\phi:C(X)\to\R, \phi$ is linear and continuous $\}$. 

For all $\phi \in C^*(X)$, there exists a Radon measure $\mu = \mu_+$, and a function $M:X\to\{\pm1\}$ which is Borel, such that
\[
\phi(f) = \int_Xf(x)M(x)\,d\mu(x)
\]
for all $f \in C(X)$. 

\proof

\qed

\section*{Lecture 2, 1/17/23}

Note: This is the first lecture with Davit. Davit will always use $\mu$ to refer to an \underline{outer} measure, not a measure. The book will be ``Measure theory and fine properties of functions." According to Davit, this is the correct book to be using. 

\defn 

Let $X$ be a nonempty set. A mapping $\mu:2^X\to[0,+\oo]$ is called a \underline{measure} if it satisfies the following 2 properties.

\begin{enumerate}
\item $\mu(\varnothing) = 0$. 
\item (Countable subadditivity and monotonicity) If $A, A_1, A_2, \dots \subseteq X$ and $A \subseteq \cup\seq{i}A_i$ then $\mu(A)\leq\sum\seq{i}\mu(A_i)$
\end{enumerate}

\rem From the second definition, we can automatically get monotonicity, i.e. if $A\subseteq B$, then $\mu(A)\leq\mu(B)$. This is because, as written, the second definition is a statement not just about $\cup\seq{i}A_i$, but about any subset of it. Indeed, let $A = A, A_1 = B$, and $A_k = \varnothing$ for $k\geq 2$. Then we have $\mu(A)\leq\mu(\cup\seq{i}A_i) = \mu(A\cup B)$. 

We will write ``$\mu$ is a measure on $X$" to mean that $\mu$ satisfies the above definition (that is, $\mu$ is an outer measure).

\defn

Let $X$ be a nonempty set and let $\mu$ be a measure on $X$. For a fixed set $C\subseteq X$, define the \underline{restriction measure} $\nu = \mu|_C$ by $\nu(A) = \mu|_A(A) = \mu(A\cap C)$.

\rem It is easy to prove that $\mu|_C$ is a measure on $X$. 

\defn (Carath\'eodory's condition). Let $X$ be a nonempty set and let $\mu$ be a measure on $X$. A subset $A\subseteq X$ is called \underline{$\mu$-measurable} if, for all subset $B\subseteq X$, we have
\[
\mu(B) = \mu(B \cap A) + \mu(B \setminus A)
\]

\rem $X$ and $\varnothing$ are easily seen to be $\mu$-measurable. 

\thm (Carath\'eodory extension theorem)

The collection of $\mu$-measurable sets on a set $X$ is a $\sigma$-algebra.

\thm Let $X$ be a nonempty set and let $\mu$ be a measure on $X$. Then the following holds: 

\begin{enumerate}
\item $\varnothing$ and $X$ are $\mu$-measurable.
\item $A \subseteq X$ is $\mu$-measurable if and only if $X \setminus A$ is $\mu$-measurable.
\item If $A\subseteq X$ is such that $\mu(A) = 0$, then $A$ is $\mu$-measurable. 
\item Let $C \subseteq X$. Then anything which is $\mu$-measurable is $\mu|_C$-measurable. 
\end{enumerate}

\rem

A measure is also finitely subadditive, which says that if $A \subseteq A_1 \cup \cdots \cup A_n$, then $\mu(A) \leq \sum_{i=1}^n\mu(A_i)$. So, to check that $\mu(B) = \mu(B \cap A) + \mu(B\setminus A)$, it will suffice to check
\[
\mu(B)\geq\mu(B\cap A) + \mu(B\setminus A)
\]

\proof

Part 1 is obvious. 

Suppose that $A$ is $\mu$-measurable. Then $\mu(B\cap A) = \mu(B \setminus A^c)$ and $\mu(B\cap A^c) = \mu(B\setminus A)$ so $\mu(B \cap A) + \mu(B\setminus A) = \mu(B \cap A^c) + \mu(B\setminus A^c)$. So $A$ is $\mu$-measurable if and only if $A^c$ is.

Suppose that $\mu(A) = 0$. Then $\mu(B\cap A) \leq \mu(A), \mu(B)$, so $\mu(B\cap A) = 0$ for any $B\subseteq X$. Now, $B\setminus A \subseteq B$, so by monotonicity $\mu(B\setminus A)\leq\mu(B)$. So $\mu(B \cap A) + \mu(B\setminus A) \leq \mu(B)$ for all $B \subseteq X$, so we are done. 

Let $A$ be $\mu$-measurable. Then for any $B\subseteq X$ we have 
\begin{align*}
\nu(B) & = \mu|_C(B) = \mu(B \cap C) \\
		 & = \mu((B\cap C) \cap A) + \mu((B\cap C)\setminus A) \\
		 & = \nu(B \cap A) + \mu((B\setminus A) \cap C)\\
		 & = \nu(B \cap A) + \nu(B\setminus A) \\
\end{align*}

\qed

\thm

Let $X$ be a nonempty set and let $\mu$ be a measure on $X$. 

Assume $A_1, A_2, \dots, A_n\subseteq X$ are $\mu$-measurable. Then
\begin{enumerate}
\item $\bigcup_{k=1}^n A_k$ and $\bigcap_{i=1}^nA_k$ are also $\mu$-measurable. 
\item If the $A_i$ are disjoint, then $\mu(\bigcup_{i=1}^nA_i = \sum_{i=1}^n\mu(A_i)$
\end{enumerate}

\proof

We prove part 2 first. Because each $A_i$ is measurable, 
\begin{align*}
\mu(\cup_{k=1}^nA_k) & = \mu((\cup_{k=1}^nA_k)\cap A_n) + (\mu(\cup_{k=1}^nA_k) \setminus A_n) \\
& = \mu(\cup_{i=1}^{n - 1}A_k) + \mu(A_n) = \cdots = \sum_{k=1}^n\mu(A_k) \\
\end{align*}

Now we prove part 1. Let $A, B \subseteq X$ be $\mu$-measurable and disjoint. Then for any $C\subseteq X$, $\mu(C) = \mu(C \cap A) + \mu(C\setminus A)$, and similarly for $B$. This is equal to 

\begin{align*}
\mu(C) & = \mu(C \cap A) + \mu((C\setminus A) \cap B) + \mu(C\setminus A \setminus B) \\
		 & = \mu(C\cap A) + \mu(C\cap B) + \mu(C\setminus(A\cup B)) + \mu(C\cap(A\cup B)) (?)\\
		 & = \mu(C \cap (A \cup B) \cap A) + \mu(C \cap (A \cup B) \setminus A) \\
		 & = \mu(C \cap A) + \mu(C\cap B) \\
		 & = \mu(C \cap (A \cup B)) + \mu(C\setminus(A \cup B))
\end{align*}

So $A \cup B$ is $\mu$-measurable. (I got a bit lost in the arithmetic, sorry)

Next, we show if $A, B\subseteq X$ are $\mu$-measurable, then $A \cap B$ is $\mu$-measurable. This is straightforward. We will continue next time. 

\section*{Lecture 3, 1/19/23}

We will continue our proof of the theorem. Assume $A, B \subseteq X$ are $\mu$-measurable. We aim to show that $A \cap B$ is also $\mu$-measurable. We need to show that, for any $C \subseteq X$, we have $\mu(C) = \mu(C\cap(A\cap B)) + \mu(C\setminus(A\cap B))$. Because $A$, $B$ are $\mu$-measurable, we have
\begin{align*}
\mu(C) &  = \mu(C \cap A) + \mu(C\setminus A) \\
		 & = \mu((C\cap A) \cap B) + \mu((C\cap A) \setminus B) + \mu(C\setminus A) \\
		 & = \mu(C\cap (A \cap B)) + \mu((C\cap A) \setminus B) + \mu(C\setminus A) \\
		 & \geq \mu(C\cap (A \cap B)) + \mu(C\setminus(A \cap B))
\end{align*}
The opposite inequality follows by subadditivity, so we have equality. 

By induction, we get that also $\cap_{k=1}^n A_k$ is $\mu$-measurable. For the union, we can get it using the fact that $\cup_{k=1}^nA_k = X \setminus \cap_{k=1}^nA_k$.

\qed

\rem If $A, B$ are $\mu$-measurable, then $A \setminus B$ is $\mu$-measurable. This follows from $A \setminus B = A \cap (X \setminus B)$

\thm 

Let $X$ be a nonempty set, and $\mu$ a measure on $X$. Assume $\{A_k\}\seq{k}\subseteq X$ are $\mu$-measurable. Then
\begin{enumerate}
\item If the $A_k$ are disjoint, then we have countable additivity:
\[
\mu\left(\cup\seq{k}A_k\right) = \sum\seq{k}\mu(A_k)
\]
\end{enumerate}
\item If $A_1 \subseteq A_2 \subseteq \cdots$, then 
\[
\lim_{k\to\oo}\mu(A_k) = \mu\left(\bigcup\seq{k}A_k\right)
\]
\item If $A_1 \supseteq A_2 \supseteq \cdots$, and $\mu(A_1) < \oo$, then 
\[
\lim_{k\to\oo}\mu(A_k) = \mu\left(\bigcap\seq{k}A_k\right)
\]


\proof

We have from before that if the $A_k$ are pairwise disjoint, then $\mu(\cup_{k=1}^nA_k) = \sum_{k=1}^n\mu(A_k)$ for any $n \in \N$. Because $\cup_{k=1}^nA_k \subseteq \cup\seq{k}A_k$, we must have that $\mu(\cup_{k=1}^nA_k) \leq \mu(\cup\seq{k}A_k)$. Using the previous fact, and passing to a limit, we have
\[
\sum\seq{k}\mu(A_k) \leq \mu(\cup\seq{k}A_k)
\]
The opposite equality is automatically true by the countable subadditivity of $\mu$, so we get equality. This completes the proof of 1. Now for part 2. 

Define $B_k = A_k \setminus A_{k - 1}$, where $A_0 \eqdef \varnothing$. We have $A_k = \cup_{i=1}^kB_i$. Note that the $B_i$ are disjoint. So we have
\[
\mu(A_k) = \sum_{i=1}^k\mu(B_i)
\]
So, in the limit, 
\[
\lim_{k\to\oo}\mu(A_k) = \lim_{k\to\oo}\sum_{i=1}^k\mu(B_i) = \sum\seq{i}\mu(B_i)
\]
So
\[
\mu(\cup\seq{i}B_i) = \mu(\cup\seq{k}A_k)
\]

Finally, let $A_1 \supseteq A_2 \supseteq \cdots, \mu(A_1)<\oo$. Define $B_k = A_1\setminus A_k$. This is decreasing sequence of $\mu$-measurable sets, so by the previous part, 
\[
\lim_{k\to\oo}\mu(B_k) = \mu(\cup\seq{k}B_k)
\]
So
\begin{align*}
\mu(B_k) = \mu(A_1\setminus A_k)  = \mu(A_1) - \mu(A_k) \implies \\
\lim_{k\to\oo}\mu(B_k) = \lim_{k\to\oo}(\mu(A_1) - \mu(A_k)) & = \mu(A_1) - \lim_{k\to\oo}\mu(A_k) \\
= \mu(\cup\seq{k}B_k) = \mu(\cup\seq{k}(A_1 \setminus A_k)) & = \mu(A_1 \setminus \cap\seq{k} A_k) \\
\geq \mu(A_1) - \mu(\cap\seq{k}A_k) & \\
\end{align*}

So $\lim_{k\to\oo}\mu(A_k) \leq \mu(\cap\seq{k}A_k)$. The opposite inequality follows easily by monotonicity. 

\qed

We are ready to prove the Carath\'eodory extension theorem. 

\proof

Let $A_1, A_2, \dots \subseteq X$ be $\mu$-measurable. It will suffice to prove that $\cup\seq{k}A_k$ is $\mu$-measurable. So we need to check that, for any $B$, 
\[
\mu(B) = \mu(B\cap (\cup\seq{k}A_k)) + \mu(B \setminus \cup\seq{k}A_k)
\]
Fix $B \subseteq X$, and consider $\nu = \mu|_B$. Recall this is defined as $\nu(C) = \mu(B \cap C)$. We would like 
\[
\nu(B) = \nu(\cup\seq{k}) + \nu(B \setminus \cup\seq{k}A_k)
\]
To this end, 
\begin{align*}
\nu(\cup\seq{k}A_k) & = \lim_{k\to\oo}\nu(\cup_{i=1}^kA_i) \\
\end{align*}
Without loss of generality, $\nu(B)<\oo$. If $\nu(B) = \oo$, then we are done trivially. 

As before, define $B_k = B\setminus \cup_{i=1}^kA_i$. Then
\[
\nu(B\setminus\cup\seq{k}A_k) = \lim_{k\to\oo}\nu(B\setminus\cup_{i=1}^kA_i)
\]
So, 
\begin{align*}
\nu(\cup\seq{k}A_k) + \nu(B\setminus\cup\seq{k}A_k) & = \lim_{k\to\oo}\left(\nu(\cup_{i=1}^kA_i) + \nu(B\setminus\cup_{i=1}^kA_i)\right) \\
& = \lim_{k\to\oo}\nu(B) = \nu(B) \\
\end{align*}
so we are done. 

\qed

\defn

Let $X$ be a nonempty set, and let $\mu$ be a measure on $X$. Then $\mu$ is said to be 
\begin{enumerate}
\item A \underline{regular measure} if, for any $A \subset X$, there exists a $\mu$-measurable $B\subseteq X$ such that $A \subseteq B$, and $\mu(A) = \mu(B)$. 
\item A \underline{Borel measure} if all Borel sets (i.e. the elements of the Borel $\sigma$-algebra) are measurable. This only applies if $X$ is also a topological space, of course. 
\item A \underline{Borel-regular measure} if $\mu$ is Borel, and for any $A \subseteq X$, there exists a Borel set $B \subseteq X$ such that $A \subseteq B$ and $\mu(A) = \mu(B)$. 
\item A \underline{Radon measure} if it is Borel-regular and $\mu(K)<\oo$ if $K$ is compact. 
\end{enumerate}

\rem

Note that being Borel and regular is weaker than being Borel-regular. 

\thm (Increasing sets \underline{for regular measures})
Let $X$ be a nonempty set, and let $\mu$ be a regular measure on $X$. Assume $A_1\subseteq A_2 \subseteq \cdots \subseteq X$. Then 
\[
\lim_{k\to\oo}\mu(A_k) = \mu\left(\bigcupk A_k\right)
\]

\rem The sets $A_k$ need not be $\mu$-measurable. 

\proof

For all $A_k$, there is a $C_k\subseteq X$ which is $\mu$-measurable, $A_k \subseteq C_k$, and $\mu(A_k) = \mu(C_k)$. Let $D_k = \cap_{i\geq k}C_i$. For $i \geq k$, we can see $A_k \subseteq A_i \subseteq C_i$. $A_k \subseteq \cup_{i\geq k}C_i = D_k$, then $\mu(A_k)\leq\mu(D_k)$. On the other hand, $D_k\subseteq C_k$, so $\mu(D_k)\leq\mu(C_k) = \mu(A_k)$. So
\begin{itemize}
\item $\mu(A_k) = \mu(D_k)$ 
\item $A_k \subseteq D_k$ 
\item $D_k$ is $\mu$-measurable and $D_1 \subseteq D_2 \subseteq \cdots $
\end{itemize}
So
\begin{align*}
\lim_{k\to\oo}\mu(A_k) & = \lim_{k\to\oo}\mu(D_k) \\
						      & = \mu\left(\bigcupk D_k\right) \\
								& \geq \mu\left(\bigcupk A_k\right) \\
\end{align*}
Because $\cupk A_k \subseteq \cupk D_k$, 
\[
\lim_{k\to\oo}\mu(A_k)\geq\mu\left(\bigcupk A_k\right) 
\]
But $A_k \subseteq \cupk A_k$, so the opposite inequality is also true, so we have equality. 

\qed

\section*{Lecture 4, 1/24/23}

\thm (Restriction and Radon measures)

Let $X$ be a topological space and let $\mu$ be a Borel-regular measure on $X$. Let $A \subseteq X$ be $\mu$-measurable with $\mu(A)<\oo$. Then the restriction measure $\nu = \mu|_A$ is Radon. 

\proof

First, $\nu$ is a finite measure, as $\nu(X) = \mu(A \cap X) = \mu(A) < \oo$ for any $C \subseteq X$. 

It is clear that $\nu$ is Borel, as $\mu$ is Borel. Next, we show $\nu$ is Borel-Regular. Without loss of generality, we may assume that $A$ is Borel, because $\mu$ is Borel-regular. Explicitly, we know there is a Borel set $B \subseteq X$ such that $A \subseteq B$ and $\mu(B) = \mu(A)$. We will show $\mu|_A = \mu|_B$. 

We have $\mu(B) = \mu(B \cap A) + \mu(B \setminus A) = \mu(A) + \mu(B \setminus A)$. So $\mu(B\setminus A) = 0$. 

So, for all $C \subseteq X$, 
\begin{align*}
\mu|_B(C) & = \mu(B \cap C)\\
			  & = \mu((B\cap C)\cap A) + \mu((B \cap C) \setminus A) \\
			  & = \mu(C \cap A) + \mu((B \cap C) \setminus A ) \\
			  & \leq \mu|_A(C) + \mu(B\setminus A) \\
			  & = \mu|_A(C) \\
\end{align*}
But $(A \cap C) \subseteq (B \cap C)$, so $\mu|_A(C) \leq \mu|_B(C)$, so we may conclude that $\mu|_A = \mu|_B$. 

So assume $A$ is Borel. Fix $C \subseteq X$. We need to prove that there exists a Borel $D\subseteq X$ such that $C\subseteq D$ and $\nu(C) = \nu(D)$. There exists a Borel $E \subseteq X$ such that $C \cap A \subseteq E$, and $\mu(C \cap A) = \mu(E)$. So $D = E \cup (X\setminus A)$ is Borel and $C \subseteq D$. 

So 
\begin{align*}
\nu(D) & = \mu((E\cup(X\setminus A))\cap A)\\ &  = \mu(E \cap A) \\ & \leq\mu(E) \\
& = \mu(C\cap A) \\
& = \nu(C) \\
\end{align*}
$C \subseteq D$ so $\nu(C)\leq\nu(D)$, so $\nu(C)=\nu(D)$.

\qed

\thm (Carath\'eodory Criterion for being Borel) 

Let $X$ be a metric space and let $\mu$ be a measure on $X$. Then $\mu$ is Borel if and only if, for all $A, B \subseteq X$ with $d(A, B) > 0$ (meaning $\inf\{d(a, b) \mid a \in A, b \in B\} > 0$), 
\[
\mu(A \cup B) = \mu(A) + \mu(B)
\]

\proof

\subsection*{$=>$}

Suppose $\mu$ is Borel. We will use $\bar{B}$ to denote the closure of $B$. Then $d(A,\bar{B}) = d(A, B) > 0$. By measurability of $\bar{B}$, 
\[
\mu(A \cup B) = \mu((A \cup B) \cap \bar{B}) + \mu((A \cup B) \setminus \bar{B}) = \mu(B) = \mu(A)
\]

\subsection*{$<=$}

Suppose that, for $A, B$ with $d(A, B) > 0$, $\mu(A \cup B) = \mu(A) + \mu(B)$. We will show that this implies $\mu$ is Borel. Let us show that every closed subset $C\subseteq X$ is $\mu$-measurable. So we have to prove that for every $A \subseteq X$, 
\[
\mu(A) = \mu(A \cap C) + \mu(A\setminus C)
\]
We have $\leq$ trivially. Assume $\mu(A)<\oo$; otherwise, this equality holds trivially. 

Define for every $n \in \N$ the set $C_n = \{x\in X \mid d(x, C) \leq \frac{1}{n}\}$. We can see $d(A\setminus C_n, C) \geq \frac{1}{n} > 0$. So
\begin{align*}
\mu({(A\setminus C_n)\cup (A \cap C)}_{\subseteq A}) & = \mu(A \setminus C_n) + \mu(A \cap C) \\
& \leq \mu(A)\\
\end{align*}
SO $\mu(A \setminus C_n) + \mu(A \cap C) \leq \mu(A)$ for all $n \in \N$. We will prove that $\lim_{n\to\oo}\mu(A\setminus C_n) = \mu(A \setminus C)$. 

Consider the annuli $R_n = \{x \in A \mid \frac{1}{n + 1} < d(x, C) \leq \frac{1}{n}$. We have
\[
(A\setminus C_1) \bigcupn R_n \subseteq A \setminus C
\]
$C$ is closed, so in fact we have equality above. Why? If a point belongs to $A \setminus C$, then it does not belong to $C$, so $d(x, C) > 0$. So there is an $n\in\N$ such that $x \in R_n$ or $x \in A \setminus C_1$. We have
\begin{align*}
\mu\left(\bigcup_{k=0}^nR_{2k + 1}\right) & = \sum_{k=0}^n\mu(R_{2k + 1})  \leq \mu(A) \\
\mu\left(\bigcup_{k=1}^nR_{2k}\right) & = \sum_{k=1}^n\mu(R_{2k}) \leq \mu(A) \\
\end{align*}

So $\sum\seq{n}\mu(R_n) \leq 2\mu(A)<\oo$, so $\lim_{n\to\oo}(\sum_{k=n}^\oo\mu(R_k)) = 0$. So $(A \setminus C_n) \bigcup_{k=n}^\oo R_k = A \setminus C$

So by subadditivity, 
\[
\mu(A\setminus C) \leq \mu(A\setminus C_n) + \sum_{k=n}^\oo\mu(R_k)
\]
So as $n\to\oo$, 
\[
\mu(A \setminus C) \leq \liminf_{n\to\oo}\mu(A\setminus C_n) \leq \mu(A \setminus C)
\]
This completes the proof. 

It is time for our third section. 

\subsection*{\underline{Approximation by open, closed, and compact sets}}

\thm Let $\mu$ be a Borel measure on $\R^n$, and let $B \subseteq \R^n$ be a Borel set. 

\begin{enumerate}

\item If $\mu(B)<\oo$, then for any $\varepsilon>0$, there exists a closed $C \subseteq B$ such that $\mu(B\setminus C) < \varepsilon$. 

\item If $\mu$ is a Radon measure, then for all $\varepsilon>0$, there exists an open $U \supseteq B$ such that $\mu(U\setminus B) < \varepsilon$. 

\end{enumerate}

\proof

\begin{enumerate}

\item Let $\nu = \mu|_B$, a finite measure on $\R^n$. 

Define the collection $\ms{F} = \{A\subseteq\R^n\mid A$ is $\mu$-measurable and for all $\varepsilon > 0, $ there exists a closed $C \subseteq  A$ such that $\mu(A\setminus C) < \varepsilon \}$

Our goal is to show that $\ms{B}_{\R^n}\subseteq\ms{F}$. Davit uses ``$\sigma_B$" to indicate the Borel $\sigma$-algebra. 

By previous discussion, $\ms{F}$ contains all closed sets. 

Now, if $A_1, A_2, \dots \in \ms{F}$, then $\bigcapk A_k\in\ms{F}$. For all $A_k$, there exists a closed $C_k \subseteq A_k$, such that $\nu(A_k\setminus C_k) < \frac{\varepsilon}{2^k}$. Then by subadditivity, 
\[
\nu(\bigcapk A_k \setminus \bigcapk C_k) \leq \nu(\bigcupk(A_k \setminus C_k)) \leq \sum\seq{k}\nu(A_k\setminus C_k) < \varepsilon
\]
and $C = \bigcapk C_k$ is closed. 


\end{enumerate}


 


\end{document}